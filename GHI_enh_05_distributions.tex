% Options for packages loaded elsewhere
\PassOptionsToPackage{unicode}{hyperref}
\PassOptionsToPackage{hyphens}{url}
\PassOptionsToPackage{dvipsnames,svgnames,x11names}{xcolor}
%
\documentclass[
  10pt,
  a4paper,oneside]{article}
\usepackage{amsmath,amssymb}
\usepackage{iftex}
\ifPDFTeX
  \usepackage[T1]{fontenc}
  \usepackage[utf8]{inputenc}
  \usepackage{textcomp} % provide euro and other symbols
\else % if luatex or xetex
  \usepackage{unicode-math} % this also loads fontspec
  \defaultfontfeatures{Scale=MatchLowercase}
  \defaultfontfeatures[\rmfamily]{Ligatures=TeX,Scale=1}
\fi
\usepackage{lmodern}
\ifPDFTeX\else
  % xetex/luatex font selection
\fi
% Use upquote if available, for straight quotes in verbatim environments
\IfFileExists{upquote.sty}{\usepackage{upquote}}{}
\IfFileExists{microtype.sty}{% use microtype if available
  \usepackage[]{microtype}
  \UseMicrotypeSet[protrusion]{basicmath} % disable protrusion for tt fonts
}{}
\makeatletter
\@ifundefined{KOMAClassName}{% if non-KOMA class
  \IfFileExists{parskip.sty}{%
    \usepackage{parskip}
  }{% else
    \setlength{\parindent}{0pt}
    \setlength{\parskip}{6pt plus 2pt minus 1pt}}
}{% if KOMA class
  \KOMAoptions{parskip=half}}
\makeatother
\usepackage{xcolor}
\usepackage[left=0.5in,right=0.5in,top=0.5in,bottom=0.5in]{geometry}
\usepackage{color}
\usepackage{fancyvrb}
\newcommand{\VerbBar}{|}
\newcommand{\VERB}{\Verb[commandchars=\\\{\}]}
\DefineVerbatimEnvironment{Highlighting}{Verbatim}{commandchars=\\\{\}}
% Add ',fontsize=\small' for more characters per line
\usepackage{framed}
\definecolor{shadecolor}{RGB}{248,248,248}
\newenvironment{Shaded}{\begin{snugshade}}{\end{snugshade}}
\newcommand{\AlertTok}[1]{\textcolor[rgb]{0.94,0.16,0.16}{#1}}
\newcommand{\AnnotationTok}[1]{\textcolor[rgb]{0.56,0.35,0.01}{\textbf{\textit{#1}}}}
\newcommand{\AttributeTok}[1]{\textcolor[rgb]{0.13,0.29,0.53}{#1}}
\newcommand{\BaseNTok}[1]{\textcolor[rgb]{0.00,0.00,0.81}{#1}}
\newcommand{\BuiltInTok}[1]{#1}
\newcommand{\CharTok}[1]{\textcolor[rgb]{0.31,0.60,0.02}{#1}}
\newcommand{\CommentTok}[1]{\textcolor[rgb]{0.56,0.35,0.01}{\textit{#1}}}
\newcommand{\CommentVarTok}[1]{\textcolor[rgb]{0.56,0.35,0.01}{\textbf{\textit{#1}}}}
\newcommand{\ConstantTok}[1]{\textcolor[rgb]{0.56,0.35,0.01}{#1}}
\newcommand{\ControlFlowTok}[1]{\textcolor[rgb]{0.13,0.29,0.53}{\textbf{#1}}}
\newcommand{\DataTypeTok}[1]{\textcolor[rgb]{0.13,0.29,0.53}{#1}}
\newcommand{\DecValTok}[1]{\textcolor[rgb]{0.00,0.00,0.81}{#1}}
\newcommand{\DocumentationTok}[1]{\textcolor[rgb]{0.56,0.35,0.01}{\textbf{\textit{#1}}}}
\newcommand{\ErrorTok}[1]{\textcolor[rgb]{0.64,0.00,0.00}{\textbf{#1}}}
\newcommand{\ExtensionTok}[1]{#1}
\newcommand{\FloatTok}[1]{\textcolor[rgb]{0.00,0.00,0.81}{#1}}
\newcommand{\FunctionTok}[1]{\textcolor[rgb]{0.13,0.29,0.53}{\textbf{#1}}}
\newcommand{\ImportTok}[1]{#1}
\newcommand{\InformationTok}[1]{\textcolor[rgb]{0.56,0.35,0.01}{\textbf{\textit{#1}}}}
\newcommand{\KeywordTok}[1]{\textcolor[rgb]{0.13,0.29,0.53}{\textbf{#1}}}
\newcommand{\NormalTok}[1]{#1}
\newcommand{\OperatorTok}[1]{\textcolor[rgb]{0.81,0.36,0.00}{\textbf{#1}}}
\newcommand{\OtherTok}[1]{\textcolor[rgb]{0.56,0.35,0.01}{#1}}
\newcommand{\PreprocessorTok}[1]{\textcolor[rgb]{0.56,0.35,0.01}{\textit{#1}}}
\newcommand{\RegionMarkerTok}[1]{#1}
\newcommand{\SpecialCharTok}[1]{\textcolor[rgb]{0.81,0.36,0.00}{\textbf{#1}}}
\newcommand{\SpecialStringTok}[1]{\textcolor[rgb]{0.31,0.60,0.02}{#1}}
\newcommand{\StringTok}[1]{\textcolor[rgb]{0.31,0.60,0.02}{#1}}
\newcommand{\VariableTok}[1]{\textcolor[rgb]{0.00,0.00,0.00}{#1}}
\newcommand{\VerbatimStringTok}[1]{\textcolor[rgb]{0.31,0.60,0.02}{#1}}
\newcommand{\WarningTok}[1]{\textcolor[rgb]{0.56,0.35,0.01}{\textbf{\textit{#1}}}}
\usepackage{longtable,booktabs,array}
\usepackage{calc} % for calculating minipage widths
% Correct order of tables after \paragraph or \subparagraph
\usepackage{etoolbox}
\makeatletter
\patchcmd\longtable{\par}{\if@noskipsec\mbox{}\fi\par}{}{}
\makeatother
% Allow footnotes in longtable head/foot
\IfFileExists{footnotehyper.sty}{\usepackage{footnotehyper}}{\usepackage{footnote}}
\makesavenoteenv{longtable}
\usepackage{graphicx}
\makeatletter
\def\maxwidth{\ifdim\Gin@nat@width>\linewidth\linewidth\else\Gin@nat@width\fi}
\def\maxheight{\ifdim\Gin@nat@height>\textheight\textheight\else\Gin@nat@height\fi}
\makeatother
% Scale images if necessary, so that they will not overflow the page
% margins by default, and it is still possible to overwrite the defaults
% using explicit options in \includegraphics[width, height, ...]{}
\setkeys{Gin}{width=\maxwidth,height=\maxheight,keepaspectratio}
% Set default figure placement to htbp
\makeatletter
\def\fps@figure{htbp}
\makeatother
\setlength{\emergencystretch}{3em} % prevent overfull lines
\providecommand{\tightlist}{%
  \setlength{\itemsep}{0pt}\setlength{\parskip}{0pt}}
\setcounter{secnumdepth}{-\maxdimen} % remove section numbering
\usepackage{caption}
\usepackage{float}
\usepackage{placeins}
\captionsetup{font=small}
\ifLuaTeX
  \usepackage{selnolig}  % disable illegal ligatures
\fi
\usepackage{bookmark}
\IfFileExists{xurl.sty}{\usepackage{xurl}}{} % add URL line breaks if available
\urlstyle{same}
\hypersetup{
  pdftitle={Enhancement of SDR in Thessaloniki},
  pdfauthor={Natsis Athanasios; Alkiviadis Bais},
  colorlinks=true,
  linkcolor={Maroon},
  filecolor={Maroon},
  citecolor={Blue},
  urlcolor={Blue},
  pdfcreator={LaTeX via pandoc}}

\title{Enhancement of SDR in Thessaloniki}
\author{Natsis Athanasios\footnote{Laboratory of Atmospheric Physics, AUTH, \href{mailto:natsisphysicist@gmail.com}{\nolinkurl{natsisphysicist@gmail.com}}} \and Alkiviadis Bais\footnote{Laboratory of Atmospheric Physics, AUTH}}
\date{2024-06-12}

\begin{document}
\maketitle
\begin{abstract}
Study of GHI enchantment.
\end{abstract}

{
\hypersetup{linkcolor=}
\setcounter{tocdepth}{4}
\tableofcontents
}
\begin{verbatim}
Reference mode:       Low_B.Low_W 
\end{verbatim}

\begin{verbatim}
Enhancemnet criteria: Enhanc_C_4 
\end{verbatim}

\FloatBarrier

\subsubsection{Frequency Distributions}\label{frequency-distributions}

\begin{figure}[H]

{\centering \includegraphics[width=1\linewidth]{GHI_enh_05_distributions_files/figure-latex/freqdistributions-1} 

}

\caption{ - empty caption - }\label{fig:freqdistributions-1}
\end{figure}
\begin{figure}[H]

{\centering \includegraphics[width=1\linewidth]{GHI_enh_05_distributions_files/figure-latex/freqdistributions-2} 

}

\caption{ - empty caption - }\label{fig:freqdistributions-2}
\end{figure}
\begin{figure}[H]

{\centering \includegraphics[width=1\linewidth]{GHI_enh_05_distributions_files/figure-latex/freqdistributions-3} 

}

\caption{ - empty caption - }\label{fig:freqdistributions-3}
\end{figure}

\FloatBarrier

\subsubsection{Distributions}\label{distributions}

\begin{verbatim}
Warning in hist.default(DATA[GLB_diff > 0, GLB_ench], breaks = breaks, freq =
FALSE, : arguments 'freq', 'col', 'main', 'xlab' are not made use of
\end{verbatim}

\begin{figure}[H]

{\centering \includegraphics[width=1\linewidth]{GHI_enh_05_distributions_files/figure-latex/relative-distributions-1} 

}

\caption{ - empty caption - }\label{fig:relative-distributions-1}
\end{figure}

\begin{verbatim}
Warning in hist.default(DATA[GLB_diff > 0, GLB_diff], breaks = breaks, freq =
FALSE, : arguments 'freq', 'col', 'main', 'xlab' are not made use of
\end{verbatim}

\begin{figure}[H]

{\centering \includegraphics[width=1\linewidth]{GHI_enh_05_distributions_files/figure-latex/relative-distributions-2} 

}

\caption{ - empty caption - }\label{fig:relative-distributions-2}
\end{figure}

\begin{verbatim}
Warning in hist.default(DATA[GLB_diff > 0, GLB_rati], breaks = breaks, freq =
FALSE, : arguments 'freq', 'col', 'main', 'xlab' are not made use of
\end{verbatim}

\begin{figure}[H]

{\centering \includegraphics[width=1\linewidth]{GHI_enh_05_distributions_files/figure-latex/relative-distributions-3} 

}

\caption{ - empty caption - }\label{fig:relative-distributions-3}
\end{figure}

\begin{verbatim}
Warning: The dot-dot notation (`..count..`) was deprecated in ggplot2 3.4.0.
i Please use `after_stat(count)` instead.
This warning is displayed once every 8 hours.
Call `lifecycle::last_lifecycle_warnings()` to see where this warning was
generated.
\end{verbatim}

\begin{figure}[H]

{\centering \includegraphics[width=1\linewidth]{GHI_enh_05_distributions_files/figure-latex/relative-distributions-4} 

}

\caption{ - empty caption - }\label{fig:relative-distributions-4}
\end{figure}

\begin{Shaded}
\begin{Highlighting}[]
  \CommentTok{\# scale\_y\_continuous(labels = function(x) paste0(x, "\%"))}

\CommentTok{\# print(p, vp = grid::viewport(gp=grid::gpar(cex=1.5)))}
\CommentTok{\# theme\_gray(base\_size = 18)}



\DocumentationTok{\#\#  Extreme cases Distributions  {-}{-}{-}{-}{-}{-}{-}{-}{-}{-}{-}{-}{-}{-}{-}{-}{-}{-}{-}{-}{-}{-}{-}{-}{-}{-}{-}{-}{-}{-}{-}{-}{-}{-}{-}{-}{-}{-}{-}{-}{-}{-}{-}{-}{-}{-}{-}}
\end{Highlighting}
\end{Shaded}

\FloatBarrier

\subsubsection{Extreme cases Distributions}\label{extreme-cases-distributions}

\begin{figure}[H]

{\centering \includegraphics[width=1\linewidth]{GHI_enh_05_distributions_files/figure-latex/extremedistributions-1} 

}

\caption{ - empty caption - }\label{fig:extremedistributions-1}
\end{figure}
\begin{figure}[H]

{\centering \includegraphics[width=1\linewidth]{GHI_enh_05_distributions_files/figure-latex/extremedistributions-2} 

}

\caption{ - empty caption - }\label{fig:extremedistributions-2}
\end{figure}
\begin{figure}[H]

{\centering \includegraphics[width=1\linewidth]{GHI_enh_05_distributions_files/figure-latex/extremedistributions-3} 

}

\caption{ - empty caption - }\label{fig:extremedistributions-3}
\end{figure}
\begin{figure}[H]

{\centering \includegraphics[width=1\linewidth]{GHI_enh_05_distributions_files/figure-latex/extremedistributions-4} 

}

\caption{ - empty caption - }\label{fig:extremedistributions-4}
\end{figure}

Enhanc\_C\_4
6202167

\begin{figure}[H]

{\centering \includegraphics[width=1\linewidth]{GHI_enh_05_distributions_files/figure-latex/extremedistributions-5} 

}

\caption{ - empty caption - }\label{fig:extremedistributions-5}
\end{figure}
\begin{figure}[H]

{\centering \includegraphics[width=1\linewidth]{GHI_enh_05_distributions_files/figure-latex/extremedistributions-6} 

}

\caption{ - empty caption - }\label{fig:extremedistributions-6}
\end{figure}
\begin{figure}[H]

{\centering \includegraphics[width=1\linewidth]{GHI_enh_05_distributions_files/figure-latex/extremedistributions-7} 

}

\caption{ - empty caption - }\label{fig:extremedistributions-7}
\end{figure}
\begin{figure}[H]

{\centering \includegraphics[width=1\linewidth]{GHI_enh_05_distributions_files/figure-latex/P-relative-distribution-diff-1} 

}

\caption{ - empty caption - }\label{fig:P-relative-distribution-diff}
\end{figure}

\begin{longtable}[]{@{}
  >{\raggedleft\arraybackslash}p{(\columnwidth - 10\tabcolsep) * \real{0.1667}}
  >{\raggedleft\arraybackslash}p{(\columnwidth - 10\tabcolsep) * \real{0.1389}}
  >{\raggedleft\arraybackslash}p{(\columnwidth - 10\tabcolsep) * \real{0.1250}}
  >{\raggedleft\arraybackslash}p{(\columnwidth - 10\tabcolsep) * \real{0.1111}}
  >{\raggedleft\arraybackslash}p{(\columnwidth - 10\tabcolsep) * \real{0.1389}}
  >{\raggedleft\arraybackslash}p{(\columnwidth - 10\tabcolsep) * \real{0.1389}}@{}}
\toprule\noalign{}
\begin{minipage}[b]{\linewidth}\raggedleft
Min.
\end{minipage} & \begin{minipage}[b]{\linewidth}\raggedleft
1st Qu.
\end{minipage} & \begin{minipage}[b]{\linewidth}\raggedleft
Median
\end{minipage} & \begin{minipage}[b]{\linewidth}\raggedleft
Mean
\end{minipage} & \begin{minipage}[b]{\linewidth}\raggedleft
3rd Qu.
\end{minipage} & \begin{minipage}[b]{\linewidth}\raggedleft
Max.
\end{minipage} \\
\midrule\noalign{}
\endhead
\bottomrule\noalign{}
\endlastfoot
0.0007673 & 11.92 & 28.58 & 40.14 & 56.28 & 412.4 \\
\end{longtable}

\begin{longtable}[]{@{}
  >{\raggedleft\arraybackslash}p{(\columnwidth - 8\tabcolsep) * \real{0.1667}}
  >{\raggedleft\arraybackslash}p{(\columnwidth - 8\tabcolsep) * \real{0.1111}}
  >{\raggedleft\arraybackslash}p{(\columnwidth - 8\tabcolsep) * \real{0.1111}}
  >{\raggedleft\arraybackslash}p{(\columnwidth - 8\tabcolsep) * \real{0.1111}}
  >{\raggedleft\arraybackslash}p{(\columnwidth - 8\tabcolsep) * \real{0.1111}}@{}}
\toprule\noalign{}
\begin{minipage}[b]{\linewidth}\raggedleft
0\%
\end{minipage} & \begin{minipage}[b]{\linewidth}\raggedleft
25\%
\end{minipage} & \begin{minipage}[b]{\linewidth}\raggedleft
50\%
\end{minipage} & \begin{minipage}[b]{\linewidth}\raggedleft
75\%
\end{minipage} & \begin{minipage}[b]{\linewidth}\raggedleft
100\%
\end{minipage} \\
\midrule\noalign{}
\endhead
\bottomrule\noalign{}
\endlastfoot
0.0007673 & 11.92 & 28.58 & 56.28 & 412.4 \\
\end{longtable}

8.11010325640623 \% valuew are is above 100 W/m\^{}2

\begin{figure}[H]

{\centering \includegraphics[width=1\linewidth]{GHI_enh_05_distributions_files/figure-latex/P-extreme-distribution-1} 

}

\caption{ - empty caption - }\label{fig:P-extreme-distribution}
\end{figure}

0\% 25\% 50\% 75\% 100\%
84.27966 141.09830 170.53714 200.15315 412.44916

\textbf{END}

\begin{verbatim}
2024-06-12 05:56:52.0 athan@tyler GHI_enh_05_distributions.R 0.730842 mins
\end{verbatim}

\end{document}
