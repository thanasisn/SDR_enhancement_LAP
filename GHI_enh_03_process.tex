% Options for packages loaded elsewhere
\PassOptionsToPackage{unicode}{hyperref}
\PassOptionsToPackage{hyphens}{url}
\PassOptionsToPackage{dvipsnames,svgnames,x11names}{xcolor}
%
\documentclass[
  10pt,
  a4paper,oneside]{article}
\usepackage{amsmath,amssymb}
\usepackage{lmodern}
\usepackage{iftex}
\ifPDFTeX
  \usepackage[T1]{fontenc}
  \usepackage[utf8]{inputenc}
  \usepackage{textcomp} % provide euro and other symbols
\else % if luatex or xetex
  \usepackage{unicode-math}
  \defaultfontfeatures{Scale=MatchLowercase}
  \defaultfontfeatures[\rmfamily]{Ligatures=TeX,Scale=1}
\fi
% Use upquote if available, for straight quotes in verbatim environments
\IfFileExists{upquote.sty}{\usepackage{upquote}}{}
\IfFileExists{microtype.sty}{% use microtype if available
  \usepackage[]{microtype}
  \UseMicrotypeSet[protrusion]{basicmath} % disable protrusion for tt fonts
}{}
\makeatletter
\@ifundefined{KOMAClassName}{% if non-KOMA class
  \IfFileExists{parskip.sty}{%
    \usepackage{parskip}
  }{% else
    \setlength{\parindent}{0pt}
    \setlength{\parskip}{6pt plus 2pt minus 1pt}}
}{% if KOMA class
  \KOMAoptions{parskip=half}}
\makeatother
\usepackage{xcolor}
\IfFileExists{xurl.sty}{\usepackage{xurl}}{} % add URL line breaks if available
\IfFileExists{bookmark.sty}{\usepackage{bookmark}}{\usepackage{hyperref}}
\hypersetup{
  pdftitle={Enhancement of SDR in Thessaloniki},
  pdfauthor={Natsis Athanasios; Alkiviadis Bais},
  colorlinks=true,
  linkcolor={Maroon},
  filecolor={Maroon},
  citecolor={Blue},
  urlcolor={Blue},
  pdfcreator={LaTeX via pandoc}}
\urlstyle{same} % disable monospaced font for URLs
\usepackage[left=0.5in,right=0.5in,top=0.5in,bottom=0.5in]{geometry}
\usepackage{color}
\usepackage{fancyvrb}
\newcommand{\VerbBar}{|}
\newcommand{\VERB}{\Verb[commandchars=\\\{\}]}
\DefineVerbatimEnvironment{Highlighting}{Verbatim}{commandchars=\\\{\}}
% Add ',fontsize=\small' for more characters per line
\usepackage{framed}
\definecolor{shadecolor}{RGB}{248,248,248}
\newenvironment{Shaded}{\begin{snugshade}}{\end{snugshade}}
\newcommand{\AlertTok}[1]{\textcolor[rgb]{0.94,0.16,0.16}{#1}}
\newcommand{\AnnotationTok}[1]{\textcolor[rgb]{0.56,0.35,0.01}{\textbf{\textit{#1}}}}
\newcommand{\AttributeTok}[1]{\textcolor[rgb]{0.77,0.63,0.00}{#1}}
\newcommand{\BaseNTok}[1]{\textcolor[rgb]{0.00,0.00,0.81}{#1}}
\newcommand{\BuiltInTok}[1]{#1}
\newcommand{\CharTok}[1]{\textcolor[rgb]{0.31,0.60,0.02}{#1}}
\newcommand{\CommentTok}[1]{\textcolor[rgb]{0.56,0.35,0.01}{\textit{#1}}}
\newcommand{\CommentVarTok}[1]{\textcolor[rgb]{0.56,0.35,0.01}{\textbf{\textit{#1}}}}
\newcommand{\ConstantTok}[1]{\textcolor[rgb]{0.00,0.00,0.00}{#1}}
\newcommand{\ControlFlowTok}[1]{\textcolor[rgb]{0.13,0.29,0.53}{\textbf{#1}}}
\newcommand{\DataTypeTok}[1]{\textcolor[rgb]{0.13,0.29,0.53}{#1}}
\newcommand{\DecValTok}[1]{\textcolor[rgb]{0.00,0.00,0.81}{#1}}
\newcommand{\DocumentationTok}[1]{\textcolor[rgb]{0.56,0.35,0.01}{\textbf{\textit{#1}}}}
\newcommand{\ErrorTok}[1]{\textcolor[rgb]{0.64,0.00,0.00}{\textbf{#1}}}
\newcommand{\ExtensionTok}[1]{#1}
\newcommand{\FloatTok}[1]{\textcolor[rgb]{0.00,0.00,0.81}{#1}}
\newcommand{\FunctionTok}[1]{\textcolor[rgb]{0.00,0.00,0.00}{#1}}
\newcommand{\ImportTok}[1]{#1}
\newcommand{\InformationTok}[1]{\textcolor[rgb]{0.56,0.35,0.01}{\textbf{\textit{#1}}}}
\newcommand{\KeywordTok}[1]{\textcolor[rgb]{0.13,0.29,0.53}{\textbf{#1}}}
\newcommand{\NormalTok}[1]{#1}
\newcommand{\OperatorTok}[1]{\textcolor[rgb]{0.81,0.36,0.00}{\textbf{#1}}}
\newcommand{\OtherTok}[1]{\textcolor[rgb]{0.56,0.35,0.01}{#1}}
\newcommand{\PreprocessorTok}[1]{\textcolor[rgb]{0.56,0.35,0.01}{\textit{#1}}}
\newcommand{\RegionMarkerTok}[1]{#1}
\newcommand{\SpecialCharTok}[1]{\textcolor[rgb]{0.00,0.00,0.00}{#1}}
\newcommand{\SpecialStringTok}[1]{\textcolor[rgb]{0.31,0.60,0.02}{#1}}
\newcommand{\StringTok}[1]{\textcolor[rgb]{0.31,0.60,0.02}{#1}}
\newcommand{\VariableTok}[1]{\textcolor[rgb]{0.00,0.00,0.00}{#1}}
\newcommand{\VerbatimStringTok}[1]{\textcolor[rgb]{0.31,0.60,0.02}{#1}}
\newcommand{\WarningTok}[1]{\textcolor[rgb]{0.56,0.35,0.01}{\textbf{\textit{#1}}}}
\usepackage{longtable,booktabs,array}
\usepackage{calc} % for calculating minipage widths
% Correct order of tables after \paragraph or \subparagraph
\usepackage{etoolbox}
\makeatletter
\patchcmd\longtable{\par}{\if@noskipsec\mbox{}\fi\par}{}{}
\makeatother
% Allow footnotes in longtable head/foot
\IfFileExists{footnotehyper.sty}{\usepackage{footnotehyper}}{\usepackage{footnote}}
\makesavenoteenv{longtable}
\usepackage{graphicx}
\makeatletter
\def\maxwidth{\ifdim\Gin@nat@width>\linewidth\linewidth\else\Gin@nat@width\fi}
\def\maxheight{\ifdim\Gin@nat@height>\textheight\textheight\else\Gin@nat@height\fi}
\makeatother
% Scale images if necessary, so that they will not overflow the page
% margins by default, and it is still possible to overwrite the defaults
% using explicit options in \includegraphics[width, height, ...]{}
\setkeys{Gin}{width=\maxwidth,height=\maxheight,keepaspectratio}
% Set default figure placement to htbp
\makeatletter
\def\fps@figure{htbp}
\makeatother
\setlength{\emergencystretch}{3em} % prevent overfull lines
\providecommand{\tightlist}{%
  \setlength{\itemsep}{0pt}\setlength{\parskip}{0pt}}
\setcounter{secnumdepth}{-\maxdimen} % remove section numbering
\usepackage{caption}
\usepackage{float}
\usepackage{placeins}
\captionsetup{font=small}
\ifLuaTeX
  \usepackage{selnolig}  % disable illegal ligatures
\fi

\title{Enhancement of SDR in Thessaloniki}
\author{Natsis Athanasios\footnote{Laboratory of Atmospheric Physics, AUTH, \href{mailto:natsisphysicist@gmail.com}{\nolinkurl{natsisphysicist@gmail.com}}} \and Alkiviadis Bais\footnote{Laboratory of Atmospheric Physics, AUTH}}
\date{2024-06-06}

\begin{document}
\maketitle
\begin{abstract}
Study of GHI enchantment.
\end{abstract}

{
\hypersetup{linkcolor=}
\setcounter{tocdepth}{4}
\tableofcontents
}
\begin{verbatim}
   Min. 1st Qu.  Median    Mean 3rd Qu.    Max. 
    0.0    89.7   180.0   180.0   270.3   360.0 
\end{verbatim}

\begin{verbatim}
    Min.  1st Qu.   Median     Mean  3rd Qu.     Max. 
-72.8133 -27.5835   0.6886   0.3233  28.1502  83.4876 
\end{verbatim}

Alpha * HAU is CS\_ref

\hypertarget{time-series-decomposition-tests}{%
\subsection{Time series decomposition tests}\label{time-series-decomposition-tests}}

interactive FALSE

\begin{Shaded}
\begin{Highlighting}[]
\FunctionTok{plot\_time\_series}\NormalTok{(ST\_daily, Date, wattGLB.mean, }\AttributeTok{.interactive =}\NormalTok{ isinter)}
\end{Highlighting}
\end{Shaded}

\begin{verbatim}
Warning: Removed 2 rows containing missing values or values outside the scale range (`geom_line()`).
Removed 2 rows containing missing values or values outside the scale range (`geom_line()`).
\end{verbatim}

\begin{figure}[H]

{\centering \includegraphics[width=1\linewidth]{GHI_enh_03_process_files/figure-latex/unnamed-chunk-6-1} 

}

\caption{ - empty caption - }\label{fig:unnamed-chunk-6-1}
\end{figure}

\begin{Shaded}
\begin{Highlighting}[]
\FunctionTok{plot\_time\_series}\NormalTok{(ST\_daily, Date, wattGLB.sum , }\AttributeTok{.interactive =}\NormalTok{ isinter)}
\end{Highlighting}
\end{Shaded}

\begin{verbatim}
Warning: Removed 2 rows containing missing values or values outside the scale range (`geom_line()`).
Removed 2 rows containing missing values or values outside the scale range (`geom_line()`).
\end{verbatim}

\begin{figure}[H]

{\centering \includegraphics[width=1\linewidth]{GHI_enh_03_process_files/figure-latex/unnamed-chunk-6-2} 

}

\caption{ - empty caption - }\label{fig:unnamed-chunk-6-2}
\end{figure}

\begin{Shaded}
\begin{Highlighting}[]
\FunctionTok{plot\_time\_series}\NormalTok{(ST\_daily, Date, wattGLB.max , }\AttributeTok{.interactive =}\NormalTok{ isinter)}
\end{Highlighting}
\end{Shaded}

\begin{verbatim}
Warning: Removed 2 rows containing missing values or values outside the scale range (`geom_line()`).
Removed 2 rows containing missing values or values outside the scale range (`geom_line()`).
\end{verbatim}

\begin{figure}[H]

{\centering \includegraphics[width=1\linewidth]{GHI_enh_03_process_files/figure-latex/unnamed-chunk-6-3} 

}

\caption{ - empty caption - }\label{fig:unnamed-chunk-6-3}
\end{figure}

\begin{Shaded}
\begin{Highlighting}[]
\FunctionTok{plot\_time\_series}\NormalTok{(ST\_daily, Date, wattGLB.N   , }\AttributeTok{.interactive =}\NormalTok{ isinter)}
\end{Highlighting}
\end{Shaded}

\begin{figure}[H]

{\centering \includegraphics[width=1\linewidth]{GHI_enh_03_process_files/figure-latex/unnamed-chunk-6-4} 

}

\caption{ - empty caption - }\label{fig:unnamed-chunk-6-4}
\end{figure}

\begin{Shaded}
\begin{Highlighting}[]
\FunctionTok{plot\_time\_series}\NormalTok{(ST\_E\_daily, Date, wattGLB.mean, }\AttributeTok{.interactive =}\NormalTok{ isinter)}
\end{Highlighting}
\end{Shaded}

\begin{figure}[H]

{\centering \includegraphics[width=1\linewidth]{GHI_enh_03_process_files/figure-latex/unnamed-chunk-6-5} 

}

\caption{ - empty caption - }\label{fig:unnamed-chunk-6-5}
\end{figure}

\begin{Shaded}
\begin{Highlighting}[]
\FunctionTok{plot\_time\_series}\NormalTok{(ST\_E\_daily, Date, wattGLB.sum , }\AttributeTok{.interactive =}\NormalTok{ isinter)}
\end{Highlighting}
\end{Shaded}

\begin{figure}[H]

{\centering \includegraphics[width=1\linewidth]{GHI_enh_03_process_files/figure-latex/unnamed-chunk-6-6} 

}

\caption{ - empty caption - }\label{fig:unnamed-chunk-6-6}
\end{figure}

\begin{Shaded}
\begin{Highlighting}[]
\FunctionTok{plot\_time\_series}\NormalTok{(ST\_E\_daily, Date, wattGLB.max , }\AttributeTok{.interactive =}\NormalTok{ isinter)}
\end{Highlighting}
\end{Shaded}

\begin{figure}[H]

{\centering \includegraphics[width=1\linewidth]{GHI_enh_03_process_files/figure-latex/unnamed-chunk-6-7} 

}

\caption{ - empty caption - }\label{fig:unnamed-chunk-6-7}
\end{figure}

\begin{Shaded}
\begin{Highlighting}[]
\FunctionTok{plot\_time\_series}\NormalTok{(ST\_E\_daily, Date, wattGLB.N   , }\AttributeTok{.interactive =}\NormalTok{ isinter)}
\end{Highlighting}
\end{Shaded}

\begin{figure}[H]

{\centering \includegraphics[width=1\linewidth]{GHI_enh_03_process_files/figure-latex/unnamed-chunk-6-8} 

}

\caption{ - empty caption - }\label{fig:unnamed-chunk-6-8}
\end{figure}

\begin{Shaded}
\begin{Highlighting}[]
\FunctionTok{plot\_time\_series}\NormalTok{(ST\_E\_monthly, Date, wattGLB.mean, }\AttributeTok{.interactive =}\NormalTok{ isinter)}
\end{Highlighting}
\end{Shaded}

\begin{figure}[H]

{\centering \includegraphics[width=1\linewidth]{GHI_enh_03_process_files/figure-latex/unnamed-chunk-6-9} 

}

\caption{ - empty caption - }\label{fig:unnamed-chunk-6-9}
\end{figure}

\begin{Shaded}
\begin{Highlighting}[]
\FunctionTok{plot\_time\_series}\NormalTok{(ST\_E\_monthly, Date, wattGLB.sum , }\AttributeTok{.interactive =}\NormalTok{ isinter)}
\end{Highlighting}
\end{Shaded}

\begin{figure}[H]

{\centering \includegraphics[width=1\linewidth]{GHI_enh_03_process_files/figure-latex/unnamed-chunk-6-10} 

}

\caption{ - empty caption - }\label{fig:unnamed-chunk-6-10}
\end{figure}

\begin{Shaded}
\begin{Highlighting}[]
\FunctionTok{plot\_time\_series}\NormalTok{(ST\_E\_monthly, Date, wattGLB.max , }\AttributeTok{.interactive =}\NormalTok{ isinter)}
\end{Highlighting}
\end{Shaded}

\begin{figure}[H]

{\centering \includegraphics[width=1\linewidth]{GHI_enh_03_process_files/figure-latex/unnamed-chunk-6-11} 

}

\caption{ - empty caption - }\label{fig:unnamed-chunk-6-11}
\end{figure}

\begin{Shaded}
\begin{Highlighting}[]
\FunctionTok{plot\_time\_series}\NormalTok{(ST\_E\_monthly, Date, wattGLB.N   , }\AttributeTok{.interactive =}\NormalTok{ isinter)}
\end{Highlighting}
\end{Shaded}

\begin{figure}[H]

{\centering \includegraphics[width=1\linewidth]{GHI_enh_03_process_files/figure-latex/unnamed-chunk-6-12} 

}

\caption{ - empty caption - }\label{fig:unnamed-chunk-6-12}
\end{figure}

\begin{Shaded}
\begin{Highlighting}[]
\FunctionTok{plot\_time\_series}\NormalTok{(ST\_E\_monthly, Date, wattGLB.N   , }\AttributeTok{.interactive =}\NormalTok{ isinter, }\AttributeTok{.smooth =}\NormalTok{ T, }\AttributeTok{.smooth\_span =} \FloatTok{0.3}\NormalTok{ )}
\end{Highlighting}
\end{Shaded}

\begin{figure}[H]

{\centering \includegraphics[width=1\linewidth]{GHI_enh_03_process_files/figure-latex/unnamed-chunk-6-13} 

}

\caption{ - empty caption - }\label{fig:unnamed-chunk-6-13}
\end{figure}

\begin{Shaded}
\begin{Highlighting}[]
\FunctionTok{plot\_acf\_diagnostics}\NormalTok{(ST\_E\_daily, Date, wattGLB.N, }\AttributeTok{.lags =} \DecValTok{1}\SpecialCharTok{:}\DecValTok{60}\NormalTok{, }\AttributeTok{.interactive =}\NormalTok{ isinter)}
\end{Highlighting}
\end{Shaded}

\begin{figure}[H]

{\centering \includegraphics[width=1\linewidth]{GHI_enh_03_process_files/figure-latex/unnamed-chunk-6-14} 

}

\caption{ - empty caption - }\label{fig:unnamed-chunk-6-14}
\end{figure}

\begin{Shaded}
\begin{Highlighting}[]
\FunctionTok{plot\_stl\_diagnostics}\NormalTok{(ST\_E\_daily, Date, wattGLB.N,}
                     \AttributeTok{.feature\_set =} \FunctionTok{c}\NormalTok{(}\StringTok{"observed"}\NormalTok{,}\StringTok{"season"}\NormalTok{, }\StringTok{"trend"}\NormalTok{, }\StringTok{"remainder"}\NormalTok{),}
                     \AttributeTok{.trend =} \DecValTok{180}\NormalTok{,}
                     \AttributeTok{.frequency =} \DecValTok{30}\NormalTok{,}
                     \AttributeTok{.interactive =}\NormalTok{ isinter}
\NormalTok{                     )}
\end{Highlighting}
\end{Shaded}

\begin{verbatim}
frequency = 30 observations
\end{verbatim}

\begin{verbatim}
trend = 180 observations
\end{verbatim}

\begin{figure}[H]

{\centering \includegraphics[width=1\linewidth]{GHI_enh_03_process_files/figure-latex/unnamed-chunk-6-15} 

}

\caption{ - empty caption - }\label{fig:unnamed-chunk-6-15}
\end{figure}

\begin{Shaded}
\begin{Highlighting}[]
\FunctionTok{plot\_stl\_diagnostics}\NormalTok{(ST\_E\_monthly, Date, wattGLB.N,}
                     \AttributeTok{.feature\_set =} \FunctionTok{c}\NormalTok{(}\StringTok{"observed"}\NormalTok{,}\StringTok{"season"}\NormalTok{, }\StringTok{"trend"}\NormalTok{, }\StringTok{"remainder"}\NormalTok{),}
                     \AttributeTok{.trend =} \DecValTok{180}\NormalTok{,}
                     \AttributeTok{.frequency =} \DecValTok{30}\NormalTok{,}
                     \AttributeTok{.interactive =}\NormalTok{ isinter}
\NormalTok{                     )}
\end{Highlighting}
\end{Shaded}

\begin{verbatim}
frequency = 30 observations
trend = 180 observations
\end{verbatim}

\begin{figure}[H]

{\centering \includegraphics[width=1\linewidth]{GHI_enh_03_process_files/figure-latex/unnamed-chunk-6-16} 

}

\caption{ - empty caption - }\label{fig:unnamed-chunk-6-16}
\end{figure}

\begin{Shaded}
\begin{Highlighting}[]
\FunctionTok{plot\_seasonal\_diagnostics}\NormalTok{(ST\_E\_daily, Date, wattGLB.N, }\AttributeTok{.interactive =}\NormalTok{ isinter,}
                          \AttributeTok{.feature\_set =} \FunctionTok{c}\NormalTok{(}\StringTok{"week"}\NormalTok{, }\StringTok{"month.lbl"}\NormalTok{, }\StringTok{"quarter"}\NormalTok{, }\StringTok{"year"}\NormalTok{))}
\end{Highlighting}
\end{Shaded}

\begin{figure}[H]

{\centering \includegraphics[width=1\linewidth]{GHI_enh_03_process_files/figure-latex/unnamed-chunk-6-17} 

}

\caption{ - empty caption - }\label{fig:unnamed-chunk-6-17}
\end{figure}

\begin{Shaded}
\begin{Highlighting}[]
\FunctionTok{plot\_seasonal\_diagnostics}\NormalTok{(ST\_E\_monthly, Date, wattGLB.N, }\AttributeTok{.interactive =}\NormalTok{ isinter)}
\end{Highlighting}
\end{Shaded}

\begin{figure}[H]

{\centering \includegraphics[width=1\linewidth]{GHI_enh_03_process_files/figure-latex/unnamed-chunk-6-18} 

}

\caption{ - empty caption - }\label{fig:unnamed-chunk-6-18}
\end{figure}

\begin{Shaded}
\begin{Highlighting}[]
\CommentTok{\# https://business{-}science.github.io/timetk/articles/TK08\_Automatic\_Anomaly\_Detection.html}

\FunctionTok{plot\_time\_series\_regression}\NormalTok{(}
    \AttributeTok{.data         =}\NormalTok{ ST\_E\_monthly,}
    \AttributeTok{.date\_var     =}\NormalTok{ Date,}
    \AttributeTok{.formula      =}\NormalTok{ wattGLB.N }\SpecialCharTok{\textasciitilde{}} \FunctionTok{as.numeric}\NormalTok{(Date)  ,}
    \AttributeTok{.facet\_ncol   =} \DecValTok{2}\NormalTok{,}
    \AttributeTok{.interactive  =}\NormalTok{ isinter,}
    \AttributeTok{.show\_summary =} \ConstantTok{TRUE}
\NormalTok{)}
\end{Highlighting}
\end{Shaded}

\begin{verbatim}
Call:
stats::lm(formula = .formula, data = df)

Residuals:
    Min      1Q  Median      3Q     Max 
-319.48 -147.87  -56.18  101.04  719.45 

Coefficients:
                  Estimate Std. Error t value Pr(>|t|)    
(Intercept)      1.379e+02  4.559e+01   3.025 0.002663 ** 
as.numeric(Date) 1.306e-07  3.650e-08   3.579 0.000391 ***
---
Signif. codes:  0 '***' 0.001 '**' 0.01 '*' 0.05 '.' 0.1 ' ' 1

Residual standard error: 196.3 on 363 degrees of freedom
Multiple R-squared:  0.03409,   Adjusted R-squared:  0.03143 
F-statistic: 12.81 on 1 and 363 DF,  p-value: 0.0003913
\end{verbatim}

\begin{figure}[H]

{\centering \includegraphics[width=1\linewidth]{GHI_enh_03_process_files/figure-latex/unnamed-chunk-6-19} 

}

\caption{ - empty caption - }\label{fig:unnamed-chunk-6-19}
\end{figure}

\begin{Shaded}
\begin{Highlighting}[]
\NormalTok{lm1 }\OtherTok{\textless{}{-}} \FunctionTok{lm}\NormalTok{(ST\_E\_monthly}\SpecialCharTok{$}\NormalTok{wattGLB.N }\SpecialCharTok{\textasciitilde{}}\NormalTok{ ST\_E\_monthly}\SpecialCharTok{$}\NormalTok{Date)}
\FunctionTok{summary}\NormalTok{(lm1)}
\end{Highlighting}
\end{Shaded}

\begin{verbatim}
Call:
lm(formula = ST_E_monthly$wattGLB.N ~ ST_E_monthly$Date)

Residuals:
    Min      1Q  Median      3Q     Max 
-319.48 -147.87  -56.18  101.04  719.45 

Coefficients:
                   Estimate Std. Error t value Pr(>|t|)    
(Intercept)       1.379e+02  4.559e+01   3.025 0.002663 ** 
ST_E_monthly$Date 1.306e-07  3.650e-08   3.579 0.000391 ***
---
Signif. codes:  0 '***' 0.001 '**' 0.01 '*' 0.05 '.' 0.1 ' ' 1

Residual standard error: 196.3 on 363 degrees of freedom
Multiple R-squared:  0.03409,   Adjusted R-squared:  0.03143 
F-statistic: 12.81 on 1 and 363 DF,  p-value: 0.0003913
\end{verbatim}

\begin{Shaded}
\begin{Highlighting}[]
\FunctionTok{plot\_time\_series\_regression}\NormalTok{(}
    \AttributeTok{.data         =}\NormalTok{ ST\_E\_daily,}
    \AttributeTok{.date\_var     =}\NormalTok{ Date,}
    \AttributeTok{.formula      =}\NormalTok{ wattGLB.N }\SpecialCharTok{\textasciitilde{}} \FunctionTok{as.numeric}\NormalTok{(Date)  ,}
    \AttributeTok{.facet\_ncol   =} \DecValTok{2}\NormalTok{,}
    \AttributeTok{.interactive  =}\NormalTok{ isinter,}
    \AttributeTok{.show\_summary =} \ConstantTok{FALSE}
\NormalTok{)}
\end{Highlighting}
\end{Shaded}

\begin{figure}[H]

{\centering \includegraphics[width=1\linewidth]{GHI_enh_03_process_files/figure-latex/unnamed-chunk-6-20} 

}

\caption{ - empty caption - }\label{fig:unnamed-chunk-6-20}
\end{figure}

\begin{Shaded}
\begin{Highlighting}[]
\CommentTok{\#  Save environment  {-}{-}{-}{-}{-}{-}{-}{-}{-}{-}{-}{-}{-}{-}{-}{-}{-}{-}{-}{-}{-}{-}{-}{-}{-}{-}{-}{-}{-}{-}{-}{-}{-}{-}{-}{-}{-}{-}{-}{-}{-}{-}{-}{-}{-}{-}{-}{-}{-}{-}{-}{-}{-}{-}{-}{-}{-}{-}{-}}

\DocumentationTok{\#\# Variables}
\NormalTok{objects }\OtherTok{\textless{}{-}} \FunctionTok{grep}\NormalTok{(}\StringTok{"\^{}tic$|\^{}tac$|\^{}Script.Name$|\^{}tag$"}\NormalTok{, }\FunctionTok{ls}\NormalTok{(), }\AttributeTok{value =}\NormalTok{ T, }\AttributeTok{invert =}\NormalTok{ T)}
\NormalTok{objects }\OtherTok{\textless{}{-}}\NormalTok{ objects[}\FunctionTok{sapply}\NormalTok{(objects, }\ControlFlowTok{function}\NormalTok{(x)}
    \FunctionTok{is.numeric}\NormalTok{(}\FunctionTok{get}\NormalTok{(x)) }\SpecialCharTok{|}
        \FunctionTok{is.character}\NormalTok{(}\FunctionTok{get}\NormalTok{(x)) }\SpecialCharTok{\&}
        \FunctionTok{object.size}\NormalTok{(}\FunctionTok{get}\NormalTok{(x)) }\SpecialCharTok{\textless{}} \DecValTok{1009} \SpecialCharTok{\&}
\NormalTok{        (}\SpecialCharTok{!}\FunctionTok{is.vector}\NormalTok{(}\FunctionTok{get}\NormalTok{(x)) }\SpecialCharTok{|}
             \SpecialCharTok{!}\FunctionTok{is.function}\NormalTok{(}\FunctionTok{get}\NormalTok{(x))), }\AttributeTok{simplify =}\NormalTok{ T)]}
\DocumentationTok{\#\# Data}
\NormalTok{objects }\OtherTok{\textless{}{-}} \FunctionTok{c}\NormalTok{(}
\NormalTok{    objects,}
    \FunctionTok{grep}\NormalTok{(}\StringTok{"\^{}ST\_"}\NormalTok{, }\FunctionTok{ls}\NormalTok{(), }\AttributeTok{value =} \ConstantTok{TRUE}\NormalTok{)}
\NormalTok{)}
\DocumentationTok{\#\# list saved}
\FunctionTok{cat}\NormalTok{(objects, }\AttributeTok{sep =} \StringTok{"}\SpecialCharTok{\textbackslash{}n}\StringTok{"}\NormalTok{)}
\end{Highlighting}
\end{Shaded}

\begin{verbatim}
aa
aatm
aday
adjpoint55
AEin1
All_daily_ratio_lim
alpha
amean
ampl
asza
atype
avar
b
BIO_ELEVA
C4_cs_ref_ratio
C4_GLB_diff_THRES
C4_lowcut_ratio
C4_lowcut_sza
C4_test_cs_ref_ratio
C4_test_GLB_diff_THRES
C4_test_lowcut_ratio
C4_test_lowcut_sza
Clear_daily_ratio_lim
CLEARdir
Cloud_daily_ratio_lim
common_data_13
common_data_14
common_data_14_2
CS_file_13
CS_file_14
CS_file_14_2
csmodel
Daily_aggregation_N_lim
Daily_confidence_limit
daylist
Days_of_year
Energy_Div
fit
I1_longterm
I2_szatrend
I3_trendsconsist
ii
Input_data_ID
Kurudz_SC
lm_mean
lm_zeropoint
mean_55
mean_BR_55
mean_BR_mean
mean_BR_median
mean_BR_min
mean_mean
mean_median
mean_min
mid
MIN_ELEVA
MIN_N
mm
Monthly_aggegation_N_lim
Monthly_confidence_limit
my.cols
my.cols.gr
pyear
R_COMPILE_PKGS
RANDOM_SEED
raw_input_data
running_mean_window_days
running_mean_window_years
SEAS_MIN_N
SelEnhanc
slope_55
slope_mean
slope_median
slope_min
SZA_aggregation_N_lim
SZA_BIN
SZA_confidence_limit
t
tsy
units
variables_fl
xlim
yearstodo
ylim
zeropoint
zeropointA
ST_daily
ST_E_daily
ST_E_daily_seas
ST_E_monthly
ST_E_monthly_seas
ST_E_sza
ST_E_sza_doy
ST_E_sza_monthly
ST_E_yearly
ST_extreme_daily
ST_extreme_monthly
ST_extreme_SZA
ST_extreme_SZA_doy
ST_extreme_SZA_monthly
ST_extreme_total
ST_extreme_yearly
ST_G0
ST_monthly
ST_sza
ST_sza_doy
ST_sza_monthly
ST_total
ST_yearly
\end{verbatim}

\begin{Shaded}
\begin{Highlighting}[]
\DocumentationTok{\#\# Save}
\FunctionTok{save}\NormalTok{(}\AttributeTok{file =} \FunctionTok{paste0}\NormalTok{(}\StringTok{"./data/"}\NormalTok{, }\FunctionTok{basename}\NormalTok{(}\FunctionTok{sub}\NormalTok{(}\StringTok{"}\SpecialCharTok{\textbackslash{}\textbackslash{}}\StringTok{.R"}\NormalTok{, }\StringTok{".Rda"}\NormalTok{, Script.Name))),}
     \AttributeTok{list =}\NormalTok{ objects,}
     \AttributeTok{compress =} \StringTok{"xz"}\NormalTok{)}
\end{Highlighting}
\end{Shaded}

\textbf{END}

\begin{verbatim}
2024-06-06 10:26:55.1 athan@tyler GHI_enh_03_process.R 5.853366 mins
\end{verbatim}

\end{document}
