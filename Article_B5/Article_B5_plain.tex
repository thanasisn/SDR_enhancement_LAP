\documentclass[preprint, 3p,
authoryear]{elsarticle} %review=doublespace preprint=single 5p=2 column
%%% Begin My package additions %%%%%%%%%%%%%%%%%%%

\usepackage[hyphens]{url}

  \journal{An awesome journal} % Sets Journal name

\usepackage{graphicx}
%%%%%%%%%%%%%%%% end my additions to header

\usepackage[T1]{fontenc}
\usepackage{lmodern}
\usepackage{amssymb,amsmath}
% TODO: Currently lineno needs to be loaded after amsmath because of conflict
% https://github.com/latex-lineno/lineno/issues/5
\usepackage{lineno} % add
\usepackage{ifxetex,ifluatex}
\usepackage{fixltx2e} % provides \textsubscript
% use upquote if available, for straight quotes in verbatim environments
\IfFileExists{upquote.sty}{\usepackage{upquote}}{}
\ifnum 0\ifxetex 1\fi\ifluatex 1\fi=0 % if pdftex
  \usepackage[utf8]{inputenc}
\else % if luatex or xelatex
  \usepackage{fontspec}
  \ifxetex
    \usepackage{xltxtra,xunicode}
  \fi
  \defaultfontfeatures{Mapping=tex-text,Scale=MatchLowercase}
  \newcommand{\euro}{€}
\fi
% use microtype if available
\IfFileExists{microtype.sty}{\usepackage{microtype}}{}
\usepackage[]{natbib}
\bibliographystyle{elsarticle-harv}

\ifxetex
  \usepackage[setpagesize=false, % page size defined by xetex
              unicode=false, % unicode breaks when used with xetex
              xetex]{hyperref}
\else
  \usepackage[unicode=true]{hyperref}
\fi
\hypersetup{breaklinks=true,
            bookmarks=true,
            pdfauthor={},
            pdftitle={Long term cloud enhancement events of global solar irradiance over Thessaloniki},
            colorlinks=false,
            urlcolor=blue,
            linkcolor=magenta,
            pdfborder={0 0 0}}

\setcounter{secnumdepth}{5}
% Pandoc toggle for numbering sections (defaults to be off)


% tightlist command for lists without linebreak
\providecommand{\tightlist}{%
  \setlength{\itemsep}{0pt}\setlength{\parskip}{0pt}}




\usepackage{caption}
\usepackage{placeins}
\captionsetup{font=small}
\usepackage{booktabs}
\usepackage{longtable}
\usepackage{array}
\usepackage{multirow}
\usepackage{wrapfig}
\usepackage{float}
\usepackage{colortbl}
\usepackage{pdflscape}
\usepackage{tabu}
\usepackage{threeparttable}
\usepackage{threeparttablex}
\usepackage[normalem]{ulem}
\usepackage{makecell}
\usepackage{xcolor}



\begin{document}


\begin{frontmatter}

  \title{Long term cloud enhancement events of global solar irradiance
over Thessaloniki}
    \author[LAP]{Athanasios N. Natsis%
  \corref{cor1}%
  \fnref{1}}
   \ead{natsisphysicist@gmail.com} 
    \author[LAP]{Alkiviadis Bais%
  %
  }
   \ead{abais@auth.gr} 
    \author[LAP]{Charikleia Meleti%
  %
  \fnref{1}}
   \ead{meleti@auth.gr} 
      \affiliation[LAP]{
    organization={Laboratory of Atmospheric Physics, Physics Department,
Aristotle University of
Thessaloniki},city={Thessaloniki},postcode={54124},country={Greece},}
    \cortext[cor1]{Corresponding author}
    \fntext[1]{This is the first author footnote.}
    \fntext[2]{Another author footnote.}
  
  \begin{abstract}
  Here will be the abstract.
  \end{abstract}
    \begin{keyword}
    cloud enhancement \sep total solar radiation \sep global horizontal
irradiance \sep 
    over irradiance
  \end{keyword}
  
 \end{frontmatter}

\hypertarget{abstract}{%
\section*{Abstract}\label{abstract}}
\addcontentsline{toc}{section}{Abstract}

We studied the enhancement of the global horizontal irradiance (GHI),
over Thessaloniki, for the period 1993 -- 2023, on one minute average
records. We identify the cloud enhancement events (CE), by creating an
appropriate clear sky irradiance reference with the use of a radiation
transfer model, and aerosol optical depth (AOD) data, from the AERONET,
and a collocated Brewer photospectrometer. We found that there is a
trend of CE cases of \(45.6\pm 21.9\,\text{cases}/\text{year}\), and a
mean total energy change, of the CE events by
\(116.9\pm 67.8\,\text{kj}/\text{year}\) with the peak of the CE events
are observed during May and June. An analysis of the total duration of
CE events, showed that durations longer than 5 minutes are very rare,
with exceptions lasting over an hour. We have detected enhancements over
the total solar irradiance, without exceeding the maximum values
reported for other sites, with more favorable CE conditions.

~ ~ CE: cloud enhancement cases one minute

ECE: extreme cloud enhancement cases one minute over TSI on horizontal
plane

CEG: cloud enhancement groups, cases events with consecutive CE

\(\text{GHI}_\text{i}\): measured one minute global horizontal
irradiance

\(\text{GHI}_\text{CSm}\): modelled clear sky one minute global
horizontal irradiance

\hypertarget{intro}{%
\section{Intro}\label{intro}}

The shortwave solar irradiance, reaching Earth's surface, is the main
energy source of the atmosphere and the biosphere, and drives and
governs the climate \citep{Gray2010}. It has direct practical
application, in industries like energy and agricultural production. The
variability of its intensity, can cause difficulties in predicting the
yield, and designing the specifications of the appropriate equipment.
Significant portion of research has been focused on predicting the
renewable energy production in a fine timescale and in near real-time
(for a review see \citet{Inman2013}; \citet{Graabak2016}).

A big aspect of this variability is the interaction of GHI with the
clouds. In general, clouds absorb part of the solar irradiance, but
under certain conditions, can enhance the total shortwave irradiance
reaching the ground. This effect, can locally increase the observed GHI,
even higher than the expected clear sky irradiance {[}see references
therein{]}.

Some of the proposed underling mechanisms of those enhancements, have
been summarized by \citet{Gueymard2017}, and include scattering on the
edge of cumulus clouds or through thin cirrus. Further investigation
with radiation transfer modeling methods and observations have pointed
as the prevailing mechanism, the forward Mie scattering
\citep{Pecenak2016, Thuillier2013, Yordanov2013, Yordanov2015}, through
the clouds. The overall phenomenon depends on different interactive
factors, which includes cloud thickness, constitution and type; and the
relative position of the sun and the clouds
\citep{Gueymard2017, Veerman2022}. As such, there are multiple
contributing mechanisms that are responsible for the observed irradiance
enhancements.

On multiple sites, cloud enhancements have been reported, to be able to
briefly exceed the value of the solar constant, resulting of clear
indexes above unit. A summary of extreme enchantment cases (ECE) have
been compiled by \citet{Martins2022}. There are also, some practical
implication of the cloud enchantments. The intensity and duration of
enhancements can effect the efficiency and stability of photovoltaic
power production \citep{Lappalainen2020, Jaervelae2020}, and extreme
enhancements cases, have the potential to compromise the integrity of
photovoltaic plants infrastructure \citep{DoNascimento2019}. It has also
been demonstrated, that these events can interfere in the comparison of
ground data and satellite observations \citep{Damiani2018}

Methods of identification cloud enchantment events in the literature,
usually include the use of a simulated clear sky radiation as baseline,
that is combined with an appropriate threshold or some other statistical
characteristics, and in some cases, with visual methods with a sky cam
\citep[ and references therein]{Vamvakas2020, Mol2023}.

In this study, we evaluated the effects CE on the GHI by studding the
occurrences, their intensity, and their duration in a thirty-year period
at the city of Thessaloniki. We used modeled clear sky irradiance, as a
baseline to identify cloud enhancements. We were able to determine some
trends of the phenomenon, it's Climatology, and some of their general
characteristics. We weren't able to find a comparable study that
provides trends for similar long term dataset, as ours. The recording of
the GHI signal is the mean of one-minute. Thus, the minimum resolution
of CE events, in this study is one minute.

In the relative bibliography different definition have been used for
these events, some of the are summarized by \citet{Gueymard2017}. Here,
we defined as CE events, the cases when the measured GHI at ground
level, exceeds the expected value under clear-sky conditions. Similar,
we define as extreme cloud enhancement events (ECE), the cases when GHI,
exceeds the Total Solar Irradiance (TSI) on horizontal plane at ground
level. Although the duration of these bursts varies, from instantaneous
to several minutes, here we are constrained by the recorded data, to
one-minute steps.

\hypertarget{data-and-methodology}{%
\section{Data and methodology}\label{data-and-methodology}}

\hypertarget{ghi-data}{%
\subsection{GHI data}\label{ghi-data}}

The monitoring site is operating in the Laboratory of Atmospheric
Physics of the Aristotle University of Thessaloniki
(\(40^\circ\,38'\,\)N, \(22^\circ\,57'\,\)E, \(80\,\)m~a.s.l.). In this
study we present data from the period 13~April 1993 to 31~December 2023.
The GHI data were measured with a Kipp~\& Zonen CM-21 pyranometer.
During the study period, the pyranometer was independently calibrated
three times at the Meteorologisches Observatorium Lindenberg, DWD,
verifying that the stability of the instrument's sensitivity was better
than \(0.7\,\%\) relative to the initial calibration by the
manufacturer. For the acquisition of radiometric data, the signal of the
pyranometer was sampled at a rate of \(1\,\text{Hz}\). The mean and the
standard deviation of these samples were calculated and recorded for
every minute. The measurements were corrected for the zero offset
(``dark signal'' in volts), which was calculated by averaging all
measurements recorded for a period of \(3\,\text{h}\), before (morning)
or after (evening) the Sun reaches an elevation angle of \(-10^\circ\).
The signal was converted to irradiance using the ramped value of the
instrument's sensitivity between subsequent calibrations.

To further improve the quality of the irradiance data, a manual
screening was performed, in order to remove inconsistent and erroneous
recordings that can occur stochastically or systematically during the
long operation of the instrument. The manual screening was aided by a
radiation data quality assurance procedure, adjusted for the site, which
was based on the methods of Long and Shi~\citep{Long2006, Long2008a}.
Thus, problematic recordings have been excluded from further processing.
Although it is impossible to detect all false data, the large number of
available data, and the aggregation scheme we used, ensures the quality
of the radiation measurements used in this study. To preserve an
unbiased representation of the data we applied a constraint, similar the
one used by \citet{CastillejoCuberos2020}. Where, for each valid hour of
day, there must exist at least 45 minutes of valid measurements,
including nighttime, when near sunrise and sunset. Days with less than 5
valid hours were rejected completely. Furthermore, due to the
significant measurement uncertainty, when the Sun is near the horizon,
and due to some systematic obstructions by nearby buildings, we have
excluded all measurements with solar zenith angle (SZA) greater than
\(78^\circ\).

\hypertarget{cloud-enhancement-detection}{%
\subsection{Cloud enhancement
detection}\label{cloud-enhancement-detection}}

To be able to detect the CE cases, we had to establish a baseline, above
which we characterized each data point as an enhancement event, by
estimating the occurring over irradiance (OIR). The OIR, is defined as
the irradiance difference of the measured one-minute \(\text{GHI}_i\)
from the CE identification criterion
(\(\text{OIR}_i = \text{GHI}_i - \text{GHI}_\text{CSlim}\)), as defined
in Equation\nobreakspace\ref{eq:CE4}. To have an estimation, and a first
insight on the phenomenon, we experimented with two simple approaches
for the reference. The Haurwitz's model \citep{Haurwitz1945}, which is
as simple clear sky model, and we had already adjusted and had good fit
with our data \citep{Natsis2023}, and the TSI at the top of the
atmosphere. We have tested both cases by using an appropriate relative
threshold and/or an additional constant offset. The initial results,
showed that we can detect a big portion of the CE events. These results
were helpful, as they are independent from unknown factors, and helped
us to establish some criteria to further improve the CE identification.
It was evident, by inspecting the daily plot of irradiance, that changes
on the atmospheric conditions introduced numerous false positive and
false negative results. To produce a more accurate reference, we had to
take into account more factors that effect the clear sky radiation. So
we used a radiation transfer model in order to include the effects of
ADO and the water vapors.

\hypertarget{modeled-clear-sky-irradiance}{%
\subsection{Modeled clear Sky
Irradiance}\label{modeled-clear-sky-irradiance}}

\hypertarget{climatology-of-clear-sky-irradiance}{%
\subsubsection{Climatology of clear sky
irradiance}\label{climatology-of-clear-sky-irradiance}}

We approximated the expected clear sky GHI by using the Libradtran
radiation transfer model \citep{Emde2016}, a similar approach, was also
used by \citet{Vamvakas2020} for creating a clear sky reference for
studding cloud enhancement events. Main factors responsible for the
attenuation of the broadband downward solar radiation in the atmosphere
are the aerosols and the water vapors. Because of the lack of
observational data for the whole period, we used some long term
climatological data to recreate the whole period. Fortunately, our site
participates in the Aerosol Robotic Network (AERONET)
\citep{Giles2019, Buis1998}, as there is in operation a Cimel photometer
since 2003, collocated with the CM-21 pyranometer. The mean monthly AOD
on different wavelengths are provided by AERONET, along with the
equivalent water column height in the atmosphere.

For completeness, we will describe here the main points of the radiation
simulation procedure. We used as input the spectrum of
\citet{Kurucz1994} in the range \(280\) to \(2500\,\text{nm}\), with the
Libradtran radiation transfer solver ``disort'' on a ``pseudospherical''
geometry and the ``LOWTRAN'' gas parameterization. For each combination
of conditions we use a SZA step of \(0.2^\circ\). For the atmospheric
characteristics, we iterated on combinations of AOD at
\(500\,\text{nm}\) (\(\tau_{500\text{nm}}\)) with additional offsets of
\(\pm1\) and \(\pm2\sigma\), and water column (\(w\)), also with offsets
of \(\pm1\) and \(\pm2\sigma\). We applied them on two atmospheric
profiles, from the Air Force Geophysics Laboratory (AFGL). The ``AFGL
atmospheric constituent profile, midlatitude summer'' (afglms) and the
``AFGL atmospheric constituent profile, midlatitude winter'' (afglmw)
\citep{Anderson1986}.

To create a look-up table, that aligns with our dataset, we applied some
adjustments. To account for the Sun's variability, in our one-minute GHI
measurements, we adapted each modeled value by scaling the model's input
spectrum integral, to the corresponding TSI, provided by NOAA
\citep{Coddington2005}. Also, we account for the effect of the Earth --
Sun distance on the irradiance, by using the distance calculated by the
Astropy \citep{AstropyCollaboration2022} software library. As needed, we
interpolate the resulting irradiances to the SZA of our measurements.
For each period of the year, we used the appropriate atmospheric profile
(afglms or afglmw). Finally, we calculated the clear sky irradiance
value at the horizontal plane. Thus, we were able to emulate different
atmospheric condition and levels of atmospheric clearness for the
climatological conditions of the site. With this method, the modeled
clear sky irradiances can be directly compared to each measured
one-minute value of GHI, for different conditions of atmospheric
clearness.

\hypertarget{long-term-change-of-clear-sky-irradiance}{%
\subsubsection{Long term change of clear sky
irradiance}\label{long-term-change-of-clear-sky-irradiance}}

The above clear sky reference values are not enough to describe the long
term variation, due to the changes of the atmospheric constituents,
mainly AOD. As observed by \citet{Natsis2023}, there is a long term
brightening effect, over Thessaloniki for the studied period. In order
to better incorporate this change, we used, estimations of AOD change
derived from two sources. For 1993 -- 2005, \citet{Kazadzis2007} have
found a change of \(-3.8\,\%\) per year at AOD on \(340\,\text{nm}\),
using a Brewer spectrophotometer. For 2005 to 2023, we derived the trend
from the available monthly climatological data of AERONET.

To create a unified trend for the long term change of the clear sky
irradiance, we simulated the values of AOD derived from those sources
with Libradtran. For both inputs, we used the AOD at \(500\,\text{nm}\),
which was inferred by the available Ångström coefficients. We choose the
SZA of \(55^\circ\) as a representative value for all the runs.

According to \citet{Kazadzis2007}, in the period 1997 -- 2005, the mean
AOD of \(340\,\text{nm}\) was calculated to \(0.403\) with the use of a
Brewer spectrophotometer, with a linear trend of \(-3.8±0.93\,\%\) per
year. This corresponds to a change of \(0.0153\) per year for the AOD of
\(340\,\text{nm}\) . Using the alpha Ångström coefficient of
\(\alpha = 1.6\), this translates to a change in Ångström beta
coefficient of \(\beta=0.00272\) per year (or \(\beta=0.084\) in 1997
and \(\beta=0.059\) in 2005). By interpolating the resulting clear sky
irradiance, we have a trend of \(+0.21\,\%\) per year for the clear sky
irradiance. For the period 2005 -- 2005 we used the mean monthly values
from AERONET, with a similar simulation scheme. We computed the trend of
clear sky radiation for each month of the year, which gave us a mean
clear sky radiation trend of \(+0.14\,\%\) per year. We applied this
long term change to the clear sky irradiance, in order to create a more
realistic representation of the irradiance for the whole study period
(Figure\nobreakspace{}\ref{fig:CS-change}).

\begin{figure}

{\centering \includegraphics[width=0.5\linewidth]{../images/P_CS_change-1} 

}

\caption{Simulated long term change of the clear sky irradiance relevant to its climatological values, due to changed of AOD.}\label{fig:CS-change}
\end{figure}

\hypertarget{ce-criteria-investigation}{%
\subsection{CE Criteria investigation}\label{ce-criteria-investigation}}

Our main focus was to positively identify over irradiance events related
to CEs. For this reason we had to select an appropriate set of
atmospheric parameterization, in Libradtran, and define an relevant
threshold for the CE identification. The selection of the atmospheric
parameterization will provide us with a reference, that has a good
agreement with the clear sky conditions. Additional, with the
implementation of the long term change of AOD, as described above, it
will be able to align well with most of the natural variation of the
physical quantity. The selection of adequate threshold, will allow us to
have a robust method to distinguish the CE cases, which will compensate
for the limited accuracy of the input data, and the natural variability
of the atmosphere.

First, we evaluated the performance of the modelled clear radiation, for
each of the atmospheric level of clearness we have created as reference,
in relation to the measured GHI data, and choose as a representative of
the clear sky radiation, the case where
\(\tau_{\text{cs}} = \tau_{500\text{nm}} - 1\sigma\) and
\(w_{h\text{cs}} = w_h - 1\sigma\) (\(\text{GHI}_\text{CSm}\)). These
values represent a typical atmosphere in Thessaloniki, with low load of
aerosols and humidity, which are the main factor that attenuate the GHI,
excluding clouds. To create the upper limit of CE identification, we
used a value relative to clear sky irradiance (\(4\,\%\)) with an
additional constant offset of \(15\,\text{W}/\text{m}^2\), as described
in Equation\nobreakspace\ref{eq:CE4}). \begin{equation}
\text{CE} : \text{E}_\text{i} > 15 + 1.04 \cdot \text{E}_\text{CSm,i} \,\,[\text{W}/\text{m}^2] \label{eq:CE4}
\end{equation} where: \(\text{E}\) the measured irradiance,
\(\text{E}_\text{CSm}\) the selected modelled clear sky irradiance, and
\(i\) each of the one-minute observation. This is the criterion of our
CE identification.

In the process of defining the above details, we implemented an
empirical method, by manual inspection of the CE identification results,
on selected days from the whole dataset. We used seven sets of days,
with characteristics relevant to the efficiency of the identification
threshold. These sets were random groups of about 20 to 30 days with the
following characteristics: (a) the strongest over irradiance CE events,
(b) the largest daily total over irradiance, (c) absence of clouds (by
implementing a clear sky identification algorithm as discussed in
\citet{Natsis2023}), (d) absence of clouds and absence of EC events, (e)
with at least \(60\,\%\) of the day length without clouds and presence
of EC events, (h) random selected days and (i) some manual selected days
that were included during the manual inspection process. Where it was
needed, for some of the edge cases, we also used images from a sky-cam,
to further aid the decisions of the manual inspection.

The definition of the CE events with this method, has a level of
subjectivity. The actual clear sky irradiance, is not known, and can
only be approximated. Although, because our main focus is to be able to
identify the OIR created by the clouds, we expect that we were able to
detect all the major events of OIR due to CE. Where, some CE events,
with very low OIR, may be not detected, these are few and their
magnitude will not effect our results.

Another aspect of the CE events, that is often reported in the relative
bibliography, are cases of extreme cloud enhancement (ECE). These are
cases of CE, where the measured intensity of the irradiance, exceeds the
equivalent TSI on the top of the atmosphere, and detected according to
the Equation\nobreakspace{}\ref{eq:ECE}. \begin{equation}
\text{ECE}: \text{GHI}_\text{i} > \cos(\theta) \times E_{i\odot} / r_{i}^2
\label{eq:ECE}
\end{equation} where: \(\theta\) the solar zenith angle, \(E_{\odot}\)
the solar constant, \(r\) the Sun -- Earth distance, and \(i\) each of
the one-minute observation.

To give a better understanding of the CE identification procedure, we
provided an example of the CE identification for a selected day, in the
Appendix (Figure\nobreakspace{}\ref{fig:example-day}), where the use of
the clear sky reference and the identification threshold are shown,
along with the relevant measurements. Similar, we provided an example
plot of data for one year, where the relation of the GHI and clear sky
reference is shown, along with the OIR intensity (Appendix
Figure\nobreakspace{}\ref{fig:example-year}).

\hypertarget{results}{%
\section{Results}\label{results}}

Our dataset, after the data selection processing, consists of 6144534
records of GHI measurements, of which \(1.764\,\%\) were identified as
CE events and \(0.036\,\%\) as ECE events. The highest recorded GHI due
to CE was \(1416.6\,W/m^2\) on 24~May 2007, and had the maximum value of
OIR at \(345.9\,W/m^2\). The relative stronger CE event was \(49.7\,\%\)
above the clear sky threshold, on 28~October 2016.

\hypertarget{long-term-trends}{%
\subsection{Long term trends}\label{long-term-trends}}

We computed the daily trend of the mean OIR of CE using a first-order
autoregressive model with a lag of one day, with the `maximum
likelihood' fitting method \citep{Gardner1980, Jones1980} by
implementing the function `arima' from the library `stats' of the R
programming language \citep{RCT2023}. The trends are reported together
with the \(2\sigma\) error. We observed a negligible change of
\(0.087\pm 0.08\,\text{W}/\text{m}^2/\text{year}\) on the mean OIR.
Although, the previous result is closer to the raw data, we preferred to
present the annual statistics, that give a more clear picture about the
long term CE trends. Hence, we calculated the annual values from the
one-minute measurements. Despite the very weak statistical significance
of the mean OIR trend, the annual number of CE occurrences shows a
steady increase of \(45.6\pm 21.9\,\text{cases}/\text{year}\)
(Figure\nobreakspace{}\ref{fig:P-energy-N}). Subsequently, the mean
annual energy of the CE events is increasing with a rate of
\(116.9\pm 67.8\,\text{kj}/\text{year}\)
(Figure~\ref{fig:P-energy-sum}), witch follows the trend of the number
of CE occurrences. We have to note that the energy related to the CE
events can not be directly linked with the total energy balance of the
atmosphere. The net solar radiation of the region is not increased, but
rather redistributed through the CE events. This also is depicted by the
``instantaneous'' values of irradiance (average over one minute), that
can exceed the equivalent clear sky irradiance to a considerable level.

\begin{figure}

{\centering \includegraphics[width=0.5\linewidth]{../images/P_energy-6} 

}

\caption{Trend of yearly CE number of occurancies.}\label{fig:P-energy-N}
\end{figure}

\begin{figure}

{\centering \includegraphics[width=0.5\linewidth]{../images/P_energy-5} 

}

\caption{Trend of the yearly excess energy due to CE over irradiance}\label{fig:P-energy-sum}
\end{figure}

\hypertarget{cloud-enhancements-climatology}{%
\subsection{Cloud enhancements
climatology}\label{cloud-enhancements-climatology}}

Another interesting aspect of the CE cases occurrences, is their
seasonal cycle. In
Figure\nobreakspace{}\ref{fig:relative-month-occurancies}, we have the
monthly box plot (whisker plot), where the values have been normalized
by the highest median value, that occurs on June. Although the number of
occurrences has a wide spread throughout the study period, the most
active period of CE occurrences is during May and June. During the
Winter (December -- February) the CE cases are about \(25\,\%\) of the
maximum. For the rest months, the occurrences gradually ramp between the
maximum and minimum. This seasonality is a combinations of the clouds
occurrences, and the relevant position of the sun during each season of
the year. Unfortunately, the lack of detailed data on cloud formation,
type and location is not allowing further analysis.

\begin{figure}

{\centering \includegraphics[width=0.5\linewidth]{../images/clim_CE_month_norm_MAX_median_N-2} 

}

\caption{Seasonal statistics of the number of CE occurancies for each month, normalized to the maximum occurancies on June. The box represents the values of the low $25\,\%$ percentile to $75\,\%$ percentile, where the thick horizontal line inside is the mean, the verical lines extend to the macimum and minimum vales, the dots are outlier values, and the rhombus is the mean.}\label{fig:relative-month-occurancies}
\end{figure}

The distribution of the OIR intensity, follows an exponential decline
Figure\nobreakspace{}\ref{fig:ovir-distribution}, where there is an
inverse relation between the CE events frequency and their intensity.
This is expected, as the stronger the CE events become, the rarer the
particular atmospheric and sun conditions can occur. This fact, is
indicative to the magnitude and the probability of the expected OIRs
events over Thessaloniki. Similar distribution of CE occurrences have
been reported by \citet{Vamvakas2020}, where the magnitude of OIR were
higher due to the location of the city of Patras, \(2.5^\circ\) closer
to the equator.

\begin{figure}

{\centering \includegraphics[width=0.5\linewidth]{../images/P-relative-distribution-diff-1} 

}

\caption{Distribution of CE over irradiance magnitude.}\label{fig:ovir-distribution}
\end{figure}

\hypertarget{groups-of-cloud-enhancement}{%
\subsection{Groups of cloud
enhancement}\label{groups-of-cloud-enhancement}}

In order to study the total duration of the CE events, we grouped the
one minute CE events to continuous CE groups (CEG). Thus, a CEG consists
of one or more successive CE cases, and can represent cloud enhancement
conditions spanning a duration of multiple minutes. We have identified
28468 groups of CE in the whole period of study, where the group with
the longest duration lasted 140 minutes on 07~July 2013. By examining
the frequency distribution of the CEG durations
(Figure\nobreakspace{}\ref{fig:ceg-duration-distribution}), we can
conclude, that although we detected some CEG with durations longer than
an hour, about \(80\,\%\) of the CEG have a duration of less than 5
minutes.

\begin{figure}

{\centering \includegraphics[width=0.5\linewidth]{../images/groups-1} 

}

\caption{Distribution of CE groups duration in minutes.}\label{fig:ceg-duration-distribution}
\end{figure}

The relation between the duration and the mean OIR intensity of the
groups have also been studied (Figure\nobreakspace{}\ref{fig:group-2d}).
We observed that the GCE events tend to have either long duration, or
large intensity. Events with strong enhancement and long duration are
very rare. Similar results on this relation, have been reported by
\citet{Zhang2018}, on a study using a far higher sampling rate than
ours.

\begin{figure}

{\centering \includegraphics[width=0.5\linewidth]{../images/P-groups-bin2d-1} 

}

\caption{Relation of the mean over irradiance and duration of GCE, where the color scale donotes the frequency of the respected events.}\label{fig:group-2d}
\end{figure}

\hypertarget{extreme-cloud-enhancement-events}{%
\subsection{Extreme cloud enhancement
events}\label{extreme-cloud-enhancement-events}}

An aspect of the CE events that is commonly reported and has some
significance on the solar energy production infrastructure are the
extreme CE events (ECE). Where solar irradiance exceeds the expected
irradiance on top of the atmosphere
(Equation\nobreakspace{}\ref{eq:ECE}). Analogous to
Figure\nobreakspace{}\ref{fig:relative-month-occurancies} we have
computed the distribution of the number of occurrences of ECE events by
month in Figure\nobreakspace{}\ref{fig:relative-month-occurancies-ECE}.
The most active period for ECE events is in the spring and the start of
the summer (March -- June), followed by a period in the late fall
(September and October). This is probably related to the weather
characteristics of these periods, where there are continuous
alternations between clear sky periods and clouds.

\begin{figure}

{\centering \includegraphics[width=0.5\linewidth]{../images/clim_ECE_month_norm_MAX_median_N-2} 

}

\caption{Seasonal statistics of the number of ECE occurancies for each month, normalized to the maximum occurancies on March. The box represents the values of the low $25\,\%$ percentile to $75\,\%$ percentile, where the thick horizontal line inside is the mean, the verical lines extend to the macimum and minimum vales, the dots are outlier values, and the rhombus is the mean.}\label{fig:relative-month-occurancies-ECE}
\end{figure}

The distribution of the ECE events
(Figure\nobreakspace{}\ref{fig:P-extreme-distribution}), shows that
there are rare cases where the OIR can exceed the TSI even more than
\(400\,\text{W}/\text{m}^2\), with the \(75\,\%\) of the cases to be
below \(200\,\text{W}/\text{m}^2\). Those finds are in accordance with
results form \citet{Vamvakas2020}, with the difference that the OIR
values are lower for Thessaloniki.

\begin{figure}

{\centering \includegraphics[width=0.5\linewidth]{../images/P-extreme-distribution-1} 

}

\caption{Distribution of ECE above clear sky threshold, for cases that are exceeding the TSI.}\label{fig:P-extreme-distribution}
\end{figure}

\hypertarget{discussion-and-conclusions}{%
\section{Discussion and conclusions}\label{discussion-and-conclusions}}

By creating a clear sky approximation of the GHI, which represents the
long- and short-term variation of the expected clear sky GHI, we were
able to identify cases of CE events. After analyzing the CE cases, we
found an increase of \(45.6\pm 21.9\,\text{cases}/\text{year}\), with
the mean annual total energy of the CE events increasing with a rate of
\(116.9\pm 67.8\,\text{kj}/\text{year}\). The most active season of CE
events over Thessaloniki is concentrated on early summer, on May and
June.

The magnitude of the ECE events doesn't exceeds the values reported from
sites with more favourable conditions for the phenomenon. Although, the
climatological characteristic of the ECE events, showed that the most
active months are spread on half of the year (March -- June and
September and October). We found that CE conditions, can have a duration
of more than an hour in rare cases, with the bulk of the cases having a
duration under 5 minutes. Some of the characteristics of CE and ECE
events we analysed, have strong similarities with results by
\citet{Vamvakas2020}, for a city southern of Thessaloniki, with the
analogues differences on the intensity of the solar radiation.

An interpretation of the CE trends, shows that the interaction of GHI
with the clouds, through this 30 year period, is a dynamic phenomenon
that's needs further investigation. Although, to approach it, we need
more long term observations of AOD and clouds characteristics.

\hypertarget{appendix}{%
\section*{Appendix}\label{appendix}}
\addcontentsline{toc}{section}{Appendix}

\begin{figure}

{\centering \includegraphics[width=0.5\linewidth]{../images/example_days-18} 

}

\caption{Example of cloud indentification data for 2019-07-11. Where are denoted with red circles all the detected CE events, the cloud intearaction with the GHI (blue crosses), and the thesholds of CE and the threshold of ECE events with red and black line respectively.}\label{fig:example-day}
\end{figure}

\begin{figure}

{\centering \includegraphics[width=0.5\linewidth]{../images/example_years-13} 

}

\caption{Example of the relation of the measured GHI versus the reference clear sky irradiance for the year 2005, where the intesity of the detected over irradianse (OIR) is color coded.}\label{fig:example-year}
\end{figure}

\bibliography{bibliography.bib}


\end{document}
