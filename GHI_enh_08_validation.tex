% Options for packages loaded elsewhere
\PassOptionsToPackage{unicode}{hyperref}
\PassOptionsToPackage{hyphens}{url}
\PassOptionsToPackage{dvipsnames,svgnames,x11names}{xcolor}
%
\documentclass[
  10pt,
  a4paper,oneside]{article}
\usepackage{amsmath,amssymb}
\usepackage{iftex}
\ifPDFTeX
  \usepackage[T1]{fontenc}
  \usepackage[utf8]{inputenc}
  \usepackage{textcomp} % provide euro and other symbols
\else % if luatex or xetex
  \usepackage{unicode-math} % this also loads fontspec
  \defaultfontfeatures{Scale=MatchLowercase}
  \defaultfontfeatures[\rmfamily]{Ligatures=TeX,Scale=1}
\fi
\usepackage{lmodern}
\ifPDFTeX\else
  % xetex/luatex font selection
\fi
% Use upquote if available, for straight quotes in verbatim environments
\IfFileExists{upquote.sty}{\usepackage{upquote}}{}
\IfFileExists{microtype.sty}{% use microtype if available
  \usepackage[]{microtype}
  \UseMicrotypeSet[protrusion]{basicmath} % disable protrusion for tt fonts
}{}
\makeatletter
\@ifundefined{KOMAClassName}{% if non-KOMA class
  \IfFileExists{parskip.sty}{%
    \usepackage{parskip}
  }{% else
    \setlength{\parindent}{0pt}
    \setlength{\parskip}{6pt plus 2pt minus 1pt}}
}{% if KOMA class
  \KOMAoptions{parskip=half}}
\makeatother
\usepackage{xcolor}
\usepackage[left=0.5in,right=0.5in,top=0.5in,bottom=0.5in]{geometry}
\usepackage{color}
\usepackage{fancyvrb}
\newcommand{\VerbBar}{|}
\newcommand{\VERB}{\Verb[commandchars=\\\{\}]}
\DefineVerbatimEnvironment{Highlighting}{Verbatim}{commandchars=\\\{\}}
% Add ',fontsize=\small' for more characters per line
\usepackage{framed}
\definecolor{shadecolor}{RGB}{248,248,248}
\newenvironment{Shaded}{\begin{snugshade}}{\end{snugshade}}
\newcommand{\AlertTok}[1]{\textcolor[rgb]{0.94,0.16,0.16}{#1}}
\newcommand{\AnnotationTok}[1]{\textcolor[rgb]{0.56,0.35,0.01}{\textbf{\textit{#1}}}}
\newcommand{\AttributeTok}[1]{\textcolor[rgb]{0.13,0.29,0.53}{#1}}
\newcommand{\BaseNTok}[1]{\textcolor[rgb]{0.00,0.00,0.81}{#1}}
\newcommand{\BuiltInTok}[1]{#1}
\newcommand{\CharTok}[1]{\textcolor[rgb]{0.31,0.60,0.02}{#1}}
\newcommand{\CommentTok}[1]{\textcolor[rgb]{0.56,0.35,0.01}{\textit{#1}}}
\newcommand{\CommentVarTok}[1]{\textcolor[rgb]{0.56,0.35,0.01}{\textbf{\textit{#1}}}}
\newcommand{\ConstantTok}[1]{\textcolor[rgb]{0.56,0.35,0.01}{#1}}
\newcommand{\ControlFlowTok}[1]{\textcolor[rgb]{0.13,0.29,0.53}{\textbf{#1}}}
\newcommand{\DataTypeTok}[1]{\textcolor[rgb]{0.13,0.29,0.53}{#1}}
\newcommand{\DecValTok}[1]{\textcolor[rgb]{0.00,0.00,0.81}{#1}}
\newcommand{\DocumentationTok}[1]{\textcolor[rgb]{0.56,0.35,0.01}{\textbf{\textit{#1}}}}
\newcommand{\ErrorTok}[1]{\textcolor[rgb]{0.64,0.00,0.00}{\textbf{#1}}}
\newcommand{\ExtensionTok}[1]{#1}
\newcommand{\FloatTok}[1]{\textcolor[rgb]{0.00,0.00,0.81}{#1}}
\newcommand{\FunctionTok}[1]{\textcolor[rgb]{0.13,0.29,0.53}{\textbf{#1}}}
\newcommand{\ImportTok}[1]{#1}
\newcommand{\InformationTok}[1]{\textcolor[rgb]{0.56,0.35,0.01}{\textbf{\textit{#1}}}}
\newcommand{\KeywordTok}[1]{\textcolor[rgb]{0.13,0.29,0.53}{\textbf{#1}}}
\newcommand{\NormalTok}[1]{#1}
\newcommand{\OperatorTok}[1]{\textcolor[rgb]{0.81,0.36,0.00}{\textbf{#1}}}
\newcommand{\OtherTok}[1]{\textcolor[rgb]{0.56,0.35,0.01}{#1}}
\newcommand{\PreprocessorTok}[1]{\textcolor[rgb]{0.56,0.35,0.01}{\textit{#1}}}
\newcommand{\RegionMarkerTok}[1]{#1}
\newcommand{\SpecialCharTok}[1]{\textcolor[rgb]{0.81,0.36,0.00}{\textbf{#1}}}
\newcommand{\SpecialStringTok}[1]{\textcolor[rgb]{0.31,0.60,0.02}{#1}}
\newcommand{\StringTok}[1]{\textcolor[rgb]{0.31,0.60,0.02}{#1}}
\newcommand{\VariableTok}[1]{\textcolor[rgb]{0.00,0.00,0.00}{#1}}
\newcommand{\VerbatimStringTok}[1]{\textcolor[rgb]{0.31,0.60,0.02}{#1}}
\newcommand{\WarningTok}[1]{\textcolor[rgb]{0.56,0.35,0.01}{\textbf{\textit{#1}}}}
\usepackage{longtable,booktabs,array}
\usepackage{calc} % for calculating minipage widths
% Correct order of tables after \paragraph or \subparagraph
\usepackage{etoolbox}
\makeatletter
\patchcmd\longtable{\par}{\if@noskipsec\mbox{}\fi\par}{}{}
\makeatother
% Allow footnotes in longtable head/foot
\IfFileExists{footnotehyper.sty}{\usepackage{footnotehyper}}{\usepackage{footnote}}
\makesavenoteenv{longtable}
\usepackage{graphicx}
\makeatletter
\def\maxwidth{\ifdim\Gin@nat@width>\linewidth\linewidth\else\Gin@nat@width\fi}
\def\maxheight{\ifdim\Gin@nat@height>\textheight\textheight\else\Gin@nat@height\fi}
\makeatother
% Scale images if necessary, so that they will not overflow the page
% margins by default, and it is still possible to overwrite the defaults
% using explicit options in \includegraphics[width, height, ...]{}
\setkeys{Gin}{width=\maxwidth,height=\maxheight,keepaspectratio}
% Set default figure placement to htbp
\makeatletter
\def\fps@figure{htbp}
\makeatother
\setlength{\emergencystretch}{3em} % prevent overfull lines
\providecommand{\tightlist}{%
  \setlength{\itemsep}{0pt}\setlength{\parskip}{0pt}}
\setcounter{secnumdepth}{-\maxdimen} % remove section numbering
\usepackage{caption}
\usepackage{float}
\usepackage{placeins}
\captionsetup{font=small}
\ifLuaTeX
  \usepackage{selnolig}  % disable illegal ligatures
\fi
\usepackage{bookmark}
\IfFileExists{xurl.sty}{\usepackage{xurl}}{} % add URL line breaks if available
\urlstyle{same}
\hypersetup{
  pdftitle={Enhancement of SDR in Thessaloniki},
  pdfauthor={Natsis Athanasios; Alkiviadis Bais},
  colorlinks=true,
  linkcolor={Maroon},
  filecolor={Maroon},
  citecolor={Blue},
  urlcolor={Blue},
  pdfcreator={LaTeX via pandoc}}

\title{Enhancement of SDR in Thessaloniki}
\author{Natsis Athanasios\footnote{Laboratory of Atmospheric Physics, AUTH, \href{mailto:natsisphysicist@gmail.com}{\nolinkurl{natsisphysicist@gmail.com}}} \and Alkiviadis Bais\footnote{Laboratory of Atmospheric Physics, AUTH}}
\date{2024-09-06}

\begin{document}
\maketitle
\begin{abstract}
Study of GHI enchantment.
\end{abstract}

{
\hypersetup{linkcolor=}
\setcounter{tocdepth}{4}
\tableofcontents
}
\begin{verbatim}
Drop all enhacement cases
\end{verbatim}

\begin{verbatim}
Drop all clouds
\end{verbatim}

\begin{figure}[H]

{\centering \includegraphics[width=1\linewidth]{GHI_enh_08_validation_files/figure-latex/unnamed-chunk-4-1} 

}

\caption{ - empty caption - }\label{fig:unnamed-chunk-4}
\end{figure}

\FloatBarrier

\section{Plot some days with strong enhancement cases}\label{plot-some-days-with-strong-enhancement-cases}

\FloatBarrier

\subsection{Days with sunny day}\label{days-with-sunny-day}

\FloatBarrier

\subsection{Days with sunny enhancement day}\label{days-with-sunny-enhancement-day}

\FloatBarrier

\subsection{Days with cloudy day}\label{days-with-cloudy-day}

\FloatBarrier

\subsection{Days with random day}\label{days-with-random-day}

\FloatBarrier

\subsection{Days with manual test days}\label{days-with-manual-test-days}

\begin{figure}[H]

{\centering \includegraphics[width=1\linewidth]{GHI_enh_08_validation_files/figure-latex/example-days-1} 

}

\caption{ - empty caption - }\label{fig:example-days-1}
\end{figure}

\begin{figure}[H]

{\centering \includegraphics[width=1\linewidth]{GHI_enh_08_validation_files/figure-latex/example-days-2} 

}

\caption{ - empty caption - }\label{fig:example-days-2}
\end{figure}

\begin{figure}[H]

{\centering \includegraphics[width=1\linewidth]{GHI_enh_08_validation_files/figure-latex/example-days-3} 

}

\caption{ - empty caption - }\label{fig:example-days-3}
\end{figure}

\begin{figure}[H]

{\centering \includegraphics[width=1\linewidth]{GHI_enh_08_validation_files/figure-latex/example-days-4} 

}

\caption{ - empty caption - }\label{fig:example-days-4}
\end{figure}

\begin{figure}[H]

{\centering \includegraphics[width=1\linewidth]{GHI_enh_08_validation_files/figure-latex/example-days-5} 

}

\caption{ - empty caption - }\label{fig:example-days-5}
\end{figure}

\newpage

\begin{Shaded}
\begin{Highlighting}[]
\FunctionTok{plot}\NormalTok{(KEEP[, Low\_B.Low\_W.glo, wattGLB], }\AttributeTok{xlab =} \StringTok{"Low\_B.Low\_W.glo Clear Sky"}\NormalTok{)}
\FunctionTok{title}\NormalTok{(}\FunctionTok{paste}\NormalTok{(}\StringTok{"Days:"}\NormalTok{, }\FunctionTok{length}\NormalTok{(}\FunctionTok{unique}\NormalTok{(KEEP[, Day])), }\StringTok{"Day fill:"}\NormalTok{, day\_fill, }\StringTok{"Points:"}\NormalTok{, KEEP[, }\FunctionTok{sum}\NormalTok{(}\SpecialCharTok{!}\FunctionTok{is.na}\NormalTok{(wattGLB))]))}
\FunctionTok{abline}\NormalTok{(}\AttributeTok{a =} \DecValTok{0}\NormalTok{, }\AttributeTok{b =} \DecValTok{1}\NormalTok{, }\AttributeTok{col =} \StringTok{"green"}\NormalTok{)}
\end{Highlighting}
\end{Shaded}

\begin{figure}[H]

{\centering \includegraphics[width=1\linewidth]{GHI_enh_08_validation_files/figure-latex/unnamed-chunk-6-1} 

}

\caption{ - empty caption - }\label{fig:unnamed-chunk-6}
\end{figure}

\begin{Shaded}
\begin{Highlighting}[]
\FunctionTok{print}\NormalTok{(}\FunctionTok{cor.test}\NormalTok{(KEEP}\SpecialCharTok{$}\NormalTok{wattGLB, KEEP}\SpecialCharTok{$}\NormalTok{Low\_B.Low\_W.glo, }\AttributeTok{method =} \FunctionTok{c}\NormalTok{(}\StringTok{"pearson"}\NormalTok{)))}
\end{Highlighting}
\end{Shaded}

\begin{verbatim}

    Pearson's product-moment correlation

data:  KEEP$wattGLB and KEEP$Low_B.Low_W.glo
t = 7903.5, df = 345175, p-value < 2.2e-16
alternative hypothesis: true correlation is not equal to 0
95 percent confidence interval:
 0.9972301 0.9972667
sample estimates:
      cor 
0.9972485 
\end{verbatim}

\begin{Shaded}
\begin{Highlighting}[]
\FunctionTok{print}\NormalTok{(}\FunctionTok{lm}\NormalTok{(KEEP[, wattGLB, Low\_B.Low\_W.glo]))}
\end{Highlighting}
\end{Shaded}

\begin{verbatim}

Call:
lm(formula = KEEP[, wattGLB, Low_B.Low_W.glo])

Coefficients:
(Intercept)      wattGLB  
     18.357        1.038  
\end{verbatim}

\newpage

\begin{Shaded}
\begin{Highlighting}[]
\FunctionTok{plot}\NormalTok{(KEEP[, Enhanc\_C\_4\_ref, wattGLB], }\AttributeTok{xlab =} \StringTok{"CE threshold"}\NormalTok{)}
\FunctionTok{title}\NormalTok{(}\FunctionTok{paste}\NormalTok{(}\StringTok{"Days:"}\NormalTok{, }\FunctionTok{length}\NormalTok{(}\FunctionTok{unique}\NormalTok{(KEEP[, Day])), }\StringTok{"Day fill:"}\NormalTok{, day\_fill, }\StringTok{"Points:"}\NormalTok{, KEEP[, }\FunctionTok{sum}\NormalTok{(}\SpecialCharTok{!}\FunctionTok{is.na}\NormalTok{(wattGLB))]))}
\FunctionTok{abline}\NormalTok{(}\AttributeTok{a =} \DecValTok{0}\NormalTok{, }\AttributeTok{b =} \DecValTok{1}\NormalTok{, }\AttributeTok{col =} \StringTok{"green"}\NormalTok{)}
\end{Highlighting}
\end{Shaded}

\begin{figure}[H]

{\centering \includegraphics[width=1\linewidth]{GHI_enh_08_validation_files/figure-latex/unnamed-chunk-7-1} 

}

\caption{ - empty caption - }\label{fig:unnamed-chunk-7}
\end{figure}

\begin{Shaded}
\begin{Highlighting}[]
\FunctionTok{print}\NormalTok{(}\FunctionTok{cor.test}\NormalTok{(KEEP}\SpecialCharTok{$}\NormalTok{wattGLB, KEEP}\SpecialCharTok{$}\NormalTok{Enhanc\_C\_4\_ref, }\AttributeTok{method =} \FunctionTok{c}\NormalTok{(}\StringTok{"pearson"}\NormalTok{)))}
\end{Highlighting}
\end{Shaded}

\begin{verbatim}

    Pearson's product-moment correlation

data:  KEEP$wattGLB and KEEP$Enhanc_C_4_ref
t = 7938, df = 345175, p-value < 2.2e-16
alternative hypothesis: true correlation is not equal to 0
95 percent confidence interval:
 0.9972540 0.9972904
sample estimates:
      cor 
0.9972722 
\end{verbatim}

\begin{Shaded}
\begin{Highlighting}[]
\FunctionTok{print}\NormalTok{(}\FunctionTok{lm}\NormalTok{(KEEP[, wattGLB, Enhanc\_C\_4\_ref]))}
\end{Highlighting}
\end{Shaded}

\begin{verbatim}

Call:
lm(formula = KEEP[, wattGLB, Enhanc_C_4_ref])

Coefficients:
(Intercept)      wattGLB  
     34.056        1.078  
\end{verbatim}

\textbf{END}

\begin{verbatim}
2024-09-06 09:43:14.1 athan@sagan GHI_enh_08_validation.R 2.207945 mins
\end{verbatim}

\end{document}
