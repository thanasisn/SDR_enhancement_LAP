\documentclass[preprint, 5p,
authoryear]{elsarticle} %review=doublespace preprint=single 5p=2 column
%%% Begin My package additions %%%%%%%%%%%%%%%%%%%

\usepackage[hyphens]{url}

  \journal{Atmospheric Research} % Sets Journal name

\usepackage{graphicx}
%%%%%%%%%%%%%%%% end my additions to header

\usepackage[T1]{fontenc}
\usepackage{lmodern}
\usepackage{amssymb,amsmath}
% TODO: Currently lineno needs to be loaded after amsmath because of conflict
% https://github.com/latex-lineno/lineno/issues/5
\usepackage{lineno} % add
\usepackage{ifxetex,ifluatex}
\usepackage{fixltx2e} % provides \textsubscript
% use upquote if available, for straight quotes in verbatim environments
\IfFileExists{upquote.sty}{\usepackage{upquote}}{}
\ifnum 0\ifxetex 1\fi\ifluatex 1\fi=0 % if pdftex
  \usepackage[utf8]{inputenc}
\else % if luatex or xelatex
  \usepackage{fontspec}
  \ifxetex
    \usepackage{xltxtra,xunicode}
  \fi
  \defaultfontfeatures{Mapping=tex-text,Scale=MatchLowercase}
  \newcommand{\euro}{€}
\fi
% use microtype if available
\IfFileExists{microtype.sty}{\usepackage{microtype}}{}
\usepackage[]{natbib}
\bibliographystyle{elsarticle-num-names}

\ifxetex
  \usepackage[setpagesize=false, % page size defined by xetex
              unicode=false, % unicode breaks when used with xetex
              xetex]{hyperref}
\else
  \usepackage[unicode=true]{hyperref}
\fi
\hypersetup{breaklinks=true,
            bookmarks=true,
            pdfauthor={},
            pdftitle={Long-term cloud enhancement events of global solar irradiance over Thessaloniki, Greece},
            colorlinks=false,
            urlcolor=blue,
            linkcolor=magenta,
            pdfborder={0 0 0}}

\setcounter{secnumdepth}{5}
% Pandoc toggle for numbering sections (defaults to be off)


% tightlist command for lists without linebreak
\providecommand{\tightlist}{%
  \setlength{\itemsep}{0pt}\setlength{\parskip}{0pt}}




\usepackage{caption}
\usepackage{placeins}
\captionsetup{font=small}
\usepackage{subcaption}
\usepackage{booktabs}
\usepackage{longtable}
\usepackage{array}
\usepackage{multirow}
\usepackage{wrapfig}
\usepackage{float}
\usepackage{colortbl}
\usepackage{pdflscape}
\usepackage{tabu}
\usepackage{threeparttable}
\usepackage{threeparttablex}
\usepackage[normalem]{ulem}
\usepackage{makecell}
\usepackage{xcolor}



\begin{document}


\begin{frontmatter}

  \title{Long-term cloud enhancement events of global solar irradiance
over Thessaloniki, Greece}
    \author[LAP]{Athanasios N. Natsis%
  \corref{cor1}%
  }
   \ead{natsisphysicist@gmail.com} 
    \author[LAP]{Alkiviadis Bais%
  %
  }
   \ead{abais@auth.gr} 
    \author[LAP]{Charikleia Meleti%
  %
  }
   \ead{meleti@auth.gr} 
      \affiliation[LAP]{
    organization={Laboratory of Atmospheric Physics, Physics Department,
Aristotle University of
Thessaloniki},city={Thessaloniki},postcode={54124},country={Greece},}
    \cortext[cor1]{Corresponding author}
  
  \begin{abstract}
  In this study we investigate the characteristics of global horizontal
  irradiance enhancement events induced by clouds over Thessaloniki for
  the period 1993 -- 2023 using data recorded every one minute. We
  identified the cloud enhancement (CE) events by creating an
  appropriate clear-sky irradiance reference with the use of a radiative
  transfer model and aerosol optical depth data from a collocated Cimel
  sun photometer and a Brewer spectrophotometer. We found a trend in CE
  events of \(45.6\pm 21.9\,\text{cases}/\text{year}\), and a trend in
  the CE events irradiation of \(116.9\pm 67.8\,\text{kJ}/\text{year}\).
  The peak of the CE events was observed during May and June. The
  analysis of the total duration of CE events showed that durations
  longer than 5 minutes are very rare, with exceptions lasting over an
  hour and up to 140 minutes. Finally, we have detected enhancements
  above the total solar irradiance at the top of the atmosphere of up to
  \(400\,\text{W}/\text{m}^{2}\), with the \(75\,\%\) of the cases below
  \(200\,\text{W}/\text{m}^{2}\). The most active period of these
  extreme events is spring -- early summer with a secondary peak in
  autumn.
  \end{abstract}
    \begin{keyword}
    cloud enhancement \sep total solar radiation \sep global horizontal
irradiance \sep 
    over irradiance
  \end{keyword}
  
 \end{frontmatter}

\hypertarget{introduction}{%
\section{Introduction}\label{introduction}}

The shortwave solar radiation, reaching Earth's surface, is the main
energy source for the atmosphere and the biosphere, and drives and
governs the climate \citep{Gray2010}. It has direct practical
application in industries related to energy and agricultural production.
The variability of its intensity can impose difficulties in predicting
the yield and in designing the specifications of the appropriate
equipment. Significant portion of research has been focused on
predicting the renewable energy production in a fine timescale and in
near real-time \citep[for a review see][]{Inman2013, Graabak2016}.

An important aspect of the variability of solar radiation is its
interaction with the clouds. In general, clouds can attenuate a fraction
of solar irradiance, but under certain conditions, can lead to
enhancement of the global horizontal irradiance (GHI) reaching the
ground. This cloud enhancement (CE) effect can locally increase the
observed GHI to levels even higher than the expected clear-sky
irradiance {[}\citet{Cordero2023}; \citet{Vamvakas2020};
\citet{CastillejoCuberos2020}; and references therein{]}.

Some of the proposed underling mechanisms of those enhancements, have
been summarized by \citet{Gueymard2017}; the most important being the
scattering of radiation on the edges of cumulus clouds. It has been also
suggested that enhancement of GHI can be produced by thin cirrus clouds
though refraction and scattering \citep{Thuillier2013}. Further
investigation with radiative transfer modeling and observations pointed
as the prevailing mechanism, the strong forward Mie scattering through
clouds of low optical depth
\citep{Pecenak2016, Thuillier2013, Yordanov2013, Yordanov2015}. Overall,
enhancement events depend on different interactive factors, which
include cloud thickness, structure and type, and the relative position
of the sun and the clouds \citep{Gueymard2017, Veerman2022}.

On multiple sites, cloud enhancements have been reported that exceed
momentarily the solar constant, resulting in clearness indices above
unit. A summary of extreme cloud enhancement (ECE) cases has been
compiled by \citet{Martins2022}. Cloud enhancements can have also some
practical implications. The intensity and duration of enhancements can
affect the efficiency and stability of photovoltaic power production
\citep{Lappalainen2020, Jaervelae2020}, while ECEs have the potential to
compromise the integrity of photovoltaic plants infrastructure
\citep{DoNascimento2019}. It has also been demonstrated that these
events can interfere in the comparison of ground-based and satellite
observations of radiation \citep{Damiani2018}. Global warming has likely
affected cloud coverage in the last few decades. \citet{Liu2023}
reported increases in cloud cover over the tropical and subtropical
oceans and decreases over most continents, while \citet{Dong2023}
reported decreases over North America and Europe. To our knowledge,
there is no evidence on whether this trend has affected also the number
and strength of CE events.

Methods of identification of CE events usually include the use of
simulated clear sky irradiance as baseline, combined with an appropriate
threshold or some other statistical characteristics, and in some cases,
with visual inspection of sky camera images \citep[ and references
therein]{Vamvakas2020, Mol2023}.

In this study, we evaluate the effects of CE on GHI by investigating
their frequency of occurrence, intensity, and duration in a thirty-year
period of GHI observations at Thessaloniki, Greece. We used modeled
clear sky irradiance as a baseline to identify cloud enhancements, and
we determined long-term trends of the above-mentioned metrics, their
climatology and some general characteristics. To our knowledge there are
no other studies that provide trends from such long dataset.

\hypertarget{data-and-methodology}{%
\section{Data and methodology}\label{data-and-methodology}}

\hypertarget{instrumentation-and-data}{%
\subsection{Instrumentation and data}\label{instrumentation-and-data}}

The data used in this study were recorded at the monitoring site of the
Laboratory of Atmospheric Physics, Aristotle University of Thessaloniki,
in Thessaloniki, Greece (\(40^\circ\,38'\,\)N, \(22^\circ\,57'\,\)E,
\(80\,\)m~a.s.l.). The GHI data were measured with a Kipp~\& Zonen CM-21
pyranometer and cover the period 01~January 1994 to 31~December 2023.
During the study period, the pyranometer was independently calibrated
three times at the Meteorologisches Observatorium Lindenberg, DWD,
verifying that the stability of the instrument's sensitivity was better
than \(0.7\,\%\) relative to the initial calibration by the
manufacturer. For the acquisition of radiometric data, the signal of the
pyranometer was sampled at a rate of \(1\,\text{Hz}\) with the mean and
standard deviation of these samples calculated and recorded every
minute. The measurements were corrected for the zero offset (``dark
signal'' in volts), which was calculated by averaging all measurements
recorded for a period of \(3\,\text{h}\), before (morning) or after
(evening) the Sun reaches an elevation angle of \(-10^\circ\). The
signal was converted to irradiance using the ramped value of the
instrument's sensitivity between subsequent calibrations.

To further improve the quality of the irradiance data, a manual
screening was performed, in order to remove inconsistent and erroneous
recordings that can occur stochastically or systematically during the
long operation of the instrument. The manual screening was aided by a
radiation data quality assurance procedure, adjusted for the site, which
was based on the methods of Long and Shi~\citep{Long2006, Long2008a}.
Thus, problematic recordings have been excluded from further processing.
Furthermore, due to the significant measurement uncertainty in GHI when
the Sun is near the horizon, and due to some systematic obstructions by
nearby buildings, we have excluded all measurements with solar zenith
angle (SZA) greater than \(78^\circ\). Finally, images from a sky camera
have been used in the manual inspection of the CE identification. The
sky camera operates since 2012 and stores images in 5 min time steps.

\hypertarget{cloud-enhancement-detection}{%
\subsection{Cloud enhancement
detection}\label{cloud-enhancement-detection}}

In this study, we define an event as CE when the measured GHI at ground
level, exceeds the expected value under clear-sky conditions. Similarly,
we define as extreme cloud enhancement events (ECE), the cases when GHI
at ground level exceeds the Total Solar Irradiance (TSI). Although the
duration of these bursts can vary from seconds to several minutes, here
we are constrained by the temporal resolution of our data to identify
events with duration of at minimum one-minute.

For the detection of CE cases we established a baseline of irradiance
above which we characterized each data point as an enhancement event and
calculated the over irradiance (OI). The OI is defined as the irradiance
difference of the measured one-minute GHI from the
\(\text{GHI}_\text{ref}\) corresponding to cloud-free atmosphere. First,
we used a simple approach for the determination of
\(\text{GHI}_\text{ref}\): The Haurwitz's model \citep{Haurwitz1945},
which is a simple clear sky model and was already adjusted and applied
to our data \citep{Natsis2023}. We created a threshold by using an
appropriate relative and/or an additional constant offset. The initial
results showed that we can detect a big portion of the actual CE events.
However, by inspecting the daily plots of irradiance it became evident
that changes in atmospheric conditions introduced numerous false
positive or false negative results. The main reason for these
discrepancies is the variability of the effects of aerosols and water
vapor which were not taken into account in the two simple methods. To
produce a more representative reference we included the effects of these
factors using a radiative transfer model (RTM). The applied methodology
is discussed in section\nobreakspace{}\ref{rtmcs}.

\hypertarget{rtmcs}{%
\subsection{Modeled clear Sky Irradiance}\label{rtmcs}}

\hypertarget{climatology-of-clear-sky-irradiance}{%
\subsubsection{Climatology of clear sky
irradiance}\label{climatology-of-clear-sky-irradiance}}

We approximated the expected clear sky \(\text{GHI}_\text{ref}\) with
the radiative transfer model uvspec, part of libRadtran
\citep{Emde2016}, similarly to the approach used by
\citet{Vamvakas2020}. In uvspec we used the solar spectrum of
\citet{Kurucz1994} in the range \(280\) to \(2500\,\text{nm}\), the
radiative transfer solver ``disort'' in ``pseudospherical'' geometry and
the ``LOWTRAN'' gas parameterization. The model was run for a range of
variables in order to create a look-up table (LUT) for the estimation of
the cloud-free reference irradiance for each individual observation of
our dataset. In this context, the model was run for SZAs in the range
\(10\) -- \(90^\circ\) with a step of \(0.2^\circ\) and for the
atmospheric profiles of the Air Force Geophysics Laboratory
\citep{Anderson1986} midlatitude summer and midlatitude winter,
representative of the warm and cold seasons.

Main factors responsible for the attenuation of the broadband downward
solar radiation under cloud free atmospheres are aerosols and water
vapor. At Thessaloniki, such measurements are available since 2003 from
the Cimel sun photometer that belongs to the Aerosol Robotic Network
(AERONET) \citep{Giles2019, Buis1998}. From the observations in the
period 2003 -- 2023 we calculated the monthly climatological means and
standard deviation (\(\sigma\)) for the aerosol optical depth (AOD) at
\(500\,\text{nm}\) and the equivalent height of the water column (WC).
The monthly climatological values of AOD and WC, as well as combinations
with additional offsets of \(\pm1\sigma\) and \(\pm2\sigma\), were used
as inputs to the RTM in the construction of the LUT.

For each measurement of the dataset, we calculated from the LUT a
\(\text{GHI}_\text{ref}\) value for the respective season and SZA (by
linear interpolation), and for the climatological values of AOD and WC
of the respective month. The same procedure was followed for the
estimation of the \(\text{GHI}_\text{ref}\) for all combinations of the
AOD and WC with the above-mentioned standard deviation offsets. Finally,
each \(\text{GHI}_\text{ref}\) value was adjusted to the actual
Sun-Earth distance derived by the Astropy software library
\citep{AstropyCollaboration2022}.

\hypertarget{long-term-change-of-clear-sky-irradiance}{%
\subsubsection{Long-term change of clear sky
irradiance}\label{long-term-change-of-clear-sky-irradiance}}

The clear-sky reference values discussed above are based on the
climatological AOD and WC; hence they cannot describe accurately the
long-term variation of \(\text{GHI}_\text{ref}\) due to long-term
changes in the two atmospheric constituents, mainly AOD. As reported by
\citet{Natsis2023}, there is a long-term brightening effect in the GHI
data of Thessaloniki for the period 1993 -- 2023, which for clear-sky
data was attributed to long-term changes in aerosol effects. Therefore,
an adjustment of the \(\text{GHI}_\text{ref}\) during the period of
study was made using simulations with the RTM based on the long-term
variations of the AOD. As AERONET data start only in 2003, we used for
the period 1993 -- 2005 estimates of changes in AOD at
\(340\,\text{nm}\) derived from a collocated Brewer spectrophotometer
\citep{Kazadzis2007} to calculate the trend in \(\text{GHI}_\text{ref}\)
due to aerosols during this period.

According to \citet{Kazadzis2007}, in the period 1997 -- 2005 the mean
AOD at \(340\,\text{nm}\) is \(0.403\) with a trend of
\(-3.8\pm0.93\,\%\) per year, corresponding to a change of \(0.0153\)
per year. Using an Ångström coefficient \(\alpha = 1.6\), this
translates to a change in the Ångström coefficient \(\beta\) of
\(0.00272\) per year (or \(\beta=0.084\) in 1997 and \(\beta=0.059\) in
2005). Simulations with uvspec for the above Ångström coefficients,
assuming constant WC of \(15.6\,\text{mm}\) taken from the Cimel, and
for a SZA of \(55^\circ\) reveal a trend of \(+0.21\,\%\) per year in
\(\text{GHI}_\text{ref}\). The SZA of \(55^\circ\) was chosen as
representative of all days in the year in order to get a rough estimate
of the annually averaged change in clear sky irradiance. For the period
2005 -- 2023 we used the mean monthly values of AOD and WC from AERONET
in a similar simulation scheme to calculate the monthly mean clear-sky
irradiance, and finally the trend of \(+0.14\,\%\) per year. We applied
these two long-term changes (see
Figure\nobreakspace{}\ref{fig:CS-change}) to the climatological
\(\text{GHI}_\text{ref}\), in order to create a more realistic
representation of the clear-sky irradiance for the whole period of
study.

\begin{figure}

{\centering \includegraphics[width=1\linewidth]{../images/P-CS-change-1} 

}

\caption{Simulated long-term change in clear sky irradiance relative to the climatological values due to changes in AOD in Thessaloniki for the period 1993 -- 2023.}\label{fig:CS-change}
\end{figure}

\hypertarget{criteria-for-the-identification-of-ce-events}{%
\subsection{Criteria for the identification of CE
events}\label{criteria-for-the-identification-of-ce-events}}

In this study our main focus was to quantify the OI related to CEs. A
key issue for achieving this goal is to define a threshold for the CE
identification, representative of the clear-sky irradiance at the time
of each GHI measurement. This depends on the selection of the
appropriate atmospheric parameterization for the RTM simulations. The
implementation of the long-term change in AOD, discussed in
section\nobreakspace{}\ref{rtmcs}, allows capturing a large part of the
natural variability of clear-sky GHI. However, the short-term
variability of AOD cannot be taken adequately into account when using
monthly values in the model simulations. We tried different approaches
in order to strengthen the robustness of the methodology and to
compensate for the limited accuracy of the RTM input data and the
unpredictable natural variability of the atmosphere.

First, we evaluated the performance of the modelled
\(\text{GHI}_\text{ref}\) in relation to the measured GHI for different
levels of atmospheric clearness, by using in the RTM the monthly
climatological AOD and WC, less their respective standard deviations.
These values represent typical atmospheres in Thessaloniki with lower
than average load of aerosols and humidity, which are the main factors
that attenuate the GHI, excluding clouds. With this approach the
simulated \(\text{GHI}_\text{ref}\) should be generally greater than the
measured GHI when aerosols are more abundant. To compensate for this, we
defined the following threshold \(E\) to compare the measured
\(\text{GHI}\): \begin{equation}
\text{CE} : E > 15 + 1.04 \cdot \text{GHI}_\text{ref} \,\,[\text{W}/\text{m}^2] \label{eq:CE4}
\end{equation}

This is the criterion of our CE identification. The constant terms were
determined through the implementation of an empirical method with manual
inspection of the CE identification results on selected days of the
whole dataset. We tested seven sets of days with different
characteristics relevant to the efficiency of the identification
threshold. These sets were random groups of about 20 -- 30 days with the
following characteristics: (a) the strongest OI CE events, (b) the
largest daily total OI, (c) absence of clouds (by implementing a clear
sky identification algorithm as discussed in \citet{Natsis2023}), (d)
absence of clouds and absence of CE events, (e) with at least \(60\,\%\)
of the day length without clouds and presence of CE events, (h) randomly
selected days, and (i) manually selected days. For the latter case and
where needed, we also used images from the sky-camera to further aid the
decision of the manual inspection.

The definition of the CE events with this method has a degree of
subjectivity, since the actual clear sky irradiance is not known and can
only be approximated. However, this method was proven capable in
detecting all major CE events. Where some CE events with very low OI may
be not detected, these are few with small over-irradiance, and it is
unlikely that will affect significantly our results.

A sub-category of the CE events that is often discussed in the
literature \citep{Cordero2023, Martins2022, Yordanov2015}, are the
extreme cloud enhancement (ECE) events. These are cases of CE where the
measured intensity of the irradiance exceeds the TSI at the top of the
atmosphere. In this case the threshold \(E\) is given by:
\begin{equation}
\text{ECE}: E > \cos(\theta) \cdot E_{\odot} \frac{r^2_\text{m}} {r^2}
\label{eq:ECE}
\end{equation} where: \(\theta\) the solar zenith angle, \(E_{\odot}\)
the solar constant adjusted for the actual Sun -- Earth distance \(r\)
and \(r_\text{m}\) is the mean Sun -- Earth distance of
\(1.496\times10^8\,\text{km}\).

An example of CE identification for a selected day is given in the
Figure\nobreakspace{}\ref{fig:example-day}, where the daily course of
the clear sky reference irradiance and the CE and ECE identification
thresholds are shown along with the actual GHI measurements. In
addition, we provide an example scatter plot between the measured and
the modeled clear-sky irradiance for one year, where the CE and ECE
events are clearly grouped above the threshold of irradiance
(Figure\nobreakspace{}\ref{fig:example-year}).

\begin{figure}[H]

{\centering \includegraphics[width=1\linewidth]{../images/example-days-18} 

}

\caption{Example of CE identification in Thessaloniki for 2019-07-11. The green line with blue symbols depicts the measured GHI in one minute steps. Red line shows the modelled threshold $E$ for the detection of CE events, which are denoted with red circles.Black curve represents the TOA TSI on horizontal plane.}\label{fig:example-day}
\end{figure}

\begin{figure}[H]

{\centering \includegraphics[width=1\linewidth]{../images/P-example-years-12} 

}

\caption{Example scatter plot of the measured GHI and the reference clear sky irradiance in Thessaloniki for the year 2005. The over-irradiance for CE and ECE events is color coded, while the remaining data points are shown in black.}\label{fig:example-year}
\end{figure}

\hypertarget{results}{%
\section{Results}\label{results}}

Following the application of the above discussed methodology to the
entire dataset (\(6\) million of one-minute GHI measurements),
\(1.78\,\%\) were identified as CE events and \(0.037\,\%\) as ECE
events. The highest recorded GHI due to CE was
\(1416.6\,\text{W}/\text{m}^2\) on 24~May 2007 at a SZA of
\(19.9^\circ\) corresponding to OI of \(345.9\,\text{W}/\text{m}^2\) or
\(32.3\,\%\) above the threshold. The strongest CE event of \(49.7\,\%\)
above the clear sky threshold was observed on 28~October 2016 at a SZA
of \(59.2^\circ\) with a GHI value of \(861.8\,\text{W}/\text{m}^2\) and
a OI of \(286.1\,\text{W}/\text{m}^2\). Both cases are ECE events with
\(168.7\,\text{W}/\text{m}^2\) and \(155.4\,\text{W}/\text{m}^2\) above
the TSI at TSI for the same SZA, respectively. In the following sections
we are discussing the long-term trends and variability of the CE events
as well as of the corresponding OI and excess irradiation.

\hypertarget{long-term-trends}{%
\subsection{Long-term trends}\label{long-term-trends}}

One aspect of this study is to investigate the time evolution of the CE
events by analyzing the GHI measurements at Thessaloniki. Cloud
enhancements can be influenced by different factors, such as the
geometry, size and optical thickness of clouds, their height in the
atmosphere and local weather regimes
\citep{Mol2023, Veerman2022, Gristey2022, Tzoumanikas2016}. Some of
these factors are related to changes in climate; hence it would be
reasonable to expect contributing to the frequency of occurrence of CE
events over Thessaloniki, as well as to the average OI and excess
irradiation. The long-term trends were calculated using a first-order
autoregressive model with the `maximum likelihood' fitting method
\citep{Gardner1980, Jones1980}, by implementing the function `arima'
from the library `stats' of the R programming language \citep{RCT2023}.
All trends are reported together with their \(2\sigma\) error.

Figure\nobreakspace{}\ref{fig:P-energy} shows the time series of the
yearly number of CE cases (each with duration of one minute), the yearly
mean OI and the yearly excess irradiation for the period 1993 -- 2023,
together with corresponding linear trends. All three quantities show
increasing trends, most pronounced for the frequency of occurrence
(\(+45.6\pm 21.9\,\text{cases}/\text{year}\)) and the excess irradiation
(\(+116.9 \pm 67.8\,\text{kJ}/\text{year}\)), which are also
statistically significant. In contrast the trend of the yearly mean OI
is negligible (\(+0\pm 0.1\,\text{W}/\text{m}^2/\text{year}\)) and of no
statistical significance. The average OI for the entire period is
\(+40.1\pm 2.5\,\text{W}/\text{m}^2\). The interannual variability of
the data about the trend lines is quite large. Furthermore, the spread
tends to increase with time (at least for the quantities of panels b and
c), suggesting a significant variability in cloud patterns over the
area, possibly associated to changes in climate.

We have to note that the excess irradiation related to the CE events
cannot be directly linked to the total energy balance of the atmosphere.
The net solar radiation of the region is not increased, but is rather
redistributed through the CE events. This is also depicted by the ECE
irradiance values, which exceed the equivalent clear sky irradiance by a
significant amount.

\begin{figure}% [h!]
        {\centering 
            \subfloat[\label{fig:P-energy-mean}]
                {\includegraphics[width=\linewidth]{../images/P-energy-3} }\\
            \subfloat[\label{fig:P-energy-N}]
                {\includegraphics[width=\linewidth]{../images/P-energy-2} }\\
            \subfloat[\label{fig:P-energy-sum}]
                {\includegraphics[width=\linewidth]{../images/P-energy-1} }
        }
    \caption{Time series for the period 1993 -- 2023 of (a) the yearly CE number of occurrences, (b) the yearly mean OI and (c) the yearly excess irradiation. The black lines represent the linear trends on the yearly data.}\label{fig:P-energy}
\end{figure}

\hypertarget{climatology-of-cloud-enhancement-events}{%
\subsection{Climatology of cloud enhancement
events}\label{climatology-of-cloud-enhancement-events}}

Next we investigated the distribution of the CE events within the year.
Figure\nobreakspace{}\ref{fig:relative-month-occurrences} shows the
monthly box and whisker plot of the CE number of occurrences normalized
with the highest median value, which occurs in June, depicting a clear
seasonal cycle. Although CE events are present throughout the year, the
most active months are May and June. During the winter (December --
February), the number of CE cases is about \(25\,\%\) of the maximum,
while in the intermediate months, the number of occurrences gradually
ramps between the maximum and minimum. This seasonality is a combined
effect of different factors, among them the types of clouds, their
frequency of occurrence, the seasonally varying relative position of the
sun, as well as the local landscape characteristics that may influence
the formation of the clouds. Unfortunately, lack of detailed data on
cloud formation, type and location is not allowing further analysis. The
interannual variability of the monthly CE events is quite high as
manifested by the large monthly extremes, especially in the summer.

\begin{figure}

{\centering \includegraphics[width=1\linewidth]{../images/clim-CE-month-norm-MAX-median-N-MW-1} 

}

\caption{Seasonal variability of the number of CE events in Thessaloniki for the period 1993 -- 2023 normalized to the maximum occurring in June, in the form of a box and whisker plot. The monthly values have also been normalized to the relative abundance of valid GHI observations. The box contains the data between the lower $25\,\%$ and the upper $75\,\%$ percentiles, the thick horizontal line and the diamond symbol represent the median and the mean values, respectively. The vertical lines (whiskers) extend between the maximum and minimum monthly values.}\label{fig:relative-month-occurrences}
\end{figure}

The distribution of the number of CE events as a function of OI is shown
in Figure\nobreakspace{}\ref{fig:ovir-distribution}. There is an inverse
relationship between the frequency of CE events and OI with an
exponential-like decline. This is expected, as the stronger the CE
events are, the rarer the conditions favoring the occurrence of CE
events. For the majority (over \(62\,\%\)) of the CE events the OI is
below the long-term average of \(40.1\,\text{W}/\text{m}^2\), while
about \(8.1\,\%\) of the events correspond to OI larger than
\(100\,\text{W}/\text{m}^2\) and up to the highest value of
\(412.4\,\text{W}/\text{m}^2\). This distribution is indicative of the
magnitude and the probability of the expected CE events over
Thessaloniki. Similar distribution of CE events, albeit with larger OI
values, has been reported by \citet{Vamvakas2020}, for the city of
Patras. This site is located \(2.5^\circ\) south of Thessaloniki and is
exposed to air masses coming mainly from the eastern Mediterranean
resulting in different cloud patterns, that may affect the
characteristics and magnitude of the CE events.

\begin{figure}

{\centering \includegraphics[width=1\linewidth]{../images/P-relative-distribution-diff-2} 

}

\caption{Relative frequency distribution of CE events in Thessaloniki for the period 1993 -- 2023 as a function of OI. The histogram was split in two plots with different y-axis scales for better readability.}\label{fig:ovir-distribution}
\end{figure}

\hypertarget{duration-of-cloud-enhancement-events}{%
\subsection{Duration of cloud enhancement
events}\label{duration-of-cloud-enhancement-events}}

The duration of the CE events is variable and can last for several
minutes or even more than an hour. In order to study the characteristics
of these consecutive events, we grouped them into bins of increasing
duration in steps of one minute. We have identified 28062 groups of CE
in the whole period of study, where the group of the longest duration of
140 minutes occurred on 07~July 2013.
Figure\nobreakspace{}\ref{fig:ceg-duration-distribution} shows the
frequency distribution of the CE events according to their duration. We
conclude that although some groups of events last for more than an hour,
about \(80\,\%\) have duration of less than 5 minutes.

\begin{figure}

{\centering \includegraphics[width=1\linewidth]{../images/groups-7} 

}

\caption{Relative frequency distribution of CE groups of consequent CE events according to their duration for Thessaloniki in the period 1993 -- 2023. The histogram was split in two plots with different y-axis scales for better readability.}\label{fig:ceg-duration-distribution}
\end{figure}

The relation between the duration and the mean OI of the groups has also
been studied (Figure\nobreakspace{}\ref{fig:group-2d}). Evidently,
events of high excess irradiation have small duration and vice versa.
The vast majority of grouped events are associated with small excess
irradiation (e.g., \(<5\,\text{kJ}/\text{m}^2\)) and small duration
(e.g., \(<5\,\text{min}\)) while groups with strong excess irradiation
and long duration are very rare. Similar results of this relation have
been reported by \citet{Zhang2018} in a study using a far higher
sampling rate (\(100\,\text{Hz}\)) than ours.

\begin{figure}

{\centering \includegraphics[width=1\linewidth]{../images/P-groups-bin2d-1} 

}

\caption{Relation of excess irradiation of CE groups with their duration for Thessaloniki in the period 1993 -- 2023. The logarithmic color scale denotes the frequency of the respective groups of events.}\label{fig:group-2d}
\end{figure}

\hypertarget{extreme-cloud-enhancement-events}{%
\subsection{Extreme cloud enhancement
events}\label{extreme-cloud-enhancement-events}}

An aspect of the CE events that is commonly reported and has some
significance on the solar energy production infrastructure are the
extreme CE events (ECE), where solar irradiance exceeds the expected
irradiance on top of the atmosphere at the same SZA
(Equation\nobreakspace{}\ref{eq:ECE}). Analogous to
Figure\nobreakspace{}\ref{fig:relative-month-occurrences} we have
computed the distribution of the number of occurrences of ECE events by
month in Figure\nobreakspace{}\ref{fig:relative-month-occurancies-ECE}.
The most active period for ECE events is in spring and early summer
(March -- June), followed by a period in late autumn (September and
October). This is probably related to the weather characteristics in
these periods, with frequent alternations between clear sky periods and
clouds.

\begin{figure}

{\centering \includegraphics[width=1\linewidth]{../images/clim-ECE-month-norm-MAX-median-N-2} 

}

\caption{Seasonal variability of the number of ECE events in Thessaloniki for the period 1993 -- 2023 normalized to the maximum occurring in March, in the form of a box and whisker plot. The box contains the data between the lower $25\,\%$ and the upper $75\,\%$ percentiles. The thick horizontal line and the diamond symbol represent the median and the mean values, respectively. The vertical lines (whiskers) extend between the maximum and minimum monthly values.}\label{fig:relative-month-occurancies-ECE}
\end{figure}

The distribution of the ECE events
(Figure\nobreakspace{}\ref{fig:P-extreme-distribution}) shows that in
rare cases the OI can exceed the TSI even by more than
\(400\,\text{W}/\text{m}^2\), while in \(75\,\%\) of the cases the OI is
below \(200\,\text{W}/\text{m}^2\). The OI for the most frequent ECEs
ranges between \(140\) and \(180\,\text{W}/\text{m}^2\). These findings
are in accordance with the results of \citet{Vamvakas2020}, the only
difference being that the OI values reported for are higher than those
for Thessaloniki.

\begin{figure}

{\centering \includegraphics[width=1\linewidth]{../images/P-extreme-distribution-1} 

}

\caption{Distribution of ECE events in Thessaloniki for the period 1993 -- 2023.}\label{fig:P-extreme-distribution}
\end{figure}

\hypertarget{conclusions}{%
\section{Conclusions}\label{conclusions}}

By creating a clear sky approximation representing the long- and
short-term variability of the expected clear sky GHI, we were able to
identify cases of CE events in Thessaloniki for the period 1993 -- 2023.
After analyzing the CE cases, we found an increase of
\(+45.6\pm 21.9\,\text{cases}/\text{year}\), with the mean annual
irradiation of the CE events increasing with a rate of
\(+116.9\pm 67.8\,\text{kJ}/\text{year}\). The most active months of CE
events are May and June. We found that continuous CE events can last up
to \(140\) minutes, while the duration of \(80\,\%\) of them is bellow
\(5\) minutes.

We have observed ECE cases exceeding the TSI by
\(400\,\text{W}/\text{m}^{2}\) with \(75\,\%\) of the cases under
\(200\,\text{W}/\text{m}^{2}\). The climatological characteristics of
the ECE events showed that the most active months are spread in half of
the year and particularly in the periods March -- June and September --
October. The magnitude of the ECE events identified in Thessaloniki
events does not exceed the values reported for sites with more
favourable conditions for the phenomenon \citep[e.g.,][]{Cordero2023}.
Some of the characteristics of CE and ECE events we analyzed have strong
similarities with the results of \citet{Vamvakas2020} for the city of
Patras, south of Thessaloniki, albeit with differences in the magnitude
of OI.

An interpretation of the CE trends shows that the interaction of GHI
with the clouds, through this 30-year period, is a dynamic phenomenon
that needs further investigation. For such studies it would be essential
to have more detailed information on cloud characteristics, especially
in order to investigate possible associations of the observed trends
with changes in climate.

\bibliography{bibliography.bib}


\end{document}
