\documentclass[preprint, 3p,
authoryear]{elsarticle} %review=doublespace preprint=single 5p=2 column
%%% Begin My package additions %%%%%%%%%%%%%%%%%%%

\usepackage[hyphens]{url}

  \journal{An awesome journal} % Sets Journal name

\usepackage{graphicx}
%%%%%%%%%%%%%%%% end my additions to header

\usepackage[T1]{fontenc}
\usepackage{lmodern}
\usepackage{amssymb,amsmath}
% TODO: Currently lineno needs to be loaded after amsmath because of conflict
% https://github.com/latex-lineno/lineno/issues/5
\usepackage{lineno} % add
\usepackage{ifxetex,ifluatex}
\usepackage{fixltx2e} % provides \textsubscript
% use upquote if available, for straight quotes in verbatim environments
\IfFileExists{upquote.sty}{\usepackage{upquote}}{}
\ifnum 0\ifxetex 1\fi\ifluatex 1\fi=0 % if pdftex
  \usepackage[utf8]{inputenc}
\else % if luatex or xelatex
  \usepackage{fontspec}
  \ifxetex
    \usepackage{xltxtra,xunicode}
  \fi
  \defaultfontfeatures{Mapping=tex-text,Scale=MatchLowercase}
  \newcommand{\euro}{€}
\fi
% use microtype if available
\IfFileExists{microtype.sty}{\usepackage{microtype}}{}
\usepackage[]{natbib}
\bibliographystyle{elsarticle-harv}

\ifxetex
  \usepackage[setpagesize=false, % page size defined by xetex
              unicode=false, % unicode breaks when used with xetex
              xetex]{hyperref}
\else
  \usepackage[unicode=true]{hyperref}
\fi
\hypersetup{breaklinks=true,
            bookmarks=true,
            pdfauthor={},
            pdftitle={Downward shortwave solar irradiance enhancement by clouds.},
            colorlinks=false,
            urlcolor=blue,
            linkcolor=magenta,
            pdfborder={0 0 0}}

\setcounter{secnumdepth}{5}
% Pandoc toggle for numbering sections (defaults to be off)


% tightlist command for lists without linebreak
\providecommand{\tightlist}{%
  \setlength{\itemsep}{0pt}\setlength{\parskip}{0pt}}




\usepackage{booktabs}
\usepackage{longtable}
\usepackage{array}
\usepackage{multirow}
\usepackage{wrapfig}
\usepackage{float}
\usepackage{colortbl}
\usepackage{pdflscape}
\usepackage{tabu}
\usepackage{threeparttable}
\usepackage{threeparttablex}
\usepackage[normalem]{ulem}
\usepackage{makecell}
\usepackage{xcolor}



\begin{document}


\begin{frontmatter}

  \title{Downward shortwave solar irradiance enhancement by clouds.}
    \author[LAP]{Athanasios N. Natsis%
  \corref{cor1}%
  \fnref{1}}
   \ead{natsisphysicist@gmail.com} 
    \author[LAP]{Alkiviadis Bais%
  %
  }
   \ead{abais@auth.gr} 
    \author[LAP]{Charikleia Meleti%
  %
  \fnref{2}}
   \ead{meleti@auth.gr} 
      \affiliation[LAP]{
    organization={Laboratory of Atmospheric Physics,
A.U.Th.},addressline={1 main
street},city={Thessaloniki},postcode={123456},country={Greece},}
    \affiliation[Another University]{
    organization={Department},addressline={A street
29},city={Manchester,},postcode={2054 NX},state={State},country={The
Netherlands},}
    \cortext[cor1]{Corresponding author}
    \fntext[1]{This is the first author footnote.}
    \fntext[2]{Another author footnote.}
  
  \begin{abstract}
  Here wii be the abstract.

  It consists of two paragraphs.
  \end{abstract}
    \begin{keyword}
    keyword1 \sep 
    keyword2
  \end{keyword}
  
 \end{frontmatter}

\hypertarget{abstract}{%
\section*{Abstract}\label{abstract}}
\addcontentsline{toc}{section}{Abstract}

\begin{itemize}
\tightlist
\item
  Phenomenon
\item
  Effects
\item
  bibliography
\item
  our approach
\item
  our results
\end{itemize}

CE: cloud enhancement cases one minute

ECE: extreme cloud enhancement cases one minute over TSI on horizontal
plane

CEG: cloud enhancement groups, cases events with consecutive CE

\(\text{GHI}_\text{i}\): measured one minute global horizontal
irradiance

\(\text{GHI}_\text{CSm}\): modeled clear sky one minute global
horizontal irradiance

\hypertarget{intro}{%
\section{Intro}\label{intro}}

The shortwave solar irradiance, reaching Earth's surface, is the main
energy source of the atmosphere and biosphere and drives and governs the
climate \citep{Gray2010}. It has direct practical application, in
different industries, like energy production and agriculture method. The
variability of its intensity, can cause difficulties in predicting the
yield, and designing the specifications of the appropriate equipment. A
lot of research has been focused on predicting the renewable energy
production in a fine timescale and in near real-time (for a review see
\citet{Inman2013}; \citet{Graabak2016}).

A big aspect of this variability is the interaction with the clouds. In
general, clouds absorb part of the solar irradiance, but under certain
conditions, can enhance the total shortwave irradiance reaching the
ground. This effect, can locally increase the observed total shortwave
irradiance higher than the expected clear sky irradiance {[}see
references therein{]}.

Some of the proposed underling mechanisms of those events, have been
summarized by \citet{Gueymard2017}, and include scattering on the edge
of cumulus clouds or through thin cirrus. Further investigation with
radiation transfer modeling methods and observations have pointed as the
prevailing mechanism, the forward Mie scattering
\citep{Pecenak2016, Thuillier2013, Yordanov2013, Yordanov2015}, through
the clouds. The overall phenomenon depends on different interactive
factors, that include cloud thickness, constitution and type; and the
relative position of the sun, the clouds, and the observer
\citep{Gueymard2017, Veerman2022}. As such, there are multiple
contributing mechanism that are responsible for the observed irradiance
enhancements.

Cloud enhancements have been reported, to be able to exceed in intensity
even the value of the solar constant, resulting of clear indexes above
unit. A summary of extreme enchantment cases have been compiled by
\citet{Martins2022}. There are also some practical implication of the
cloud enchantments. The intensity and duration of enhancements can
effect the efficiency and stability of photovoltaic power production
\citep{Lappalainen2020, Jaervelae2020}, and extreme enhancements cases,
have the potential to compromise the integrity of photovoltaic plants
infrastructure \citep{DoNascimento2019}. It has also been demonstrated,
that these events can interfere in the comparison of ground data and
satellite observations \citep{Damiani2018}

Methods of identification cloud enchantment events in the literature,
usually include the use of a simulated clear sky radiation as baseline,
that is combined with an appropriate threshold or some other statistical
characteristics, and in some cases, with visual methods with a sky cam
\citep[ and references therein]{Vamvakas2020, Mol2023}.

In this study, we evaluated the effects cloud enhancements on the total
downward radiation by studding the occurrences, their intensity, and
their duration in a thirty-year period at the city of Thessaloniki. We
used modeled clear sky irradiance, as a baseline to identify cloud
enhancements. We were able to determine some trends of the phenomenon,
it's Climatology, and some of their general characteristics. We weren't
able to find a comparable study that provides trends for similar long
term dataset, as ours. The recording of the radiation signal is the mean
of one-minute. Thus, the minimum resolution of cloud enhancement events
(CE), in this study is one minute.

In the relative bibliography different definition have been used for
these events, some of the are summarized by \citet{Gueymard2017}. Here,
we defined as cloud enhancement events (CE) the cases when the measured
global horizontal irradiance (GHI) at ground level, exceeds the expected
value under clear-sky conditions. Similar, we define as extreme cloud
enhancement events (ECE), the cases when GHI, exceeds the Total Solar
Irradiance on horizontal plane at ground level. Although the duration of
these bursts varies, from instantaneous to several minutes, here we are
constrained by the recorded data, to one-minute steps.

\hypertarget{data-and-methodology}{%
\section{Data and methodology}\label{data-and-methodology}}

\hypertarget{ghi-data}{%
\subsection{GHI data}\label{ghi-data}}

The monitoring site is operating in the Laboratory of Atmospheric
Physics of the Aristotle University of Thessaloniki
(\(40^\circ\,38'\,\)N, \(22^\circ\,57'\,\)E, \(80\,\)m~a.s.l.). In this
study we present data from the period 13~April 1993 to 31~December 2023.
The GHI data were measured with a Kipp~\& Zonen CM-21 pyranometer.
During the study period, the pyranometer was independently calibrated
three times at the Meteorologisches Observatorium Lindenberg, DWD,
verifying that the stability of the instrument's sensitivity was better
than \(0.7\,\%\) relative to the initial calibration by the
manufacturer. For the acquisition of radiometric data, the signal of the
pyranometer was sampled at a rate of \(1\,\text{Hz}\). The mean and the
standard deviation of these samples were calculated and recorded for
every minute. The measurements were corrected for the zero offset
(``dark signal'' in volts), which was calculated by averaging all
measurements recorded for a period of \(3\,\text{h}\), before (morning)
or after (evening) the Sun reaches an elevation angle of \(-10^\circ\).
The signal was converted to irradiance using the ramped value of the
instrument's sensitivity between subsequent calibrations.

To further improve the quality of the irradiance data, a manual
screening was performed to remove inconsistent and erroneous recordings
that can occur stochastically or systematically during the continuous
operation of the instruments. The manual screening was aided by a
radiation data quality assurance procedure, adjusted for the site, which
was based on the methods of Long and Shi~\citep{Long2006, Long2008a}.
Thus, problematic recordings have been excluded from further processing.
Although it is impossible to detect all false data, the large number of
available data, and the aggregation scheme we used, ensures the quality
of the radiation measurements used in this study. To preserve an
unbiased representation of the data we applied a constraint, similar the
one used by \citet{CastillejoCuberos2020}. Where, for each valid hour of
day, there must exist at least 45 minutes of valid measurements,
including nighttime near sunrise and sunset. Days with less than 5 valid
hours are rejected completely. Furthermore, due to the significant
measurement uncertainty when the Sun is near the horizon, and due to
some systematic obstructions by nearby buildings, we have excluded all
measurements with solar zenith angle (SZA) greater than \(78^\circ\).

\hypertarget{cloud-enhancement-detection}{%
\subsection{Cloud enhancement
detection}\label{cloud-enhancement-detection}}

To be able to detect the CE cases, we had to establish a baseline, above
which we can characterize each data point as an enhancement event, by
estimating the occurring over irradiance (OIR). The OIR, here is defined
as the irradiance difference of the measured one-minute \(\text{GHI}_i\)
from the CE identification criterion in
Equation\nobreakspace\ref{eq:CE4}
(\(\text{OIR}_i = \text{GHI}_i - \text{GHI}_\text{CSlim}\)). To have an
estimation, and a first insight on the phenomenon, we experimented with
two simple approaches for the reference. The Haurwitz's model
\citep{Haurwitz1945}, which is as simple clear sky model, and we had
already adjusted and had good fit with our data \citep{Natsis2023}, and
the TSI at the top of the atmosphere. We have tested both cases by using
an appropriate relative threshold and/or an additional constant offset.
The initial results, showed that we can detect a big portion of the CE
events. These results were helpful, and helped to establish some
criteria to further improve the CE identification. It was evident, by
inspecting the daily plot of irradiance, that changes on the atmospheric
conditions introduced numerous false positive and false negative
results. To produce a more accurate reference, we had to take into
account more factors that effect the clear sky radiation. So we used a
radiation transfer model.

\hypertarget{modeled-clear-sky-irradiance}{%
\subsection{Modeled clear Sky
Irradiance}\label{modeled-clear-sky-irradiance}}

We approximated the expected clear sky GHI by using a radiation transfer
model. The simulations was performed by the well established model
Libradtran \citep{Emde2016}, a similar approach, was also used by
\citet{Vamvakas2020} for creating a clear sky reference. Because of the
lack of observational data for the whole period, we used some long term
climatological data for the main factors responsible for the attenuation
of the broadband downward solar radiation in the atmosphere. Which are
mainly, the aerosols and the water vapors. Fortunately, our site
participates in the Aerosol Robotic Network (AERONET)
\citep{Giles2019, Buis1998}, as we operate a Cimel photometer since
2003, collocated with the CM-21 pyranometer. The mean monthly aerosol
optical depth (AOD) on different wavelengths is provided by AERONET,
along with the equivalent water column height in the atmosphere.

For completeness, we will describe here the main points of the radiation
simulation procedure. We used as input the spectrum of
\citet{Kurucz1994} in the range \(280\) to \(2500\,\text{nm}\), with the
Libradtran radiation transfer solver ``disort'' on a ``pseudospherical''
geometry and the ``LOWTRAN'' gas parameterization. For each combination
of conditions we use a SZA step of \(0.2^\circ\). For the atmospheric
characteristics, we iterated for combinations of AOD at
\(500\,\text{nm}\) (\(\tau_{500\text{nm}}\)) with additional offsets of
\(\pm1\) and \(\pm2\sigma\), and water column (\(wc\)), also with
offsets of \(\pm1\) and \(\pm2\sigma\). We applied them on two
atmospheric profiles, from the Air Force Geophysics Laboratory (AFGL).
The ``AFGL atmospheric constituent profile, midlatitude summer''
(afglms) and the ``AFGL atmospheric constituent profile, midlatitude
winter'' (afglmw) \citep{Anderson1986}.

To create a look-up table that aligns with our dataset, we applied some
adjustments. To account for the Sun's variability, in our one-minute GHI
measurements, we adapted each modeled value by scaling the\\
model's input spectrum integral, to the corresponding total solar
irradiance (TSI) provided by NOAA \citep{Coddington2005}.\\
Also, we applied the effect of the Earth -- Sun distance on the
irradiance, by using the distance calculated by the Astropy
\citep{AstropyCollaboration2022} software library. As needed, we
interpolate the resulting irradiances to the exact SZA of our
measurements. For each period of the year, we used the appropriate
atmospheric profile (afglms or afglmw). Finally, we calculated the clear
sky irradiance value at the horizontal plane. Thus, we were able to
emulate different atmospheric condition and levels of atmospheric
clearness for the climatological conditions of the site. With this
method, the modeled clear sky irradiances can be directly compared to
each measured one-minute value of GHI, for different conditions of
atmospheric clearness.

\hypertarget{ce-criteria-investigation}{%
\subsection{CE Criteria investigation}\label{ce-criteria-investigation}}

The use of the actual modeled values of clear sky GHI alone, can not
provide us with a robust method to distinguish the CE cases, due to the
limited accuracy of the input data. Our main focus was to positively
identify over irradiance events from CEs, thus we used a relative
factor, to create an upper envelope of the clear sky irradiance, above
which, any GHI value can safely attributed to CE. We evaluated the
performance of the modeled clear radiation for each of atmospheric level
of clearness, as reference, in order to conclude which is the most
appropriate.

To select the exact values of these thresholds factors (Equation
\ref{eq:CE4}), we implemented an empirical method, by manual inspection
of the CE identification, on specially selected days from the whole
dataset. We used seven sets of selected days, with characteristics
relevant to the efficiency of the identification threshold. These sets
were random groups of about 20 to 30 days with the following
characteristics: (a) the largest over irradiance CE events, (b) the
largest daily total over irradiance, (c) without clouds (by implementing
a clear sky identification algorithm as discussed in
\citet{Natsis2023}), (d) without clouds and without EC events, (e) with
at least \(60\,\%\) of the day length without clouds and some EC events,
(h) random days and (i) some manual selected days that were included
during the manual inspection. Where it was needed, for some of the edge
cases, we also used images from a sky-cam, to further aid the decisions
of the manual inspection.

After evaluating the modeled clear radiation for the different
atmospheric conditions, in relation to the measured GHI data, we choose
as a representative of the clear sky radiation, the case where
\(\tau_{\text{cs}} = \tau_{500\text{nm}} - 1\sigma\) and
\(w_{h\text{cs}} = w_h - 1\sigma\) (\(\text{GHI}_\text{CSm}\)). These
values represent a typical atmosphere in Thessaloniki with low load of
aerosols and humidity, which are the main factor that attenuate the GHI,
excluding clouds. To create the limit of CE identification, we created a
two branched threshold, as a function of SZA. A constant factor for low
SZAs, and a higher ramped factor for higher SZAs
(Equation\nobreakspace\ref{eq:CE4}). This is the criterion of our CE
identification. \begin{equation}
\text{CE} : \text{E}_\text{i} > \text{E}_\text{CSlim,i}, \text{where} \begin{cases}
 \text{E}_\text{CSlim,i} = 1.05 \cdot \text{E}_\text{CSm,i}, & \text{$\theta \leq 60^\circ$}\\
\text{E}_\text{CSlim,i} = \left ({ 1.18 + \frac{1.05 - 1.18}{60 - 78} \cdot (\theta- 78) } \right ) \cdot \text{E}_\text{CSm,i}, & \text{$ 78^\circ > \theta > 60^\circ$}\\
\text{Excluded measurements}, & \theta > 78^\circ
\end{cases}\label{eq:CE4}
\end{equation} where: \(\theta\) is the solar zenith angle, \(\text{E}\)
the measured irradiance, \(\text{E}_\text{CSm}\) the selected modeled
clear sky irradiance, and \(i\) each of the one-minute observation.

We have to note, that the differentiation of the threshold factor was
needed, because of the high irradiance values we observed, early in the
morning and late in the afternoon. We have confirmed, by inspecting
images from the sky cam, that we have duration of elevated irradiance,
either due to the clearness of the atmosphere, or some interferences by
reflections on nearby bright surfaces. Although, these cases may mask
some of the CE events. The actual CE events, produce higher irradiance
and thus are identified as such. As a side effect, the reported OIR will
be slight underestimated for those SZAs, but the overall contribution of
those cases to the total daily energy, is minimal due the low
occurrences and the lower irradiances.

\ldots\ldots\ldots\ldots.

These selections have some subjectivity, as the definition of clear sky,
is depended on the intended usage. Our main focus is to be able to
identify the OIR created by the clouds. Although, the aerosols and water
vapor are always present and can not be completely removed. The effect
of the different clearness levels

\ldots\ldots\ldots\ldots.

Another aspect of the CE events, that is often reported in the relative
bibliography, are cases of extreme cloud enhancement (ECE). These are
cases of CE where the measured intensity of the irradiance, exceeds the
equivalent TSI on the top of the atmosphere, and satisfy the
Equation\nobreakspace{}\ref{eq:ECE}. \begin{equation}
\text{ECE}: \text{GHI}_\text{i} > \cos(\theta) \times E_{i\odot} / r_{i}^2
\label{eq:ECE}
\end{equation} where: \(\theta\) the solar zenith angle, \(E_{\odot}\)
the solar constant, \(r\) the Sun -- Earth distance, and \(i\) each of
the one-minute observation.

\ldots\ldots.

\begin{itemize}
\tightlist
\item
  Include Example of days plot in the Appendix! \ldots{} this is clearer
  to understand and describe \ldots.
\end{itemize}

\ldots\ldots.

\hypertarget{results}{%
\section{Results}\label{results}}

Our dataset, after the data selection processing, consists of 6144534
records of GHI, of which \(1.799\,\%\) are CE and \(0.036\,\%\) are ECE
events. The highest GHI recorder was \(1416.6\,W/m^2\) on 24~May 2007.
The absolute stronger CE event had an OIR of \(341.79\,W/m^2\) on 15~May
2014. The relative stronger CE event was \(54\,\%\) above the clear sky
threshold, on 28~October 2016.

\hypertarget{trends}{%
\subsection{Trends}\label{trends}}

We computed the daily trend of the mean OIR of CE
(Figure\nobreakspace{}\ref{fig:CEmeanDaily}), using a first-order
autoregressive model with lag of 1 day, using the `maximum likelihood'
fitting method \citep{Gardner1980, Jones1980} by implementing the
function `arima' from the library `stats' of the R programming language
\citep{RCT2023}. The trends are reported together with the \(2\sigma\)
error. We observe an increase of \(0.216\pm 0.083\,W/m^2/y\) on the mean
OIR.

\begin{figure}

{\centering \includegraphics[width=0.5\linewidth]{../images/P_daily_trend-1} 

}

\caption{Daily mean values and trend of the CE over irradiance.}\label{fig:CEmeanDaily}
\end{figure}

Although, the previous result is closer to the raw data, we preferred to
present the annual statistics, that give a more clear picture about the
long term CE trends. Hence, we calculated the annual values from the
one-minute measurements. The trend of OIR is \(0.23\pm 0.11\,W/m^2/y\)
(Figure\nobreakspace{}\ref{fig:P-energy-mean}), but this value alone is
not very useful due the intrinsic high variability of this metric.

\begin{figure}

{\centering \includegraphics[width=0.5\linewidth]{../images/P_energy-7} 

}

\caption{Trends of the mean OIR per CE.}\label{fig:P-energy-mean}
\end{figure}

A better indicator of changes on the characteristic of CE would be the
number of CE occurrences and the total energy of the CE over irradiance.
The annual number of CE occurrences, shows a steady increase of
\(122.2\pm 22.8\,\text{cases}/y\)
(Figure\nobreakspace{}\ref{fig:P-energy-N}). We have to note that the
energy related to the CE events can not be directly linked with the
total energy balance on the atmosphere. The net sun radiation of the
region is not increased, but rather redistributed through the CE.
Although, the instantaneous values, can exceed the equivalent clear sky
irradiance to a considerable level.

\begin{figure}

{\centering \includegraphics[width=0.5\linewidth]{../images/P_energy-6} 

}

\caption{Trend of yearly CE number of occurancies.}\label{fig:P-energy-N}
\end{figure}

Subsequently, the mean annual increase of energy due to CE events is
\(341.6\pm 73.1\,kJ/y\) (Figure~\ref{fig:P-energy-sum}), witch follows
the trend of the number of occurrences.

\begin{figure}

{\centering \includegraphics[width=0.5\linewidth]{../images/P_energy-5} 

}

\caption{Trend of the yearly excess energy due to CE over irradiance}\label{fig:P-energy-sum}
\end{figure}

\begin{figure}

{\centering \includegraphics[width=0.5\linewidth]{../images/P_energy-8} 

}

\caption{Trend of the yearly median over irradiance due to CE over irradiance}\label{fig:P-energy-median}
\end{figure}

\hypertarget{climatology}{%
\subsection{Climatology}\label{climatology}}

Another interesting aspect of the CE cases, is their seasonal cycle. In
Figure\nobreakspace{}\ref{fig:relative-month-occurancies}, we have the
box plot (whisker plot), where the values have been normalized by the
highest median value, that occurs in May. Although the number of
occurrences has a wide spread throughout the study period, the most
active period of CE occurrences is during May and June. During the
Winter (December -- February) the CE cases are about \(25\,\%\) of the
maximum. The rest of the months the occurrences ramp between the maximum
and minimum.

\begin{figure}

{\centering \includegraphics[width=0.5\linewidth]{../images/clim_CE_month_norm_MAX_median_N-2} 

}

\caption{Statistics of the number of CE occurancies for each month. The box represents the values of the low $25\,\%$ percentile to $75\,\%$ percentile, where the thick horizontal line inside is the mean, the verical lines extend to the macimum and minimum vales, the dots are outlier values, and the rhombus is the mean.}\label{fig:relative-month-occurancies}
\end{figure}

The distribution of the CE over irradiance spreads uniformly
Figure\nobreakspace{}\ref{fig:ovir-distribution}. Where there is an
inverse relation between the events frequency and events intensity. This
is expected as, the stronger the CE events become, the rarer the
particular atmospheric and sun conditions occur.

\begin{figure}

{\centering \includegraphics[width=0.5\linewidth]{../images/P-relative-distribution-diff-1} 

}

\caption{Distribution of CE over irradiance}\label{fig:ovir-distribution}
\end{figure}

\hypertarget{groups-stats}{%
\subsection{Groups stats}\label{groups-stats}}

In order to further study the characteristics of the CE events, we
grouped the single minute CE events, to continuous CE groups (CEG).
Thus, a CEG consists of one or more successive CE cases, and can have a
duration of multiple of minutes. We have identified 29009 groups, where
the group with the longest duration lasted 140 minutes on 07~July 2013.
By examining the frequency distribution of the CEG durations
(Figure\nobreakspace{}\ref{fig:ceg-duration-distribution}), we can
conclude that longer durations are getting increasingly rare, with
durations above 10 minutes very rare.

\begin{figure}

{\centering \includegraphics[width=0.5\linewidth]{../images/groups-1} 

}

\caption{Distribution of CE groups}\label{fig:ceg-duration-distribution}
\end{figure}

\ldots{} \citet{Zhang2018} \ldots.

\begin{figure}

{\centering \includegraphics[width=0.5\linewidth]{../images/P-groups-bin2d-1} 

}

\caption{Relation of mean over irradiance and CE group duration}\label{fig:unnamed-chunk-3}
\end{figure}

\hypertarget{extreme-ce-above-tsi}{%
\subsection{Extreme CE above TSI}\label{extreme-ce-above-tsi}}

Another aspect of the CE events are the cases when the irradiance is
above the expected irradiance on top of the atmosphere, we have defined
this as ECE (Equation\nobreakspace{}\ref{eq:ECE}).

\begin{itemize}
\tightlist
\item
  max value ECE relative to other works max values
\end{itemize}

Analogous to Figure\nobreakspace{}\ref{fig:relative-month-occurancies}
we have computed the distribution of the number of occurrences of ECE
events by month in
Figure\nobreakspace{}\ref{fig:relative-month-occurancies-ECE}. Here the
most active period is in the spring (March -- May), followed by the
months of June, September and October.

\ldots\ldots.

\begin{figure}

{\centering \includegraphics[width=0.5\linewidth]{../images/clim_ECE_month_norm_MAX_median_N-2} 

}

\caption{Distribution of }\label{fig:relative-month-occurancies-ECE}
\end{figure}

\begin{figure}

{\centering \includegraphics[width=0.5\linewidth]{../images/extremedistributions-2} 

}

\caption{Distribution of ECE above 'clear sky' reference}\label{fig:unnamed-chunk-4}
\end{figure}

\begin{itemize}
\tightlist
\item
  SZA
\end{itemize}

\hypertarget{discussion-and-conclusions}{%
\section{Discussion and conclusions}\label{discussion-and-conclusions}}

Compare to other location stats. max, ECE, distributions \ldots.

Climatology results

\hypertarget{appendix}{%
\section*{Appendix}\label{appendix}}
\addcontentsline{toc}{section}{Appendix}

\bibliography{bibliography.bib}


\end{document}
