\documentclass[preprint, 5p,
authoryear]{elsarticle} %review=doublespace preprint=single 5p=2 column
%%% Begin My package additions %%%%%%%%%%%%%%%%%%%

\usepackage[hyphens]{url}

  \journal{Atmospheric Research} % Sets Journal name

\usepackage{graphicx}
%%%%%%%%%%%%%%%% end my additions to header

\usepackage[T1]{fontenc}
\usepackage{lmodern}
\usepackage{amssymb,amsmath}
% TODO: Currently lineno needs to be loaded after amsmath because of conflict
% https://github.com/latex-lineno/lineno/issues/5
\usepackage{lineno} % add
\usepackage{ifxetex,ifluatex}
\usepackage{fixltx2e} % provides \textsubscript
% use upquote if available, for straight quotes in verbatim environments
\IfFileExists{upquote.sty}{\usepackage{upquote}}{}
\ifnum 0\ifxetex 1\fi\ifluatex 1\fi=0 % if pdftex
  \usepackage[utf8]{inputenc}
\else % if luatex or xelatex
  \usepackage{fontspec}
  \ifxetex
    \usepackage{xltxtra,xunicode}
  \fi
  \defaultfontfeatures{Mapping=tex-text,Scale=MatchLowercase}
  \newcommand{\euro}{€}
\fi
% use microtype if available
\IfFileExists{microtype.sty}{\usepackage{microtype}}{}
\usepackage[]{natbib}
\bibliographystyle{elsarticle-num-names}

\ifxetex
  \usepackage[setpagesize=false, % page size defined by xetex
              unicode=false, % unicode breaks when used with xetex
              xetex]{hyperref}
\else
  \usepackage[unicode=true]{hyperref}
\fi
\hypersetup{breaklinks=true,
            bookmarks=true,
            pdfauthor={},
            pdftitle={Long term cloud enhancement events of global solar irradiance over Thessaloniki, Greece},
            colorlinks=false,
            urlcolor=blue,
            linkcolor=magenta,
            pdfborder={0 0 0}}

\setcounter{secnumdepth}{5}
% Pandoc toggle for numbering sections (defaults to be off)


% tightlist command for lists without linebreak
\providecommand{\tightlist}{%
  \setlength{\itemsep}{0pt}\setlength{\parskip}{0pt}}




\usepackage{caption}
\usepackage{placeins}
\captionsetup{font=small}
\usepackage{subcaption}
\usepackage{booktabs}
\usepackage{longtable}
\usepackage{array}
\usepackage{multirow}
\usepackage{wrapfig}
\usepackage{float}
\usepackage{colortbl}
\usepackage{pdflscape}
\usepackage{tabu}
\usepackage{threeparttable}
\usepackage{threeparttablex}
\usepackage[normalem]{ulem}
\usepackage{makecell}
\usepackage{xcolor}



\begin{document}


\begin{frontmatter}

  \title{Long term cloud enhancement events of global solar irradiance
over Thessaloniki, Greece}
    \author[LAP]{Athanasios N. Natsis%
  \corref{cor1}%
  }
   \ead{natsisphysicist@gmail.com} 
    \author[LAP]{Alkiviadis Bais%
  %
  }
   \ead{abais@auth.gr} 
    \author[LAP]{Charikleia Meleti%
  %
  }
   \ead{meleti@auth.gr} 
      \affiliation[LAP]{
    organization={Laboratory of Atmospheric Physics, Physics Department,
Aristotle University of
Thessaloniki},city={Thessaloniki},postcode={54124},country={Greece},}
    \cortext[cor1]{Corresponding author}
    \fntext[1]{This is the first author footnote.}
    \fntext[2]{Another author footnote.}
  
  \begin{abstract}
  Here will be the abstract.
  \end{abstract}
    \begin{keyword}
    cloud enhancement \sep total solar radiation \sep global horizontal
irradiance \sep 
    over irradiance
  \end{keyword}
  
 \end{frontmatter}

\hypertarget{abstract}{%
\section*{Abstract}\label{abstract}}
\addcontentsline{toc}{section}{Abstract}

We studied the enhancement of the global horizontal irradiance (GHI),
over Thessaloniki, for the period 1993 -- 2023, on one minute average
records. We identify the cloud enhancement events (CE), by creating an
appropriate clear sky irradiance reference with the use of a radiation
transfer model, and aerosol optical depth (AOD) data, from the AERONET,
and a collocated Brewer photospectrometer. We found that there is a
trend of CE cases of \(45.6\pm 21.9\,\text{cases}/\text{year}\), and a
mean total energy change, of the CE events by
\(116.9\pm 67.8\,\text{kj}/\text{year}\) with the peak of the CE events
are observed during May and June. An analysis of the total duration of
CE events, showed that durations longer than 5 minutes are very rare,
with exceptions lasting over an hour. We have detected enhancements over
the total solar irradiance, without exceeding the maximum values
reported for other sites, with more favorable CE conditions.

CE: cloud enhancement cases one minute

ECE: extreme cloud enhancement cases one minute over TSI on horizontal
plane

CEG: cloud enhancement groups, cases events with consecutive CE

\(\text{GHI}_\text{i}\): measured one minute global horizontal
irradiance

\(\text{GHI}_\text{CSm}\): modelled clear sky one minute global
horizontal irradiance

\hypertarget{introduction}{%
\section{Introduction}\label{introduction}}

The shortwave solar radiation, reaching Earth's surface, is the main
energy source of the atmosphere and the biosphere, and drives and
governs the climate \citep{Gray2010}. It has direct practical
application in industries like energy and agricultural production. The
variability of its intensity can impose difficulties in predicting the
yield and in designing the specifications of the appropriate equipment.
Significant portion of research has been focused on predicting the
renewable energy production in a fine timescale and in near real-time
(for a review see \citet{Inman2013}; \citet{Graabak2016}).

An important aspect of the variability of solar radiation is its
interaction with the clouds. In general, clouds can attenuate a fraction
of solar irradiance, but under certain conditions, can lead to
enhancement of the global horizontal irradiance (GHI) reaching the
ground. This effect can locally increase the observed GHI to levels even
higher than the expected clear-sky irradiance \citep[ and references
therein]{Cordero2023, Vamvakas2020, CastillejoCuberos2020, Vamvakas2020}.

Some of the proposed underling mechanisms of those enhancements, have
been summarized by \citet{Gueymard2017}; the most important being the
scattering of radiation on the edges of cumulus clouds. It has been also
suggested that enhancement of GHI can be produced by thin cirrus clouds
though refraction and scattering \citep{Thuillier2013}. Further
investigation with radiative transfer modeling and observations pointed
as the prevailing mechanism, the strong forward Mie scattering through
clouds of low optical depth
\citep{Pecenak2016, Thuillier2013, Yordanov2013, Yordanov2015}. Overall,
enhancement event depend on different interactive factors, which include
cloud thickness, structure and type, and the relative position of the
sun and the clouds \citep{Gueymard2017, Veerman2022}.

On multiple sites, cloud enhancements (CE) have been reported to be able
to briefly exceed the value of the solar constant, resulting of
clearness indices above unit. A summary of extreme enchantment cases
(ECE) has been compiled by \citet{Martins2022}. There are also some
practical implications of the cloud enchantments. The intensity and
duration of enhancements can effect the efficiency and stability of
photovoltaic power production \citep{Lappalainen2020, Jaervelae2020}
while ECEs have the potential to compromise the integrity of
photovoltaic plants infrastructure \citep{DoNascimento2019}. It has also
been demonstrated that these events can interfere in the comparison of
ground-based and satellite observations \citep{Damiani2018}. Global
warming has likely affected cloud coverage in the last few decades.
\citet{Liu2023} reported increases in cloud cover over the tropical and
subtropical oceans and decreases over most continents, while
\citet{Dong2023} reported decreases over North America and Europe. To
our knowledge, there is no evidence on whether this trend has affected
also the number and strength of CE events.

Methods of identification of CE events usually include the use of
simulated clear sky irradiance as baseline, that is combined with an
appropriate threshold or some other statistical characteristics, and in
some cases, with visual inspection of sky camera images \citep[ and
references therein]{Vamvakas2020, Mol2023}.

In this study, we evaluate the effects CE on GHI by investigating their
frequency of occurrence, intensity, and duration in a thirty-year period
of GHI observations at Thessaloniki, Greece. We used modeled clear sky
irradiance as a baseline to identify cloud enhancements and we
determined long term trends of the above mentioned metrics, their
climatology and some general characteristics. To our knowledge there are
no other studies that provide trends from such long dataset.

In this study, we define an event as CE when the measured GHI at ground
level, exceeds the expected value under clear-sky conditions. Similarly,
we define as extreme cloud enhancement events (ECE), the cases when GHI
at ground level exceeds the Total Solar Irradiance (TSI). Although the
duration of these bursts can vary from seconds to several minutes, here
we are constrained by the temporal resolution of our data to identify
evens with duration of at minimum one-minute.

\hypertarget{data-and-methodology}{%
\section{Data and methodology}\label{data-and-methodology}}

\hypertarget{instrumentation-and-data}{%
\subsection{Instrumentation and data}\label{instrumentation-and-data}}

The data used in this study were recorded at the monitoring site of the
Laboratory of Atmospheric Physics, Aristotle University of Thessaloniki,
in Thessaloniki, Greece (\(40^\circ\,38'\,\)N, \(22^\circ\,57'\,\)E,
\(80\,\)m~a.s.l.). The GHI data were measured with a Kipp~\& Zonen CM-21
pyranometer and cover the period 13~April 1993 to 31~December 2023.
During the study period, the pyranometer was independently calibrated
three times at the Meteorologisches Observatorium Lindenberg, DWD,
verifying that the stability of the instrument's sensitivity was better
than \(0.7\,\%\) relative to the initial calibration by the
manufacturer. For the acquisition of radiometric data, the signal of the
pyranometer was sampled at a rate of \(1\,\text{Hz}\) with the mean and
standard deviation of these samples calculated and recorded every
minute. The measurements were corrected for the zero offset (``dark
signal'' in volts), which was calculated by averaging all measurements
recorded for a period of \(3\,\text{h}\), before (morning) or after
(evening) the Sun reaches an elevation angle of \(-10^\circ\). The
signal was converted to irradiance using the ramped value of the
instrument's sensitivity between subsequent calibrations.

To further improve the quality of the irradiance data, a manual
screening was performed, in order to remove inconsistent and erroneous
recordings that can occur stochastically or systematically during the
long operation of the instrument. The manual screening was aided by a
radiation data quality assurance procedure, adjusted for the site, which
was based on the methods of Long and Shi~\citep{Long2006, Long2008a}.
Thus, problematic recordings have been excluded from further processing.
Although it is impossible to detect all false data, the large number of
available data, and the aggregation scheme we used, ensures the quality
of the radiation measurements used in this study. To preserve an
unbiased representation of the data we applied a constraint, similar the
one used by \citet{CastillejoCuberos2020}. For each valid hour of day,
there must exist at least 45 minutes of valid measurements, including
nighttime, when near sunrise and sunset. Days with less than 5 valid
hours were rejected completely. Furthermore, due to the significant
measurement uncertainty in GHI when the Sun is near the horizon, and due
to some systematic obstructions by nearby buildings, we have excluded
all measurements with solar zenith angle (SZA) greater than
\(78^\circ\).

Finally, images from a sky camera have been used in the manual
inspection of the CE identification. The sky camera operates since 2012
and stores images in 5 min time steps.

\hypertarget{cloud-enhancement-detection}{%
\subsection{Cloud enhancement
detection}\label{cloud-enhancement-detection}}

For the detection of CE cases we established a baseline of irradiance
above which we characterized each data point as an enhancement event and
calculated the over irradiance (OIR). The OIR is defined as the
irradiance difference of the measured one-minute \(\text{GHI}_i\) from
the \(\text{GHI}_\text{ref}\) corresponding to cloud-free atmosphere.
First, we tested two simple approaches for the determination of
\(\text{GHI}_\text{ref}\): The Haurwitz's model \citep{Haurwitz1945},
which is a simple clear sky model and was already adjusted and applied
to our data \citep{Natsis2023}, and the total solar irradiance (TSI) at
the top of the atmosphere, adjusted for the Sun-Earth distance. We have
tested both methods by using an appropriate relative threshold and/or an
additional constant offset. The initial results showed that we can
detect a big portion of the CE events. However, by inspecting the daily
plots of irradiance it became evident that changes in atmospheric
conditions introduced numerous false positive and false negative
results. The main reason for these discrepancies is the variability of
the effects of aerosols and water vapor which were not taken into
account in the two simple methods. To produce a more representative
reference we included the effects of these factors through variables
using a radiative transfer model (RTM). The applied methodology is
discussed in section\nobreakspace{}\ref{rtmcs}

\hypertarget{rtmcs}{%
\subsection{Modeled clear Sky Irradiance}\label{rtmcs}}

\hypertarget{climatology-of-clear-sky-irradiance}{%
\subsubsection{Climatology of clear sky
irradiance}\label{climatology-of-clear-sky-irradiance}}

We approximated the expected clear sky \(\text{GHI}_\text{ref}\) with
the radiative transfer model uvspec, part of libRadtran
\citep{Emde2016}, similarly to the approach used by
\citet{Vamvakas2020}. In uvspec we used the solar spectrum of
\citet{Kurucz1994} in the range \(280\) to \(2500\,\text{nm}\), the
radiative trasfer solver ``disort'' in ``pseudospherical'' geometry and
the ``LOWTRAN'' gas parameterization. The model was run for a range of
variables in order to create a look up table (LUT) for the estimation of
the cloud-free reference irradiance for each individual observation of
our dataset. In this context, the model was run for SZAs in the range
\(10\) -- \(90^\circ\) with a step of \(0.2^\circ\) and for the
atmospheric profiles of the Air Force Geophysics Laboratory
\citep{Anderson1986} midlatitude summer and midlatitude winter,
representative of the warm and cold seasons.

Main factors responsible for the attenuation of the broadband downward
solar radiation under cloud free atmospheres are aerosols and water
vapor. At Thessaloniki, such measurements are available since 2003 from
the Cimel sun photometer that belongs to the Aerosol Robotic Network
(AERONET) \citep{Giles2019, Buis1998}. From the observations in the
period 2003 -- 2023 we calculated the monthly climatological means and
standard deviation (\(\sigma\)) for the aerosol optical depth (AOD) at
\(500\,\text{nm}\) and the equivalent height of the water column (WC).
The monthly climatological values of AOD and water column, as well as
combinations with additional offsets of \(\pm1\sigma\) and
\(\pm2\sigma\) , were used as inputs to the RTM in the construction of
the LUT.

For each measurement of the dataset, we calculated from the LUT a
\(\text{GHI}_\text{ref}\) value for the respective season and SZA (by
linear interpolation), and for the climatological values of AOD and WC
of the respective month. The same procedure was followed for the
estimation of the \(\text{GHI}_\text{ref}\) for all combinations of the
AOD and WC with the above mentioned standard deviation offsets. Finally,
each \(\text{GHI}_\text{ref}\) value was adjusted to the actual
Sun-Earth distance derived by the Astropy software library
\citep{AstropyCollaboration2022}.

\hypertarget{long-term-change-of-clear-sky-irradiance}{%
\subsubsection{Long term change of clear sky
irradiance}\label{long-term-change-of-clear-sky-irradiance}}

The above clear sky reference values are based on the climatological AOD
and WC, hence they cannot describe accurately the long term variation of
\(\text{GHI}_\text{ref}\) due to changes in the two atmospheric
constituents, mainly AOD. As reported by \citet{Natsis2023}, there is a
long term brightening effect in the GHI data of Thessaloniki for the
period 1993 -- 2023, which for clear-sky data was attributed to
long-term changes in aerosol effects. Therefore, an adjustment of the
\(\text{GHI}_\text{ref}\) used in this study is necessary. As AERONET
data start only in 2003, we used for the period 1993 -\/-- 2005
estimates of changes in AOD at \(340\,\text{nm}\) derived from a
collocated Brewer spectrophotometer derived from two sources. For 1993
-- 2005 \citep{Kazadzis2007} to calculate the trend in
\(\text{GHI}_\text{ref}\) during this period.

To create a unified trend for the long term change of the clear sky
irradiance, we simulated the values of AOD derived from those sources
with Libradtran. For both inputs, we used the AOD at \(500\,\text{nm}\),
which was inferred by the available Ångström coefficients. We choose the
SZA of \(55^\circ\) as a representative value for all the runs.

According to this study, in the period 1997 -- 2005 the mean AOD at
\(340\,\text{nm}\) \(0.403\) with a trend of \(-3.8\pm0.93\,\%\) per
year, corresponding to a change of \(0.0153\) per year. Using an
Ångström coefficient \(\alpha = 1.6\), this translates to a change in
the Ångström coefficient \(\beta=0.00272\) per year (or \(\beta=0.084\)
in 1997 and \(\beta=0.059\) in 2005). Simulations with uvspec for the
above Ångström coefficients, with WC of \(15.7\,\text{mm}\) and
\(15.4\,\text{mm}\) for 1997 and 2005 respectively and for a SZA of
\(55^\circ\) reveal a trend of \(+0.21\,\%\) per year in
\(\text{GHI}_\text{ref}\). The SZA of \(55^\circ\) was chosen as
representative of all days in the year in order to get a rough estimate
of the annually averaged change in clear sky irradiance. For the period
2005 -- 2023 we used the mean monthly values of AOD and WC from AERONET
in a similar simulation scheme to calculate the monthly mean of clear
sky irradiance, and finally the trend of \(+0.14\,\%\) per year. We
applied these two long term changes (see
Figure\nobreakspace{}\ref{fig:CS-change}) to the climatological
\(\text{GHI}_\text{ref}\), in order to create a more realistic
representation of the clear-sky irradiance for the whole period of
study.

\begin{figure}

{\centering \includegraphics[width=1\linewidth]{../images/P-CS-change-1} 

}

\caption{Simulated long term change in clear sky irradiance relative to the climatological values due to changes in AOD.}\label{fig:CS-change}
\end{figure}

\hypertarget{criteria-for-the-identification-of-ce-events}{%
\subsection{Criteria for the identification of CE
events}\label{criteria-for-the-identification-of-ce-events}}

In this study our main focus was to quantify over irradiance events
related to CEs. A key issue for achieving this goal is to define a
threshold for the CE identification, representative of the clear-sky
irradiance at the time of each GHI measurement. This depends on the
selection of the appropriate atmospheric parameterization for the RTM
simulations. The implementation of the long term change of AOD,
discussed in section\nobreakspace{}\ref{rtmcs}, allows capturing a large
part of the natural variability of clear-sky GHI. However, the
short-term variability of AOD cannot be taken adequately into account
when using monthly values for the model simulations. We tried different
approaches to increase the robustness of the methodology and compensate
for the limited accuracy of the RTM input data and the unpredictable
natural variability of the atmosphere.

First, we evaluated the performance of the modelled
\(\text{GHI}_\text{ref}\) in relation to the measured GHI for different
levels of atmospheric clearness, by using in the RTM the monthly
climatological AOD and WC, less their respective standard deviations.
These values represent typical atmosphere in Thessaloniki with lower
than average load of aerosols and humidity, which are the main factor
that attenuate the GHI, excluding clouds. With this approach the
simulated \(\text{GHI}_\text{ref}\) should be generally greater than the
measured GHI when aerosols are more abundant. To compensate for this, we
increased the \(\text{GHI}_\text{ref}\) by (\(4\,\%\)) with an
additional constant offset of \(15\,\text{W}/\text{m}^2\), as described
in Equation\nobreakspace\ref{eq:CE4}. \begin{equation}
\text{CE} : \text{E}_\text{i} > 15 + 1.04 \cdot \text{E}_\text{CSm,i} \,\,[\text{W}/\text{m}^2] \label{eq:CE4}
\end{equation} where: \(\text{E}\) the measured irradiance,
\(\text{E}_\text{CSm}\) the selected modelled clear sky irradiance, and
\(i\) each of the one-minute observation. This is the criterion of our
CE identification.

These values were determined through the implementation of an empirical
method with manual inspection of the CE identification results on
selected days of the whole dataset. We used seven sets of days with
characteristics relevant to the efficiency of the identification
threshold. These sets were random groups of about 20 -- 30 days with the
following characteristics: (a) the strongest over irradiance CE events,
(b) the largest daily total over irradiance, (c) absence of clouds (by
implementing a clear sky identification algorithm as discussed in
\citet{Natsis2023}), (d) absence of clouds and absence of EC events, (e)
with at least \(60\,\%\) of the day length without clouds and presence
of EC events, (h) randomly selected days, and (i) manually selected
days. Where needed in some of the edge cases, we also used images from a
sky-camera to further aid the decision of the manual inspection.

The definition of the CE events with this method has a degree of
subjectivity, since the actual clear sky irradiance is not known and can
only be approximated. However, this method is capable in detecting all
major CE events. Where some CE events with very low OIR may be not
detected, these are few with small over-irradiance and it is unlikely
that will effect significantly our results.

A sub-category of the CE events that is often discussed in the relevant
bibliography \citep{Cordero2023, Martins2022, Yordanov2015}, are the
extreme cloud enhancement (ECE) events. These are cases of CE where the
measured intensity of the irradiance exceeds the TSI at the top of the
atmosphere: \begin{equation}
\text{ECE}: \text{GHI}_\text{i} > \cos(\theta) \times E_{i\odot} / r_{i}^2
\label{eq:ECE}
\end{equation} where: \(\theta\) the solar zenith angle, \(E_{\odot}\)
the solar constant adjusted for the actual Sun -- Earth distance, and
\(i\) each of the one-minute observation.

An example of CE identification for a selected day is given in the
Figure\nobreakspace{}\ref{fig:example-day}, where the daily course of
the clear sky reference irradiance and the CE and ECE identification
thresholds are shown along with the actual GHI measurements. In
addition, we provide an example scatter plot between the measured and
the modeled clear-sky irradiance for one year, where the CE and ECE
events are clearly grouped above the threshold of irradiance
(Figure\nobreakspace{}\ref{fig:example-year}).

\begin{figure}[H]

{\centering \includegraphics[width=1\linewidth]{../images/example-days-18} 

}

\caption{Example of cloud identification for 2019-07-11. The green line with blue symbols depicts the measured GHI in one minute steps. The red line shows the modelled threshold for the detection of CE events, which are denoted with red circles. The black curve represents the TSI at the top of the atmosphere, adjusted for the actual Sun-Earth distance and multiplied by the cosine of the cosine of SZA.}\label{fig:example-day}
\end{figure}

\begin{figure}[H]

{\centering \includegraphics[width=1\linewidth]{../images/P-example-years-13} 

}

\caption{Example scatter plot of the measured GHI and the reference clear sky irradiance for the year 2005. The over-irradiance for CE and ECE events is color coded, while the remaining data points are shown in black.}\label{fig:example-year}
\end{figure}

\hypertarget{results}{%
\section{Results}\label{results}}

Following the application of the above discussed methodology to the
entire dataset (6.1 million of one-minute GHI measurements),
\(1.764\,\%\) were identified as CE events and \(0.036\,\%\) as ECE
events. The highest recorded GHI due to CE was
\(1416.6\,\text{W}/\text{m}^2\) on 24~May 2007, corresponding to OIR of
\(345.9\,\text{W}/\text{m}^2\). The stronger CE event of \(49.7\,\%\)
above the clear sky threshold was observed on 28~October 2016. In the
following sections we are discussing the long-term trends and
variability of the CE events as well as of the corresponding OIR and
excess irradiation.

\hypertarget{long-term-trends}{%
\subsection{Long-term trends}\label{long-term-trends}}

The main focus of this study is to investigate the time evolution of the
CE events by analyzing the GHI measurements at Thessaloniki. Cloud
enhancements can be influenced by different factors, such as the
geometry, size and optical thickness of clouds, their height in the
atmosphere and on local weather regimes \textbf{!!!(ref)!!!}. Some of
these factors are related to changes in climate; hence it would be
reasonable to expect capturing their contributions to the frequency of
occurrence of CE events over Thessaloniki, as well as to the average OIR
and excess irradiation. The long-term trends were calculated using a
first-order autoregressive model with the `maximum likelihood' fitting
method \citep{Gardner1980, Jones1980}, by implementing the function
`arima' from the library `stats' of the R programming language
\citep{RCT2023}. All trends are reported together with their \(2\sigma\)
error.

Figure\nobreakspace{}\ref{fig:P-energy} shows the time series of the
yearly number of CE cases (each with duration of one minute), the yearly
mean OIR and the yearly excess irradiation for the period 1993 -- 2023,
together with corresponding linear trends. All three quantities show
increasing trends, most pronounced for the frequency of occurrence
(\(45.6\pm 21.9\,\text{cases}/\text{year}\)) and the excess irradiation
(\(116.9\pm 67.8\,\text{kJ}/\text{year}\)), which are also statistically
significant. In contrast the trend of the yearly mean OIR is negligible
(\(0.1\pm 0.1\,\text{W}/\text{m}^2/\text{year}\)) and of no statistical
significance. The average OIR for the entire period is about
\(39.9\,\text{W}/\text{m}^2\) with standard deviation of
\(2.7\,\text{W}/\text{m}^2\). The interannual variability of the data
about the trend lines is quite large. Furthermore, it tends to increase
with time (at least for the quantities of panels b and c), suggesting a
significant variability in cloud patterns over the area, possibly
associated to changes in climate.

We have to note that the excess irradiation related to the CE events can
not be directly linked to the total energy balance of the atmosphere.
The net solar radiation of the region is not increased, but is rather
redistributed through the CE events. This is also depicted by the ECE
irradiance values, which can exceed the equivalent clear sky irradiance
by a significant amount.

\begin{figure}% [h!]
        {\centering 
            \subfloat[\label{fig:P-energy-mean}]
                {\includegraphics[width=\linewidth]{../images/P-energy-3} }\\
            \subfloat[\label{fig:P-energy-N}]
                {\includegraphics[width=\linewidth]{../images/P-energy-2} }\\
            \subfloat[\label{fig:P-energy-sum}]
                {\includegraphics[width=\linewidth]{../images/P-energy-1} }
        }
    \caption{Time series for the period 1992 -- 2023 of (a) the yearly CE number of occurrences, (b) the yearly mean OIR and (c) the yearly excess irradiation. The black lines represent the linear trends on the yearly data.}\label{fig:P-energy}
\end{figure}

\hypertarget{climatology-of-cloud-enhancement-events}{%
\subsection{Climatology of cloud enhancement
events}\label{climatology-of-cloud-enhancement-events}}

!!! monthly weigth !!!

Another interesting aspect of the CE events is their distribution within
the year. Figure\nobreakspace{}\ref{fig:relative-month-occurrences}
shows the monthly box and whisker plot of the CE number of occurrence
normalized with the highest median value, that occurs in June, depicting
a clear seasonal cycle. Although CE events are present throughout the
year, the most active period is during May and June. During the winter
(December -- February), the number of CE cases is about \(25\,\%\) of
the maximum, while in the intermediate months, the number of occurrences
gradually ramps between the maximum and minimum. This seasonality is a
combined effect of different factors, among them the types of clouds,
their frequency of occurrence the seasonally varying relative position
of the sun, as well as the local landscape characteristics that may
influence the formation of clouds. Unfortunately, lack of detailed data
on cloud formation, type and location is not allowing further analysis.
The interannual variability of the monthly CE events is quite high as
manifested by the large monthly extremes, especially in the summer.

\begin{figure}

{\centering \includegraphics[width=1\linewidth]{../images/clim-CE-month-norm-MAX-median-N-MW-1} 

}

\caption{Seasonal variability of the number of CE events normalized to the maximum occurring in June, in the form of a box and whisker plot. The monthly values have been also normalized to the relative abundance of valid GHI observations. The box contains the data between the lower $25\,\%$ and the upper $75\,\%$ percentiles,The thick horizontal line and the diamond symbol represent the median and the mean values, respectively. The vertical lines (whiskers) extend between the maximum and minimum monthly values and the solid circles are outliers.}\label{fig:relative-month-occurrences}
\end{figure}

The distribution of the number of CE events as a function of OIR is
shown in Figure\nobreakspace{}\ref{fig:ovir-distribution}. Apparently,
there is an inverse relation between the frequency of CE events and OIR
with an exponential-like decline. This is expected, as the stronger the
CE events become, the rarer are the conditions favoring the occurrence
of CE events. For the majority (over \(60\%\)) of the CE events the OIR
is below the long term average of \(40\,\text{W}/\text{m}^2\), while
about \(10\%\) of the events correspond to OIR larger than
\(100\,\text{W}/\text{m}^2\) and up to highest value of 300
W/m\textsuperscript{2} . particular atmospheric and sun conditions can
occur. This fact, is indicative to the magnitude and the probability of
the expected OIRs events over Thessaloniki. Similar distribution of CE
occurrences have been reported by \citet{Vamvakas2020}, where the
magnitude of OIR were higher due to the location of the city of Patras,
\(2.5^\circ\) closer to the equator.

\begin{figure}

{\centering \includegraphics[width=1\linewidth]{../images/P-relative-distribution-diff-1} 

}

\caption{Distribution of CE over irradiance magnitude.}\label{fig:ovir-distribution}
\end{figure}

\hypertarget{groups-of-cloud-enhancement}{%
\subsection{Groups of cloud
enhancement}\label{groups-of-cloud-enhancement}}

In order to study the total duration of the CE events, we grouped the
one minute CE events to continuous CE groups (CEG). Thus, a CEG consists
of one or more successive CE cases, and can represent cloud enhancement
conditions spanning a duration of multiple minutes. We have identified
28468 groups of CE in the whole period of study, where the group with
the longest duration lasted 140 minutes on 07~July 2013. By examining
the frequency distribution of the CEG durations
(Figure\nobreakspace{}\ref{fig:ceg-duration-distribution}), we can
conclude, that although we detected some CEG with durations longer than
an hour, about \(80\,\%\) of the CEG have a duration of less than 5
minutes.

\begin{figure}

{\centering \includegraphics[width=1\linewidth]{../images/groups-1} 

}

\caption{Distribution of CE groups duration in minutes.}\label{fig:ceg-duration-distribution}
\end{figure}

The relation between the duration and the mean OIR intensity of the
groups have also been studied (Figure\nobreakspace{}\ref{fig:group-2d}).
We observed that the GCE events tend to have either long duration, or
large intensity. Events with strong enhancement and long duration are
very rare. Similar results on this relation, have been reported by
\citet{Zhang2018}, on a study using a far higher sampling rate than
ours.

\begin{figure}

{\centering \includegraphics[width=1\linewidth]{../images/P-groups-bin2d-1} 

}

\caption{Relation of the mean over irradiance and duration of GCE, where the color scale denotes the frequency of the respected events.}\label{fig:group-2d}
\end{figure}

\hypertarget{extreme-cloud-enhancement-events}{%
\subsection{Extreme cloud enhancement
events}\label{extreme-cloud-enhancement-events}}

An aspect of the CE events that is commonly reported and has some
significance on the solar energy production infrastructure are the
extreme CE events (ECE). Where solar irradiance exceeds the expected
irradiance on top of the atmosphere
(Equation\nobreakspace{}\ref{eq:ECE}). Analogous to
Figure\nobreakspace{}\ref{fig:relative-month-occurancies} we have
computed the distribution of the number of occurrences of ECE events by
month in Figure\nobreakspace{}\ref{fig:relative-month-occurancies-ECE}.
The most active period for ECE events is in the spring and the start of
the summer (March -- June), followed by a period in the late fall
(September and October). This is probably related to the weather
characteristics of these periods, where there are continuous
alternations between clear sky periods and clouds.

\begin{figure}

{\centering \includegraphics[width=1\linewidth]{../images/clim-ECE-month-norm-MAX-median-N-2} 

}

\caption{Seasonal statistics of the number of ECE occurancies for each month, normalized to the maximum occurancies on March. The box represents the values of the low $25\,\%$ percentile to $75\,\%$ percentile, where the thick horizontal line inside is the mean, the vertical lines extend to the max imum and minimum vales, the dots are outlier values, and the rhombus is the mean.}\label{fig:relative-month-occurancies-ECE}
\end{figure}

The distribution of the ECE events
(Figure\nobreakspace{}\ref{fig:P-extreme-distribution}), shows that
there are rare cases where the OIR can exceed the TSI even more than
\(400\,\text{W}/\text{m}^2\), with the \(75\,\%\) of the cases to be
below \(200\,\text{W}/\text{m}^2\). Those finds are in accordance with
results form \citet{Vamvakas2020}, with the difference that the OIR
values are lower for Thessaloniki.

\begin{figure}

{\centering \includegraphics[width=1\linewidth]{../images/P-extreme-distribution-1} 

}

\caption{Distribution of ECE above clear sky threshold, for cases that are exceeding the TSI.}\label{fig:P-extreme-distribution}
\end{figure}

\hypertarget{discussion-and-conclusions}{%
\section{Discussion and conclusions}\label{discussion-and-conclusions}}

By creating a clear sky approximation of the GHI, which represents the
long- and short-term variation of the expected clear sky GHI, we were
able to identify cases of CE events. After analyzing the CE cases, we
found an increase of \(45.6\pm 21.9\,\text{cases}/\text{year}\), with
the mean annual total energy of the CE events increasing with a rate of
\(116.9\pm 67.8\,\text{kj}/\text{year}\). The most active season of CE
events over Thessaloniki is concentrated on early summer, on May and
June.

The magnitude of the ECE events doesn't exceeds the values reported from
sites with more favourable conditions for the phenomenon. Although, the
climatological characteristic of the ECE events, showed that the most
active months are spread on half of the year (March -- June and
September and October). We found that CE conditions, can have a duration
of more than an hour in rare cases, with the bulk of the cases having a
duration under 5 minutes. Some of the characteristics of CE and ECE
events we analysed, have strong similarities with results by
\citet{Vamvakas2020}, for a city southern of Thessaloniki, with the
analogues differences on the intensity of the solar radiation.

An interpretation of the CE trends, shows that the interaction of GHI
with the clouds, through this 30 year period, is a dynamic phenomenon
that's needs further investigation. Although, to approach it, we need
more long term observations of AOD and clouds characteristics.

\bibliography{bibliography.bib}


\end{document}
