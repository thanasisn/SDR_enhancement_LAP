% Options for packages loaded elsewhere
\PassOptionsToPackage{unicode}{hyperref}
\PassOptionsToPackage{hyphens}{url}
%
\documentclass[
]{article}
\usepackage{amsmath,amssymb}
\usepackage{iftex}
\ifPDFTeX
  \usepackage[T1]{fontenc}
  \usepackage[utf8]{inputenc}
  \usepackage{textcomp} % provide euro and other symbols
\else % if luatex or xetex
  \usepackage{unicode-math} % this also loads fontspec
  \defaultfontfeatures{Scale=MatchLowercase}
  \defaultfontfeatures[\rmfamily]{Ligatures=TeX,Scale=1}
\fi
\usepackage{lmodern}
\ifPDFTeX\else
  % xetex/luatex font selection
\fi
% Use upquote if available, for straight quotes in verbatim environments
\IfFileExists{upquote.sty}{\usepackage{upquote}}{}
\IfFileExists{microtype.sty}{% use microtype if available
  \usepackage[]{microtype}
  \UseMicrotypeSet[protrusion]{basicmath} % disable protrusion for tt fonts
}{}
\makeatletter
\@ifundefined{KOMAClassName}{% if non-KOMA class
  \IfFileExists{parskip.sty}{%
    \usepackage{parskip}
  }{% else
    \setlength{\parindent}{0pt}
    \setlength{\parskip}{6pt plus 2pt minus 1pt}}
}{% if KOMA class
  \KOMAoptions{parskip=half}}
\makeatother
\usepackage{xcolor}
\usepackage[margin=1in]{geometry}
\usepackage{longtable,booktabs,array}
\usepackage{calc} % for calculating minipage widths
% Correct order of tables after \paragraph or \subparagraph
\usepackage{etoolbox}
\makeatletter
\patchcmd\longtable{\par}{\if@noskipsec\mbox{}\fi\par}{}{}
\makeatother
% Allow footnotes in longtable head/foot
\IfFileExists{footnotehyper.sty}{\usepackage{footnotehyper}}{\usepackage{footnote}}
\makesavenoteenv{longtable}
\usepackage{graphicx}
\makeatletter
\def\maxwidth{\ifdim\Gin@nat@width>\linewidth\linewidth\else\Gin@nat@width\fi}
\def\maxheight{\ifdim\Gin@nat@height>\textheight\textheight\else\Gin@nat@height\fi}
\makeatother
% Scale images if necessary, so that they will not overflow the page
% margins by default, and it is still possible to overwrite the defaults
% using explicit options in \includegraphics[width, height, ...]{}
\setkeys{Gin}{width=\maxwidth,height=\maxheight,keepaspectratio}
% Set default figure placement to htbp
\makeatletter
\def\fps@figure{htbp}
\makeatother
\setlength{\emergencystretch}{3em} % prevent overfull lines
\providecommand{\tightlist}{%
  \setlength{\itemsep}{0pt}\setlength{\parskip}{0pt}}
\setcounter{secnumdepth}{5}
\newlength{\cslhangindent}
\setlength{\cslhangindent}{1.5em}
\newlength{\csllabelwidth}
\setlength{\csllabelwidth}{3em}
\newlength{\cslentryspacingunit} % times entry-spacing
\setlength{\cslentryspacingunit}{\parskip}
\newenvironment{CSLReferences}[2] % #1 hanging-ident, #2 entry spacing
 {% don't indent paragraphs
  \setlength{\parindent}{0pt}
  % turn on hanging indent if param 1 is 1
  \ifodd #1
  \let\oldpar\par
  \def\par{\hangindent=\cslhangindent\oldpar}
  \fi
  % set entry spacing
  \setlength{\parskip}{#2\cslentryspacingunit}
 }%
 {}
\usepackage{calc}
\newcommand{\CSLBlock}[1]{#1\hfill\break}
\newcommand{\CSLLeftMargin}[1]{\parbox[t]{\csllabelwidth}{#1}}
\newcommand{\CSLRightInline}[1]{\parbox[t]{\linewidth - \csllabelwidth}{#1}\break}
\newcommand{\CSLIndent}[1]{\hspace{\cslhangindent}#1}
\usepackage{caption}
\usepackage{placeins}
\captionsetup{font=small}
\usepackage{booktabs}
\usepackage{longtable}
\usepackage{array}
\usepackage{multirow}
\usepackage{wrapfig}
\usepackage{float}
\usepackage{colortbl}
\usepackage{pdflscape}
\usepackage{tabu}
\usepackage{threeparttable}
\usepackage{threeparttablex}
\usepackage[normalem]{ulem}
\usepackage{makecell}
\usepackage{xcolor}
\ifLuaTeX
  \usepackage{selnolig}  % disable illegal ligatures
\fi
\IfFileExists{bookmark.sty}{\usepackage{bookmark}}{\usepackage{hyperref}}
\IfFileExists{xurl.sty}{\usepackage{xurl}}{} % add URL line breaks if available
\urlstyle{same}
\hypersetup{
  pdftitle={SDR enhancement by clouds},
  pdfauthor={Athanasios N. Natsis; Alkiviadis Bais; Charikleia Meleti},
  pdfkeywords={keyword1, keyword2},
  hidelinks,
  pdfcreator={LaTeX via pandoc}}

\title{SDR enhancement by clouds}
\usepackage{etoolbox}
\makeatletter
\providecommand{\subtitle}[1]{% add subtitle to \maketitle
  \apptocmd{\@title}{\par {\large #1 \par}}{}{}
}
\makeatother
\subtitle{A Short Subtitle}
\author{Athanasios N. Natsis\footnote{Laboratory of Atmospheric Physics \emph{\href{mailto:natsisphysicist@gmail.com}{\nolinkurl{natsisphysicist@gmail.com}}}} \and Alkiviadis Bais\footnote{Laboratory of Atmospheric Physics \emph{\href{mailto:abais@auth.gr}{\nolinkurl{abais@auth.gr}}}} \and Charikleia Meleti\footnote{Laboratory of Atmospheric Physics \emph{\href{mailto:meleti@auth.gr}{\nolinkurl{meleti@auth.gr}}}}}
\date{2024-03-20}

\begin{document}
\maketitle

{
\setcounter{tocdepth}{4}
\tableofcontents
}
\hypertarget{abstract}{%
\section*{Abstract}\label{abstract}}
\addcontentsline{toc}{section}{Abstract}

\begin{itemize}
\tightlist
\item
  Phenomenon
\item
  Effects
\item
  bibliography
\item
  our approach
\item
  our results
\end{itemize}

CE: cloud enhancement cases one minute

ECE: extreme cloud enhancement cases one minute over TSI on horizontal plane

CEG: cloud enhancement groups, cases events with consecutive CE

\(\text{GHI}_\text{i}\): measured one minute global horizontal irradiance

\(\text{GHI}_\text{mCSi}\): modeled clear sky one minute global horizontal irradiance

\hypertarget{intro}{%
\section{Intro}\label{intro}}

CE and ECE usefulness
importance
problems

reported trends

CE mechanism

In this study, we define as cloud enhancement events (CE) the cases when the measured
global horizontal irradiance (GHI) at ground level, exceeds the expected value under
clear-sky conditions.
Similar, we define the extreme cloud enhancement events (ECE), the cases when GHI,
exceeds the Total Solar Irradiance on horizontal plane at ground level.
Although the duration of these bursts varies from instantaneous to several minutes,
here we are constrained by the recorded data, to one-minute steps.

Identify CE cases

Characteristics of CE

Climatology and impact on the GHI

Effects of clouds

\begin{itemize}
\tightlist
\item
  enhancement physics
\item
  conditions
\item
  other interferences
\end{itemize}

Identification methods from bib

\begin{itemize}
\tightlist
\item
  clearness index
\item
  modeled approaches
\item
  algorithmic id
\item
  AI
\item
  Visual
\end{itemize}

Constrains of the site

\begin{itemize}
\tightlist
\item
  data
\item
  instrument
\item
  location
\end{itemize}

\hypertarget{data-and-methodology}{%
\section{Data and methodology}\label{data-and-methodology}}

\hypertarget{ghi-data}{%
\subsection{GHI data}\label{ghi-data}}

Our monitoring site is operating in the Laboratory of Atmospheric Physics of the
Aristotle University of Thessaloniki (\(40^\circ\,38'\,\)N,
\(22^\circ\,57'\,\)E, \(80\,\)m~a.s.l.).
In this study we present data from the period
13~April 1993 to
31~December 2023.

The GHI data were measured with a Kipp~\& Zonen CM-21 pyranometer.
During the study period, the pyranometer was independently calibrated three times at
the Meteorologisches Observatorium Lindenberg, DWD, verifying that the stability of
the instrument's sensitivity was better than \(0.7\%\) relative to the initial
calibration by the manufacturer.

For the acquisition of radiometric data, the signal of the pyranometer was sampled
at a rate of \(1\,\text{Hz}\).
The mean and the standard deviation of these samples were calculated and recorded
every minute.
The measurements were corrected for the zero offset (``dark signal'' in volts), which
was calculated by averaging all measurements recorded for a period of \(3\,\text{h}\),
before (morning) or after (evening) the Sun reaches an elevation angle of \(-10^\circ\).
The signal was converted to irradiance using a ramped value of the instrument's
sensitivity between subsequent calibrations.

To further improve the quality of the irradiance data, a manual screening was
performed to remove inconsistent and erroneous recordings that can occur
stochastically or systematically during the continuous operation of the instruments.
The manual screening was aided by a radiation data quality assurance procedure,
adjusted for the site, which was based on the methods of Long and Shi~(Long and Shi 2006, 2008).
Thus, problematic recordings have been excluded from further processing.
Although it is impossible to detect all false data, the large number of available
data, and the aggregation scheme we used, ensures the quality of the radiation
measurements used in this study.

To preserve an unbiased representation of the data we applied a constraint similar
the one used by Castillejo-Cuberos and Escobar (2020). Where, for each valid hour of day, there
must exist at list 45 minutes of valid measurements, including nighttime near
sunrise and sunset. Days with less than 5 valid hours are rejected completely.
Furthermore, due to the significant measurement uncertainty when the Sun is near the
horizon, and to some systematic obstructions by nearby buildings, we have excluded
all measurements with solar zenith angle (SZA) greater than \(78^\circ\).

In this study, we evaluated the effects of the clouds on the total downward radiation
by studding the occurrences of the cloud enhancement cases, their intensity, and their
duration. The recording of the radiation signal is the mean of one-minute. Thus, we
will refer to this time step as a cloud enhancement event (CE).

\hypertarget{clear-sky-irradiance}{%
\subsection{Clear Sky Irradiance}\label{clear-sky-irradiance}}

In order to identify CE cases, we need an approximation of the expected clear sky GHI
for the whole period of the study, with a temporal resolution of one minute. Because
of the lack of exact data about atmospheric factors, responsible for the attenuation
of the solar radiation in the atmosphere, we can not recreate a minute by minute
representation of the reference radiation. Thus, we used some climatological data
with a radiation transfer model.

We decided to use the Libradtran (Emde et al. 2016) radiation transfer model, to model
the clear sky equivalent of GHI, with monthly climatological data of AOD and water
column from the Aerosol Robotic Network (AERONET) (Giles et al. 2019; Buis et al. 1998). Our site
participates in AERONET, where we operate a Cimel photometer, collocated to
the CM-21 pyranometer, since 2003

For the production of \ldots..

We also used the Sun's total solar irradiance from NOAA (Coddington et al. 2005) and the
Earth - Sun distance calculated by Astropy (Astropy Collaboration et al. 2022). We simulated
the clear sky radiation, using the solar radiation spectrum of Kurucz (1994) in the
range \(280\) to \(2500\,\text{nm}\), with a SZA step of \(0.2^\circ\). For the
atmospheric characteristics, we used combinations of AOD at \(500\,\text{nm}\)
(\(t_{500\text{nm}}\)) and additional offsets of \(\pm1\) and \(\pm2\sigma\), water column
(\(w_h\)) with offsets of \(\pm1\) and \(\pm2\sigma\), and applied it to two atmospheric
profiles ``AFGL atmospheric constituent profile, midlatitude summer'' (afglms) and
``AFGL atmospheric constituent profile, midlatitude winter'' (afglmw) (Anderson et al. 1986).
Thus, we were able to emulate different atmospheric condition and level of
atmospheric clearness for the climatological conditions of the site.

Furthermore, to account for the Sun's variability in our one-minute GHI measurements,
we adjusted the model's input spectrum integral to the one of the TSI provided by
NOAA. Finally, we adjusted for the effect of the Sun - Earth distance, and
calculated the clear sky irradiance value at the horizontal plane by using the cosine
of the corresponding interpolated SZA. For each period of the year, we used the
appropriate atmospheric profile (afglms or afglmw). Thus, the modeled GHI\_modeled we
can directly compare each measured one minute value of GHI, with the expected clear
sky radiation.

\begin{itemize}
\tightlist
\item
  rte\_solver disort
\item
  geometry pseudospherical
\item
  mol\_abs LOWTRAN
\end{itemize}

Of course, have to choose which atmospheric conditions qualify as a good clear sky analogous radiation.

After some evaluation of the modeled radiation in different atmospheric conditions,
in relation to the GHI data, we choose as a representative of the clear sky radiation,
the case with
\(t_{\text{cs}} = t_{500\text{nm}} - 1\sigma\) and \(w_{h\text{cs}} = w_h - 1\sigma\).

These values will represent a typical atmosphere in
Thessaloniki with low load of aerosols and humidity, which are the main factor that
attenuate the GHI, excluding clouds.
We will refer to this irradiance as \(\text{GHI}_\text{mCSi}\).

\hypertarget{thresholds-investigation}{%
\subsection{Thresholds investigation}\label{thresholds-investigation}}

The use of the actual modeled values of clear sky GHI can not provide us with a
robust method to distinguish the CE cases, due to the limited accuracy of the input
data. Our main focus was to positively identify over irradiance events from CEs,
thus we used a relative factor, to create an upper envelope of the clear sky
irradiance, above which, any GHI value can safely attributed to CE.

To select the exact values of the thresholds (Equation \ref{eq:CE4}), we implemented
manual inspection of the CE identification, on specially selected days from the whole
dataset. We used seven sets of selected days, with characteristics relevant to the
efficiency of the identification threshold. These sets were random groups of about
20 to 30 days with the following characteristics:
(a) the largest over irradiance CE events,
(b) the largest daily total over irradiance,
(c) without clouds (by implementing a clear sky identification algorithm as discussed in Natsis, Bais, and Meleti (2023)),
(d) without clouds and without EC events,
(e) with at least \(60\%\) of the day length without clouds and some EC events,
(h) random days and
(i) some manual selected days that were included during the manual inspection.

Where it was needed, for some of the edge cases, we also used images from a sky-cam,
along with the manual inspection.

Reasoning for the final selection\ldots\ldots\ldots..

\hypertarget{criteria-for-enhancement-identification}{%
\subsection{Criteria for enhancement identification}\label{criteria-for-enhancement-identification}}

Many have used models or some statistical method to determine then CE thresholds.

\ldots\ldots. cite \ldots\ldots{} Descriptions

We used as reference a modeled clear sky approximation,

Criteria for cloud enhancement identification are set according the Equation\nobreakspace\ref{eq:CE4}

\begin{equation}
\text{CE} : \begin{cases}
 \text{GHI}_\text{i} > 1.05 \cdot \text{GHI}_\text{mCSi}, & \text{$\theta \leq 60^\circ$}\\
\text{GHI}_\text{i} > \left ({ 1.18 + \frac{1.05 - 1.18}{60 - 78} \cdot (\theta- 78) } \right ) \cdot \text{GHI}_\text{mCSi}, & \text{$ 78^\circ > \theta > 60^\circ$}\\
\text{Excluded measurements}, & \theta > 78^\circ
\end{cases}\label{eq:CE4b}
\end{equation}
where: \(\theta\) solar zenith angle.

A secondary criterion was used for extreme cloud enhancement cases (ECE)
identification, relative to the TSI at horizontal plane
(Equation\nobreakspace{}\ref{eq:ECE})

\begin{equation}
\text{ECE}: \text{GHI} > \cos(\theta) \times \text{TSI}_\text{TOA}
\label{eq:ECE}
\end{equation}

Include Example of days plot? Appendix?

\ldots{} this is more clear to understand and describe \ldots.

\hypertarget{other-method-used-and-rejected}{%
\subsection{Other method used and rejected}\label{other-method-used-and-rejected}}

\begin{itemize}
\tightlist
\item
  some other criteria were tested
\end{itemize}

\FloatBarrier

\hypertarget{results}{%
\section{Results}\label{results}}

Our dataset, after the data selection processing, consists of
6144534 records of GHI, of which
\(1.799\,\%\) are CE and
\(0.036\,\%\) are ECE events.
The highest GHI recorder was
\(1416.6\,W/m^2\)
on 24~May 2007.
The absolute stronger CE event had an over irradiance of
\(341.79\,W/m^2\) on
15~May 2014.
The relative stronger CE event was
\(54\,\%\) above the
clear sky threshold, on
28~October 2016.

\hypertarget{trends}{%
\subsection{Trends}\label{trends}}

We computed the daily trend of the mean over irradiance of CE
(Figure\nobreakspace{}\ref{fig:CEmeanDaily}), using a first-order autoregressive
model with lag of 1 day, using the `maximum likelihood' fitting method (Gardner, Harvey, and Phillips 1980; Jones 1980) by implementing the function `arima' from the library `stats' of the R
programming language (R Core Team 2023). The trends were reported together with the \(2\sigma\)
error. We observe an increase of
\(0.216\pm 0.083\,W/m^2\)
on the mean over irradiance.

\begin{figure}[h!]

{\centering \includegraphics[width=0.6\linewidth]{../images/P_daily_trend-1} 

}

\caption{Daily mean values and trend of the CE over irradiance.}\label{fig:CEmeanDaily}
\end{figure}

Although, the previous result is closer to the raw data, we preferred to present the
annual statistics, that give a more clear picture about the long term CE trends.
Hence, we calculated the annual values from the one-minute measurements.
The annual mean over irradiance is
\(0.23\pm 0.11\,W/m^2\)
(Figure\nobreakspace{}\ref{fig:P-energy-mean}),
but this value is not very useful due the high
variability of this metric.

\begin{figure}[h!]

{\centering \includegraphics[width=0.6\linewidth]{../images/P_energy-7} 

}

\caption{Trends of the mean over irradiance per CE.}\label{fig:P-energy-mean}
\end{figure}

A better indicator of changes on the characteristic of CE would be the number of CE
occurrences and the total energy of the CE over irradiance. The annual number of CE
occurrences, shows a steady increase of
\(122.2\pm 22.8\,/y\)
(Figure\nobreakspace{}\ref{fig:P-energy-N}).
We have to note that the energy related to the CE events can not be directly linked
with the total energy balance on the atmosphere. The net sun radiation of the region
is not increased, but rather redistributed through the CE.

\begin{figure}[h!]

{\centering \includegraphics[width=0.6\linewidth]{../images/P_energy-6} 

}

\caption{Trend of yearly CE number of occurancies.}\label{fig:P-energy-N}
\end{figure}

Subsequently, the annual energy observed during CE events shows a annual increase of
\(341.6\pm 73.1\,kJ/y\)
(Figure~\ref{fig:P-energy-sum}), witch follows the trend of the number of
occurrences.

\begin{figure}[h!]

{\centering \includegraphics[width=0.6\linewidth]{../images/P_energy-5} 

}

\caption{Trend of the yearly excess energy due to CE over irradiance}\label{fig:P-energy-sum}
\end{figure}

\begin{figure}[h!]

{\centering \includegraphics[width=0.6\linewidth]{../images/P_energy-8} 

}

\caption{Trend of the yearly median over irradiance due to CE over irradiance}\label{fig:P-energy-median}
\end{figure}

\FloatBarrier

\hypertarget{climatology}{%
\subsection{Climatology}\label{climatology}}

Another interesting aspect of the CE cases, is the seasonal cycle. In
Figure\nobreakspace{}\ref{fig:relative-month-occurancies}, we have the box plot
(whisker plot), where the values have been normalized by the highest median value,
that occurs in May. Although the number of occurrences has a wide spread throughout
the study period, the most active period of CE occurrences is during May and June.
During the Winter (December -- February) the CE cases are about 25\% of the maximum.
The rest of the months the occurrences seems to ramp between the maximum and minimum.

\begin{figure}[h!]

{\centering \includegraphics[width=0.6\linewidth]{../images/clim_CE_month_norm_MAX_median_N-2} 

}

\caption{Statistics of the number of CE occurancies for each month. The box represents the values of the low 25\% percentile to 75\% percentile, where the thick horizontal line inside is the mean, the verical lines extend to the macimum and minimum vales, the dots are outlier values, and the rhombus is the mean.}\label{fig:relative-month-occurancies}
\end{figure}

The distribution of the CE over irradiance spreads uniformly
Figure\nobreakspace{}\ref{fig:ovir-distribution}. Where there is an inverse relation
between the events frequency and events intensity. This is expected as, the stronger
the CE events become, the more rare are the particular atmospheric and sun conditions
to occur.

\begin{figure}[h!]

{\centering \includegraphics[width=0.6\linewidth]{../images/P-relative-distribution-diff-1} 

}

\caption{Distribution of CE over irradiance}\label{fig:ovir-distribution}
\end{figure}

\FloatBarrier

\hypertarget{groups-stats}{%
\subsection{Groups stats}\label{groups-stats}}

In order to further study the characteristics of the CE events, we grouped the singe
minute CE events, to continuous CE groups (CEG). Thus, a CEG consists of one or more
successive CE cases, and can have a duration of multiple of minutes. We have
identified 29009 groups, where the group with the longest duration lasted
140 minutes on
07~July 2013.
By examining the frequency distribution of the CEG durations
(Figure\nobreakspace{}\ref{fig:ceg-duration-distribution}), we can conclude that longer durations are getting increasingly rare, with durations above 10 minutes very rare.

\begin{figure}[h!]

{\centering \includegraphics[width=0.6\linewidth]{../images/groups-1} 

}

\caption{Distribution of CE groups}\label{fig:ceg-duration-distribution}
\end{figure}

\ldots{} (\textbf{Zhang2018?}) \ldots.

\begin{figure}[h!]

{\centering \includegraphics[width=0.6\linewidth]{../images/P-groups-bin2d-1} 

}

\caption{Relation of mean over irradiance and CE group duration}\label{fig:unnamed-chunk-3}
\end{figure}

\FloatBarrier

\hypertarget{extreme-ce-above-tsi}{%
\subsection{Extreme CE above TSI}\label{extreme-ce-above-tsi}}

Another aspect of the CE events are the cases when the irradiance is above the
expected irradiance on top of the atmosphere, we have defined this as ECE
(Equation\nobreakspace{}\ref{eq:ECE}).

\begin{itemize}
\tightlist
\item
  max value ECE
  relative to other works max values
\end{itemize}

Analogous to Figure\nobreakspace{}\ref{fig:relative-month-occurancies} we have
computed the distribution of the number of occurrences of ECE events by month in
Figure\nobreakspace{}\ref{fig:relative-month-occurancies-ECE}. Here the most active
period is in the spring (March -- May), followed by the months of June, September and
October.

\ldots\ldots.

\begin{figure}[h!]

{\centering \includegraphics[width=0.6\linewidth]{../images/clim_ECE_month_norm_MAX_median_N-2} 

}

\caption{Distribution of }\label{fig:relative-month-occurancies-ECE}
\end{figure}

\begin{figure}[h!]

{\centering \includegraphics[width=0.6\linewidth]{../images/extremedistributions-2} 

}

\caption{Distribution of ECE above 'clear sky' reference}\label{fig:unnamed-chunk-4}
\end{figure}

\begin{itemize}
\tightlist
\item
  SZA
\end{itemize}

\FloatBarrier

\hypertarget{discussion-and-conclusions}{%
\section{Discussion and conclusions}\label{discussion-and-conclusions}}

Compare to other location stats.
max, ECE, distributions \ldots.

Climatology results

\hypertarget{appendix}{%
\section*{Appendix}\label{appendix}}
\addcontentsline{toc}{section}{Appendix}

\hypertarget{references}{%
\section*{References}\label{references}}
\addcontentsline{toc}{section}{References}

\hypertarget{refs}{}
\begin{CSLReferences}{1}{0}
\leavevmode\vadjust pre{\hypertarget{ref-Anderson1986}{}}%
Anderson, G. P., J. H. Chetwynd, S. A. Clough, E. P. Shettle, and F. X. Kneizys. 1986. {``{AFGL} Atmospheric Constituent Profiles (0-120km).''} Air Force Geophysics Laboratory, Optical Physics Division.

\leavevmode\vadjust pre{\hypertarget{ref-AstropyCollaboration2022}{}}%
Astropy Collaboration, Adrian M. Price-Whelan, Pey Lian Lim, Nicholas Earl, Nathaniel Starkman, Larry Bradley, David L. Shupe, et al. 2022. {``{The Astropy Project: Sustaining and Growing a Community-oriented Open-source Project and the Latest Major Release (v5.0) of the Core Package}''} 935 (2): 167. \url{https://doi.org/10.3847/1538-4357/ac7c74}.

\leavevmode\vadjust pre{\hypertarget{ref-Buis1998}{}}%
Buis, J. P. P., A. Setzer, B. N. N. Holben, T. F. F. Eck, I. Slutsker, D. Tanre, E. Vermote, et al. 1998. {``AERONET---a Federated Instrument Network and Data Archive for Aerosol Characterization.''} \emph{Remote Sensing of Environment} 66 (1): 1--16.

\leavevmode\vadjust pre{\hypertarget{ref-CastillejoCuberos2020}{}}%
Castillejo-Cuberos, Armando, and Rodrigo Escobar. 2020. {``Detection and Characterization of Cloud Enhancement Events for Solar Irradiance Using a Model-Independent, Statistically-Driven Approach.''} \emph{Solar Energy} 209 (October): 547--67. \url{https://doi.org/10.1016/j.solener.2020.09.046}.

\leavevmode\vadjust pre{\hypertarget{ref-Coddington2005}{}}%
Coddington, Odele, Judith L. Lean, Doug Lindholm, Peter Pilewskie, Martin Snow, and NOAA CDR Program. 2005. {``{NOAA} Climate Data Record ({CDR}) of Total Solar Irradiance ({TSI}), {NRLTSI} Version 2. {D}aily.''} \url{https://doi.org/10.7289/V55B00C1}.

\leavevmode\vadjust pre{\hypertarget{ref-Emde2016}{}}%
Emde, Claudia, Robert Buras-Schnell, Arve Kylling, Bernhard Mayer, Josef Gasteiger, Ulrich Hamann, Jonas Kylling, et al. 2016. {``The {libRadtran} Software Package for Radiative Transfer Calculations (Version 2.0.1).''} \emph{Geoscientific Model Development} 9 (5): 1647--72. \url{https://doi.org/10.5194/gmd-9-1647-2016}.

\leavevmode\vadjust pre{\hypertarget{ref-Gardner1980}{}}%
Gardner, G., A. C. Harvey, and G. D. A. Phillips. 1980. {``Algorithm {AS} 154: An Algorithm for Exact Maximum Likelihood Estimation of Autoregressive-Moving Average Models by Means of Kalman Filtering.''} \emph{Applied Statistics} 29 (3): 311. \url{https://doi.org/10.2307/2346910}.

\leavevmode\vadjust pre{\hypertarget{ref-Giles2019}{}}%
Giles, David M., Alexander Sinyuk, Mikhail G. Sorokin, Joel S. Schafer, Alexander Smirnov, Ilya Slutsker, Thomas F. Eck, et al. 2019. {``Advancements in the Aerosol Robotic Network ({AERONET}) Version 3 Database -- Automated Near-Real-Time Quality Control Algorithm with Improved Cloud Screening for Sun Photometer Aerosol Optical Depth ({AOD}) Measurements.''} \emph{Atmospheric Measurement Techniques} 12 (1): 169--209. \url{https://doi.org/10.5194/amt-12-169-2019}.

\leavevmode\vadjust pre{\hypertarget{ref-Jones1980}{}}%
Jones, Richard H. 1980. {``Maximum Likelihood Fitting of {ARMA} Models to Time Series with Missing Observations.''} \emph{Technometrics} 22 (3): 389--95. \url{https://doi.org/10.1080/00401706.1980.10486171}.

\leavevmode\vadjust pre{\hypertarget{ref-Kurucz1994}{}}%
Kurucz, Robert L. 1994. {``Synthetic Infrared Spectra.''} In \emph{Infrared Solar Physics}, edited by D. M. Rabin, J. T. Jefferies, and C. Lindsey, 523--31. Dordrecht: Springer Netherlands.

\leavevmode\vadjust pre{\hypertarget{ref-Long2006}{}}%
Long, Charles N., and Y. Shi. 2006. {``The QCRad Value Added Product: Surface Radiation Measurement Quality Control Testing, Including Climatology Configurable Limits.''} DOE/SC-ARM/TR-074. Office of Science, Office of Biological; Environmental Research, U.S. Department of Energy.

\leavevmode\vadjust pre{\hypertarget{ref-Long2008a}{}}%
---------. 2008. {``An Automated Quality Assessment and Control Algorithm for Surface Radiation Measurements.''} \emph{The Open Atmospheric Science Journal}, 23--37.

\leavevmode\vadjust pre{\hypertarget{ref-Natsis2023}{}}%
Natsis, Athanasios, Alkiviadis Bais, and Charikleia Meleti. 2023. {``Trends from 30-Year Observations of Downward Solar Irradiance in Thessaloniki, Greece.''} \emph{Applied Sciences} 14 (1): 252. \url{https://doi.org/10.3390/app14010252}.

\leavevmode\vadjust pre{\hypertarget{ref-RCT2023}{}}%
R Core Team. 2023. \emph{R: A Language and Environment for Statistical Computing}. Vienna, Austria: R Foundation for Statistical Computing. \url{https://www.R-project.org/}.

\end{CSLReferences}

\end{document}
