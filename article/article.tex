% Options for packages loaded elsewhere
\PassOptionsToPackage{unicode}{hyperref}
\PassOptionsToPackage{hyphens}{url}
%
\documentclass[
]{article}
\usepackage{amsmath,amssymb}
\usepackage{iftex}
\ifPDFTeX
  \usepackage[T1]{fontenc}
  \usepackage[utf8]{inputenc}
  \usepackage{textcomp} % provide euro and other symbols
\else % if luatex or xetex
  \usepackage{unicode-math} % this also loads fontspec
  \defaultfontfeatures{Scale=MatchLowercase}
  \defaultfontfeatures[\rmfamily]{Ligatures=TeX,Scale=1}
\fi
\usepackage{lmodern}
\ifPDFTeX\else
  % xetex/luatex font selection
\fi
% Use upquote if available, for straight quotes in verbatim environments
\IfFileExists{upquote.sty}{\usepackage{upquote}}{}
\IfFileExists{microtype.sty}{% use microtype if available
  \usepackage[]{microtype}
  \UseMicrotypeSet[protrusion]{basicmath} % disable protrusion for tt fonts
}{}
\makeatletter
\@ifundefined{KOMAClassName}{% if non-KOMA class
  \IfFileExists{parskip.sty}{%
    \usepackage{parskip}
  }{% else
    \setlength{\parindent}{0pt}
    \setlength{\parskip}{6pt plus 2pt minus 1pt}}
}{% if KOMA class
  \KOMAoptions{parskip=half}}
\makeatother
\usepackage{xcolor}
\usepackage[margin=1in]{geometry}
\usepackage{longtable,booktabs,array}
\usepackage{calc} % for calculating minipage widths
% Correct order of tables after \paragraph or \subparagraph
\usepackage{etoolbox}
\makeatletter
\patchcmd\longtable{\par}{\if@noskipsec\mbox{}\fi\par}{}{}
\makeatother
% Allow footnotes in longtable head/foot
\IfFileExists{footnotehyper.sty}{\usepackage{footnotehyper}}{\usepackage{footnote}}
\makesavenoteenv{longtable}
\usepackage{graphicx}
\makeatletter
\def\maxwidth{\ifdim\Gin@nat@width>\linewidth\linewidth\else\Gin@nat@width\fi}
\def\maxheight{\ifdim\Gin@nat@height>\textheight\textheight\else\Gin@nat@height\fi}
\makeatother
% Scale images if necessary, so that they will not overflow the page
% margins by default, and it is still possible to overwrite the defaults
% using explicit options in \includegraphics[width, height, ...]{}
\setkeys{Gin}{width=\maxwidth,height=\maxheight,keepaspectratio}
% Set default figure placement to htbp
\makeatletter
\def\fps@figure{htbp}
\makeatother
\setlength{\emergencystretch}{3em} % prevent overfull lines
\providecommand{\tightlist}{%
  \setlength{\itemsep}{0pt}\setlength{\parskip}{0pt}}
\setcounter{secnumdepth}{5}
\newlength{\cslhangindent}
\setlength{\cslhangindent}{1.5em}
\newlength{\csllabelwidth}
\setlength{\csllabelwidth}{3em}
\newlength{\cslentryspacingunit} % times entry-spacing
\setlength{\cslentryspacingunit}{\parskip}
\newenvironment{CSLReferences}[2] % #1 hanging-ident, #2 entry spacing
 {% don't indent paragraphs
  \setlength{\parindent}{0pt}
  % turn on hanging indent if param 1 is 1
  \ifodd #1
  \let\oldpar\par
  \def\par{\hangindent=\cslhangindent\oldpar}
  \fi
  % set entry spacing
  \setlength{\parskip}{#2\cslentryspacingunit}
 }%
 {}
\usepackage{calc}
\newcommand{\CSLBlock}[1]{#1\hfill\break}
\newcommand{\CSLLeftMargin}[1]{\parbox[t]{\csllabelwidth}{#1}}
\newcommand{\CSLRightInline}[1]{\parbox[t]{\linewidth - \csllabelwidth}{#1}\break}
\newcommand{\CSLIndent}[1]{\hspace{\cslhangindent}#1}
\usepackage{caption}
\usepackage{placeins}
\captionsetup{font=small}
\usepackage{booktabs}
\usepackage{longtable}
\usepackage{array}
\usepackage{multirow}
\usepackage{wrapfig}
\usepackage{float}
\usepackage{colortbl}
\usepackage{pdflscape}
\usepackage{tabu}
\usepackage{threeparttable}
\usepackage{threeparttablex}
\usepackage[normalem]{ulem}
\usepackage{makecell}
\usepackage{xcolor}
\ifLuaTeX
  \usepackage{selnolig}  % disable illegal ligatures
\fi
\IfFileExists{bookmark.sty}{\usepackage{bookmark}}{\usepackage{hyperref}}
\IfFileExists{xurl.sty}{\usepackage{xurl}}{} % add URL line breaks if available
\urlstyle{same}
\hypersetup{
  pdftitle={SDR enhancement by clouds},
  pdfauthor={Athanasios N. Natsis; Alkiviadis Bais; Charikleia Meleti},
  pdfkeywords={keyword1, keyword2},
  hidelinks,
  pdfcreator={LaTeX via pandoc}}

\title{SDR enhancement by clouds}
\usepackage{etoolbox}
\makeatletter
\providecommand{\subtitle}[1]{% add subtitle to \maketitle
  \apptocmd{\@title}{\par {\large #1 \par}}{}{}
}
\makeatother
\subtitle{A Short Subtitle}
\author{Athanasios N. Natsis\footnote{Laboratory of Atmospheric Physics \emph{\href{mailto:natsisphysicist@gmail.com}{\nolinkurl{natsisphysicist@gmail.com}}}} \and Alkiviadis Bais\footnote{Laboratory of Atmospheric Physics \emph{\href{mailto:abais@auth.gr}{\nolinkurl{abais@auth.gr}}}} \and Charikleia Meleti\footnote{Laboratory of Atmospheric Physics \emph{\href{mailto:meleti@auth.gr}{\nolinkurl{meleti@auth.gr}}}}}
\date{2024-04-06}

\begin{document}
\maketitle

{
\setcounter{tocdepth}{4}
\tableofcontents
}
\hypertarget{abstract}{%
\section*{Abstract}\label{abstract}}
\addcontentsline{toc}{section}{Abstract}

\begin{itemize}
\tightlist
\item
  Phenomenon
\item
  Effects
\item
  bibliography
\item
  our approach
\item
  our results
\end{itemize}

CE: cloud enhancement cases one minute

ECE: extreme cloud enhancement cases one minute over TSI on horizontal plane

CEG: cloud enhancement groups, cases events with consecutive CE

\(\text{GHI}_\text{i}\): measured one minute global horizontal irradiance

\(\text{GHI}_\text{mCSi}\): modeled clear sky one minute global horizontal irradiance

\hypertarget{intro}{%
\section{Intro}\label{intro}}

The shortwave solar irradiance, reaching Earth's surface, is the main energy source
of the atmosphere and biosphere and drives and governs the climate (Gray et al. 2010). It
has direct practical application, in different industries, like energy production and
agriculture method. The variability of its intensity, can cause difficulties in
predicting the yield, and designing the specifications of the appropriate equipment.
A lot of research has been focused on predicting the renewable energy production in a
fine timescale and in near real-time (for a review see Inman, Pedro, and Coimbra (2013); Graabak and Korpås (2016)).

A big aspect of this variability is the interaction with the clouds. In general,
clouds absorb part of the solar irradiance, but under certain conditions, can enhance
the total shortwave irradiance reaching the ground. This effect, can locally increase
the observed total shortwave irradiance higher than the expected clear sky irradiance
{[}see references therein{]}.

Some of the proposed underling mechanisms of those events, have been summarized by
Gueymard (2017), and include scattering on the edge of cumulus clouds or through thin
cirrus. Further investigation with radiation transfer modeling methods and
observations have pointed as the prevailing mechanism, the forward Mie scattering
(Pecenak et al. 2016; Thuillier et al. 2013; G. H. Yordanov et al. 2013; Georgi Hristov Yordanov, Saetre, and Midtgård 2015), through the clouds.
The overall phenomenon depends on different interactive factors, that include cloud
thickness, constitution and type; and the relative position of the sun, the clouds,
and the observer (Gueymard 2017; Veerman, Van Stratum, and Van Heerwaarden 2022). As such, there are multiple
contributing mechanism that are responsible for the observed irradiance enhancements.

Cloud enhancements have been reported, to be able to exceed in intensity even the
value of the solar constant, resulting of clear indexes above unit. A summary of
extreme enchantment cases have been compiled by Martins, Mantelli, and Rüther (2022). There are also some
practical implication of the cloud enchantments. The intensity and duration of
enhancements can effect the efficiency and stability of photovoltaic power production
(Lappalainen and Kleissl 2020; Järvelä and Valkealahti 2020), and extreme enhancements cases, have the
potential to compromise the integrity of photovoltaic plants infrastructure
(Do Nascimento et al. 2019). It has also been demonstrated, that these events can interfere
in the comparison of ground data and satellite observations (Damiani et al. 2018)

Methods of identification cloud enchantment events in the literature, usually include
the use of a simulated clear sky radiation as baseline, that is combined with an
appropriate threshold or some other statistical characteristics, and in some cases,
with visual methods with a sky cam (Vamvakas, Salamalikis, and Kazantzidis 2020; Mol, Knap, and Van Heerwaarden 2023 and references therein).

In this study, we evaluated the effects cloud enhancements on the total downward
radiation by studding the occurrences, their intensity, and their duration in a
thirty-year period at the city of Thessaloniki. We used modeled clear sky
irradiance, as a baseline to identify cloud enhancements. We were able to determine
some trends of the phenomenon, it's Climatology, and some of their general
characteristics. We weren't able to find a comparable study that provides trends for
similar long term dataset, as ours. The recording of the radiation signal is the
mean of one-minute. Thus, the minimum resolution of cloud enhancement events (CE), in
this study is one minute.

In the relative bibliography different definition have been used for these events,
some of the are summarized by Gueymard (2017). Here, we defined as cloud enhancement
events (CE) the cases when the measured global horizontal irradiance (GHI) at ground
level, exceeds the expected value under clear-sky conditions. Similar, we define as
extreme cloud enhancement events (ECE), the cases when GHI, exceeds the Total Solar
Irradiance on horizontal plane at ground level. Although the duration of these
bursts varies, from instantaneous to several minutes, here we are constrained by the
recorded data, to one-minute steps.

\hypertarget{data-and-methodology}{%
\section{Data and methodology}\label{data-and-methodology}}

\hypertarget{ghi-data}{%
\subsection{GHI data}\label{ghi-data}}

The monitoring site is operating in the Laboratory of Atmospheric Physics of the
Aristotle University of Thessaloniki (\(40^\circ\,38'\,\)N,
\(22^\circ\,57'\,\)E, \(80\,\)m~a.s.l.).
In this study we present data from the period
13~April 1993 to
31~December 2023.
The GHI data were measured with a Kipp~\& Zonen CM-21 pyranometer. During the study
period, the pyranometer was independently calibrated three times at the
Meteorologisches Observatorium Lindenberg, DWD, verifying that the stability of the
instrument's sensitivity was better than \(0.7\,\%\) relative to the initial
calibration by the manufacturer.
For the acquisition of radiometric data, the signal of the pyranometer was sampled at
a rate of \(1\,\text{Hz}\). The mean and the standard deviation of these samples were
calculated and recorded for every minute. The measurements were corrected for the
zero offset (``dark signal'' in volts), which was calculated by averaging all
measurements recorded for a period of \(3\,\text{h}\), before (morning) or after
(evening) the Sun reaches an elevation angle of \(-10^\circ\). The signal was
converted to irradiance using the ramped value of the instrument's sensitivity
between subsequent calibrations.

To further improve the quality of the irradiance data, a manual screening was
performed to remove inconsistent and erroneous recordings that can occur
stochastically or systematically during the continuous operation of the instruments.
The manual screening was aided by a radiation data quality assurance procedure,
adjusted for the site, which was based on the methods of Long and Shi~(Long and Shi 2006, 2008). Thus, problematic recordings have been excluded from further
processing. Although it is impossible to detect all false data, the large number of
available data, and the aggregation scheme we used, ensures the quality of the
radiation measurements used in this study.
To preserve an unbiased representation of the data we applied a constraint, similar
the one used by Castillejo-Cuberos and Escobar (2020). Where, for each valid hour of day, there must
exist at least 45 minutes of valid measurements, including nighttime near sunrise and
sunset. Days with less than 5 valid hours are rejected completely.
Furthermore, due to the significant measurement uncertainty when the Sun is near the
horizon, and due to some systematic obstructions by nearby buildings, we have
excluded all measurements with solar zenith angle (SZA) greater than
\(78^\circ\).

\hypertarget{cloud-enhancement-detection}{%
\subsection{Cloud enhancement detection}\label{cloud-enhancement-detection}}

To be able to detect the CE cases, we had to establish a baseline, above which we can
characterize each data point as an enhancement event, by estimating the occurring
over irradiance (OIR). The OIR, here is defined as the irradiance difference of the
measured one-minute \(\text{GHI}_i\) from the CE identification criterion in
Equation\nobreakspace\ref{eq:CE4}
(\(\text{OIR}_i = \text{GHI}_i - \text{GHI}_\text{CSlim}\)).
To have an estimation, and a first insight on the phenomenon, we experimented with two
simple approaches for the reference. The Haurwitz's model (Haurwitz 1945), which is
as simple clear sky model, and we had already adjusted and had good fit with our data
(Natsis, Bais, and Meleti 2023), and the TSI at the top of the atmosphere. We have tested both cases
by using an appropriate relative threshold and/or an additional constant offset. The
initial results, showed that we can detect a big portion of the CE events. These
results were helpful, and helped to establish some criteria to further improve the CE
identification. It was evident, by inspecting the daily plot of irradiance, that
changes on the atmospheric conditions introduced numerous false positive and false
negative results. To produce a more accurate reference, we had to take into account
more factors that effect the clear sky radiation. So we used a radiation transfer
model.

\hypertarget{modeled-clear-sky-irradiance}{%
\subsection{Modeled clear Sky Irradiance}\label{modeled-clear-sky-irradiance}}

We approximated the expected clear sky GHI by using a radiation transfer model. The
simulations was performed by the well established model Libradtran (Emde et al. 2016), a
similar approach, was also used by Vamvakas, Salamalikis, and Kazantzidis (2020) for creating a clear sky reference.
Because of the lack of observational data for the whole period, we used some long
term climatological data for the main factors responsible for the attenuation of the
broadband downward solar radiation in the atmosphere. Which are mainly, the aerosols
and the water vapors. Fortunately, our site participates in the Aerosol Robotic
Network (AERONET) (Giles et al. 2019; Buis et al. 1998), as we operate a Cimel photometer since
2003, collocated with the CM-21 pyranometer. The mean monthly aerosol optical depth
(AOD) on different wavelengths is provided by AERONET, along with the equivalent
water column height in the atmosphere.

For completeness, we will describe here the main points of the radiation simulation
procedure. We used as input the spectrum of Kurucz (1994) in the range \(280\) to
\(2500\,\text{nm}\), with the Libradtran radiation transfer solver ``disort'' on a
``pseudospherical'' geometry and the ``LOWTRAN'' gas parameterization. For each
combination of conditions we use a SZA step of \(0.2^\circ\).
For the atmospheric characteristics, we iterated for combinations of AOD at
\(500\,\text{nm}\) (\(\tau_{500\text{nm}}\)) with additional offsets of \(\pm1\) and
\(\pm2\sigma\), and water column (\(w_h\)), also with offsets of \(\pm1\) and \(\pm2\sigma\).
We applied them on two atmospheric profiles, from the Air Force Geophysics
Laboratory (AFGL). The ``AFGL atmospheric constituent profile, midlatitude summer''
(afglms) and the ``AFGL atmospheric constituent profile, midlatitude winter'' (afglmw)
(Anderson et al. 1986).

To create a look-up table that aligns with our dataset, we applied some adjustments.
To account for the Sun's variability in our one-minute GHI measurements, we corrected
the model's input spectrum integral, to the one of the total solar irradiance (TSI)
provided by NOAA (Coddington et al. 2005). Also, we applied the effect of the Earth -- Sun
distance on the irradiance, by using the distance calculated by the Astropy
(Astropy Collaboration et al. 2022) software library. As needed, we interpolate the resulting
irradiances to the exact SZA of our measurements. For each period of the year, we
used the appropriate atmospheric profile (afglms or afglmw). Finally, we calculated
the clear sky irradiance value at the horizontal plane. Thus, we were able to
emulate different atmospheric condition and levels of atmospheric clearness for the
climatological conditions of the site. With this method, the modeled clear sky
irradiances can be directly compared to each measured one-minute value of GHI, for
different conditions of atmospheric clearness.

\hypertarget{ce-criteria-investigation}{%
\subsection{CE Criteria investigation}\label{ce-criteria-investigation}}

The use of the actual modeled values of clear sky GHI alone, can not provide us with
a robust method to distinguish the CE cases, due to the limited accuracy of the input
data. Our main focus was to positively identify over irradiance events from CEs, thus
we used a relative factor, to create an upper envelope of the clear sky irradiance,
above which, any GHI value can safely attributed to CE. We evaluated the performance
of the modeled clear radiation for each of atmospheric level of clearness, as
reference, in order to conclude which is the most appropriate.

To select the exact values of these thresholds factors (Equation \ref{eq:CE4}), we
implemented an empirical method, by manual inspection of the CE identification, on
specially selected days from the whole dataset. We used seven sets of selected days,
with characteristics relevant to the efficiency of the identification threshold.
These sets were random groups of about 20 to 30 days with the following
characteristics:
(a) the largest over irradiance CE events,
(b) the largest daily total over irradiance,
(c) without clouds (by implementing a clear sky identification algorithm as discussed in Natsis, Bais, and Meleti (2023)),
(d) without clouds and without EC events,
(e) with at least \(60\,\%\) of the day length without clouds and some EC events,
(h) random days and
(i) some manual selected days that were included during the manual inspection.
Where it was needed, for some of the edge cases, we also used images from a sky-cam,
to further aid the decisions of the manual inspection.

After evaluating the modeled clear radiation for the different atmospheric
conditions, in relation to the measured GHI data, we choose as a representative of
the clear sky radiation, the case where
\(\tau_{\text{cs}} = \tau_{500\text{nm}} - 1\sigma\) and \(w_{h\text{cs}} = w_h - 1\sigma\)
(\(\text{GHI}_\text{mCSi}\)).
These values represent a typical atmosphere in Thessaloniki with low load of aerosols
and humidity, which are the main factor that attenuate the GHI, excluding clouds. To
create the limit of CE identification, we created a two branched threshold, as a
function of SZA. A constant factor for low SZAs, and a higher ramped factor for
higher SZAs (Equation\nobreakspace\ref{eq:CE4}).
This is the criterion of our CE identification.
\begin{equation}
\text{CE} : \text{GHI}_\text{i} > \text{GHI}_\text{CSlim,i}, \text{where} \begin{cases}
 \text{GHI}_\text{CSlim,i} = 1.05 \cdot \text{GHI}_\text{mCSi}, & \text{$\theta \leq 60^\circ$}\\
\text{GHI}_\text{CSlim,i} = \left ({ 1.18 + \frac{1.05 - 1.18}{60 - 78} \cdot (\theta- 78) } \right ) \cdot \text{GHI}_\text{mCSi}, & \text{$ 78^\circ > \theta > 60^\circ$}\\
\text{Excluded measurements}, & \theta > 78^\circ
\end{cases}\label{eq:CE4}
\end{equation}
where: \(\theta\) is the solar zenith angle, \(\text{GHI}_\text{i}\) the measured
irradiance, and \(\text{GHI}_\text{mCSi}\) the selected modeled clear sky irradiance.

We have to note, that the differentiation of the threshold factor was needed, because
of the high irradiance values we observed, early in the morning and late in the
afternoon. We have confirmed, by inspecting images from the sky cam, that we have
duration of elevated irradiance, either due to the clearness of the atmosphere, or some
interferences by reflections on nearby bright surfaces. Although, these cases may
mask some of the CE events. The actual CE events, produce higher irradiance and thus
are identified as such. As a side effect, the reported OIR will be slight
underestimated for those SZAs, but the overall contribution of those cases to the
total daily energy, is minimal due the low occurrences and the lower irradiances.

\ldots\ldots\ldots\ldots.

These selections have some subjectivity, as the definition of clear sky, is depended
on the intended usage. Our main focus is to be able to identify the OIR created by the
clouds. Although, the aerosols and water vapor are always present and can not be
completely removed.
The effect of the different clearness levels

\ldots\ldots\ldots\ldots.

Another aspect of the CE events, that is often reported in the relative bibliography,
are cases of extreme cloud enhancement (ECE). These are cases of CE where the
measured intensity of the irradiance, exceeds the equivalent TSI on the top of the
atmosphere, and satisfy the Equation\nobreakspace{}\ref{eq:ECE}.
\begin{equation}
\text{ECE}: \text{GHI}_\text{i} > \cos(\theta) \times E_{i\odot} / r_{i}^2
\label{eq:ECE}
\end{equation}
where: \(\theta\) the solar zenith angle, \(E_{\odot}\) the solar constant, \(r\) the Sun -- Earth distance, \(i\) each of the one-minute observation

\ldots\ldots.

\begin{itemize}
\tightlist
\item
  Include Example of days plot in the Appendix?
  \ldots{} this is clearer to understand and describe \ldots.
\end{itemize}

\ldots\ldots.

\FloatBarrier

\hypertarget{results}{%
\section{Results}\label{results}}

Our dataset, after the data selection processing, consists of
6144534 records of GHI, of which
\(1.799\,\%\) are CE and
\(0.036\,\%\) are ECE events.
The highest GHI recorder was
\(1416.6\,W/m^2\)
on 24~May 2007.
The absolute stronger CE event had an OIR of
\(341.79\,W/m^2\) on
15~May 2014.
The relative stronger CE event was
\(54\,\%\) above the
clear sky threshold, on
28~October 2016.

\hypertarget{trends}{%
\subsection{Trends}\label{trends}}

We computed the daily trend of the mean OIR of CE
(Figure\nobreakspace{}\ref{fig:CEmeanDaily}), using a first-order autoregressive
model with lag of 1 day, using the `maximum likelihood' fitting method (Gardner, Harvey, and Phillips 1980; Jones 1980) by implementing the function `arima' from the library `stats' of the R
programming language (R Core Team 2023). The trends are reported together with the \(2\sigma\)
error. We observe an increase of
\(0.216\pm 0.083\,W/m^2/y\)
on the mean OIR.

\begin{figure}[h!]

{\centering \includegraphics[width=0.6\linewidth]{../images/P_daily_trend-1} 

}

\caption{Daily mean values and trend of the CE over irradiance.}\label{fig:CEmeanDaily}
\end{figure}

Although, the previous result is closer to the raw data, we preferred to present the
annual statistics, that give a more clear picture about the long term CE trends.
Hence, we calculated the annual values from the one-minute measurements. The annual
mean OIR is
\(0.23\pm 0.11\,W/m^2/y\)
(Figure\nobreakspace{}\ref{fig:P-energy-mean}),
but this value alone is not very useful due the intrinsic high variability of this
metric.

\begin{figure}[h!]

{\centering \includegraphics[width=0.6\linewidth]{../images/P_energy-7} 

}

\caption{Trends of the mean OIR per CE.}\label{fig:P-energy-mean}
\end{figure}

A better indicator of changes on the characteristic of CE would be the number of CE
occurrences and the total energy of the CE over irradiance. The annual number of CE
occurrences, shows a steady increase of
\(122.2\pm 22.8\,\text{cases}/y\)
(Figure\nobreakspace{}\ref{fig:P-energy-N}).
We have to note that the energy related to the CE events can not be directly linked
with the total energy balance on the atmosphere. The net sun radiation of the region
is not increased, but rather redistributed through the CE. Although, the
instantaneous values, can exceed the equivalent clear sky irradiance to a
considerable level.

\begin{figure}[h!]

{\centering \includegraphics[width=0.6\linewidth]{../images/P_energy-6} 

}

\caption{Trend of yearly CE number of occurancies.}\label{fig:P-energy-N}
\end{figure}

Subsequently, the total annual energy observed during CE events shows an increase of
\(341.6\pm 73.1\,kJ/y\)
(Figure~\ref{fig:P-energy-sum}), witch follows the trend of the number of
occurrences.

\begin{figure}[h!]

{\centering \includegraphics[width=0.6\linewidth]{../images/P_energy-5} 

}

\caption{Trend of the yearly excess energy due to CE over irradiance}\label{fig:P-energy-sum}
\end{figure}

\begin{figure}[h!]

{\centering \includegraphics[width=0.6\linewidth]{../images/P_energy-8} 

}

\caption{Trend of the yearly median over irradiance due to CE over irradiance}\label{fig:P-energy-median}
\end{figure}

\FloatBarrier

\hypertarget{climatology}{%
\subsection{Climatology}\label{climatology}}

Another interesting aspect of the CE cases, is their seasonal cycle. In
Figure\nobreakspace{}\ref{fig:relative-month-occurancies}, we have the box plot
(whisker plot), where the values have been normalized by the highest median value,
that occurs in May. Although the number of occurrences has a wide spread throughout
the study period, the most active period of CE occurrences is during May and June.
During the Winter (December -- February) the CE cases are about \(25\,\%\) of the maximum.
The rest of the months the occurrences ramp between the maximum and minimum.

\begin{figure}[h!]

{\centering \includegraphics[width=0.6\linewidth]{../images/clim_CE_month_norm_MAX_median_N-2} 

}

\caption{Statistics of the number of CE occurancies for each month. The box represents the values of the low $25\,\%$ percentile to $75\,\%$ percentile, where the thick horizontal line inside is the mean, the verical lines extend to the macimum and minimum vales, the dots are outlier values, and the rhombus is the mean.}\label{fig:relative-month-occurancies}
\end{figure}

The distribution of the CE over irradiance spreads uniformly
Figure\nobreakspace{}\ref{fig:ovir-distribution}. Where there is an inverse relation
between the events frequency and events intensity. This is expected as, the stronger
the CE events become, the rarer the particular atmospheric and sun conditions
occur.

\begin{figure}[h!]

{\centering \includegraphics[width=0.6\linewidth]{../images/P-relative-distribution-diff-1} 

}

\caption{Distribution of CE over irradiance}\label{fig:ovir-distribution}
\end{figure}

\FloatBarrier

\hypertarget{groups-stats}{%
\subsection{Groups stats}\label{groups-stats}}

In order to further study the characteristics of the CE events, we grouped the single
minute CE events, to continuous CE groups (CEG). Thus, a CEG consists of one or more
successive CE cases, and can have a duration of multiple of minutes. We have
identified 29009 groups, where the group with the longest duration lasted
140 minutes on
07~July 2013.
By examining the frequency distribution of the CEG durations
(Figure\nobreakspace{}\ref{fig:ceg-duration-distribution}), we can conclude that
longer durations are getting increasingly rare, with durations above 10 minutes very
rare.

\begin{figure}[h!]

{\centering \includegraphics[width=0.6\linewidth]{../images/groups-1} 

}

\caption{Distribution of CE groups}\label{fig:ceg-duration-distribution}
\end{figure}

\ldots{} (\textbf{Zhang2018?}) \ldots.

\begin{figure}[h!]

{\centering \includegraphics[width=0.6\linewidth]{../images/P-groups-bin2d-1} 

}

\caption{Relation of mean over irradiance and CE group duration}\label{fig:unnamed-chunk-3}
\end{figure}

\FloatBarrier

\hypertarget{extreme-ce-above-tsi}{%
\subsection{Extreme CE above TSI}\label{extreme-ce-above-tsi}}

Another aspect of the CE events are the cases when the irradiance is above the
expected irradiance on top of the atmosphere, we have defined this as ECE
(Equation\nobreakspace{}\ref{eq:ECE}).

\begin{itemize}
\tightlist
\item
  max value ECE
  relative to other works max values
\end{itemize}

Analogous to Figure\nobreakspace{}\ref{fig:relative-month-occurancies} we have
computed the distribution of the number of occurrences of ECE events by month in
Figure\nobreakspace{}\ref{fig:relative-month-occurancies-ECE}. Here the most active
period is in the spring (March -- May), followed by the months of June, September and
October.

\ldots\ldots.

\begin{figure}[h!]

{\centering \includegraphics[width=0.6\linewidth]{../images/clim_ECE_month_norm_MAX_median_N-2} 

}

\caption{Distribution of }\label{fig:relative-month-occurancies-ECE}
\end{figure}

\begin{figure}[h!]

{\centering \includegraphics[width=0.6\linewidth]{../images/extremedistributions-2} 

}

\caption{Distribution of ECE above 'clear sky' reference}\label{fig:unnamed-chunk-4}
\end{figure}

\begin{itemize}
\tightlist
\item
  SZA
\end{itemize}

\FloatBarrier

\hypertarget{discussion-and-conclusions}{%
\section{Discussion and conclusions}\label{discussion-and-conclusions}}

Compare to other location stats.
max, ECE, distributions \ldots.

Climatology results

\hypertarget{appendix}{%
\section*{Appendix}\label{appendix}}
\addcontentsline{toc}{section}{Appendix}

\hypertarget{references}{%
\section*{References}\label{references}}
\addcontentsline{toc}{section}{References}

\hypertarget{refs}{}
\begin{CSLReferences}{1}{0}
\leavevmode\vadjust pre{\hypertarget{ref-Anderson1986}{}}%
Anderson, G. P., J. H. Chetwynd, S. A. Clough, E. P. Shettle, and F. X. Kneizys. 1986. {``{AFGL} Atmospheric Constituent Profiles (0-120km).''} Air Force Geophysics Laboratory, Optical Physics Division.

\leavevmode\vadjust pre{\hypertarget{ref-AstropyCollaboration2022}{}}%
Astropy Collaboration, Adrian M. Price-Whelan, Pey Lian Lim, Nicholas Earl, Nathaniel Starkman, Larry Bradley, David L. Shupe, et al. 2022. {``{The Astropy Project: Sustaining and Growing a Community-oriented Open-source Project and the Latest Major Release (v5.0) of the Core Package}''} 935 (2): 167. \url{https://doi.org/10.3847/1538-4357/ac7c74}.

\leavevmode\vadjust pre{\hypertarget{ref-Buis1998}{}}%
Buis, J. P. P., A. Setzer, B. N. N. Holben, T. F. F. Eck, I. Slutsker, D. Tanre, E. Vermote, et al. 1998. {``AERONET---a Federated Instrument Network and Data Archive for Aerosol Characterization.''} \emph{Remote Sensing of Environment} 66 (1): 1--16.

\leavevmode\vadjust pre{\hypertarget{ref-CastillejoCuberos2020}{}}%
Castillejo-Cuberos, Armando, and Rodrigo Escobar. 2020. {``Detection and Characterization of Cloud Enhancement Events for Solar Irradiance Using a Model-Independent, Statistically-Driven Approach.''} \emph{Solar Energy} 209 (October): 547--67. \url{https://doi.org/10.1016/j.solener.2020.09.046}.

\leavevmode\vadjust pre{\hypertarget{ref-Coddington2005}{}}%
Coddington, Odele, Judith L. Lean, Doug Lindholm, Peter Pilewskie, Martin Snow, and NOAA CDR Program. 2005. {``{NOAA} Climate Data Record ({CDR}) of Total Solar Irradiance ({TSI}), {NRLTSI} Version 2. {D}aily.''} \url{https://doi.org/10.7289/V55B00C1}.

\leavevmode\vadjust pre{\hypertarget{ref-Damiani2018}{}}%
Damiani, Alessandro, Hitoshi Irie, Takashi Horio, Tamio Takamura, Pradeep Khatri, Hideaki Takenaka, Takashi Nagao, Takashi Y. Nakajima, and Raul R. Cordero. 2018. {``Evaluation of Himawari-8 Surface Downwelling Solar Radiation by Ground-Based Measurements.''} \emph{Atmos. Meas. Tech.}

\leavevmode\vadjust pre{\hypertarget{ref-DoNascimento2019}{}}%
Do Nascimento, Lucas Rafael, Trajano De Souza Viana, Rafael Antunes Campos, and Ricardo Rüther. 2019. {``Extreme Solar Overirradiance Events: Occurrence and Impacts on Utility-Scale Photovoltaic Power Plants in Brazil.''} \emph{Solar Energy} 186 (July): 370--81. \url{https://doi.org/10.1016/j.solener.2019.05.008}.

\leavevmode\vadjust pre{\hypertarget{ref-Emde2016}{}}%
Emde, Claudia, Robert Buras-Schnell, Arve Kylling, Bernhard Mayer, Josef Gasteiger, Ulrich Hamann, Jonas Kylling, et al. 2016. {``The {libRadtran} Software Package for Radiative Transfer Calculations (Version 2.0.1).''} \emph{Geoscientific Model Development} 9 (5): 1647--72. \url{https://doi.org/10.5194/gmd-9-1647-2016}.

\leavevmode\vadjust pre{\hypertarget{ref-Gardner1980}{}}%
Gardner, G., A. C. Harvey, and G. D. A. Phillips. 1980. {``Algorithm {AS} 154: An Algorithm for Exact Maximum Likelihood Estimation of Autoregressive-Moving Average Models by Means of Kalman Filtering.''} \emph{Applied Statistics} 29 (3): 311. \url{https://doi.org/10.2307/2346910}.

\leavevmode\vadjust pre{\hypertarget{ref-Giles2019}{}}%
Giles, David M., Alexander Sinyuk, Mikhail G. Sorokin, Joel S. Schafer, Alexander Smirnov, Ilya Slutsker, Thomas F. Eck, et al. 2019. {``Advancements in the Aerosol Robotic Network ({AERONET}) Version 3 Database -- Automated Near-Real-Time Quality Control Algorithm with Improved Cloud Screening for Sun Photometer Aerosol Optical Depth ({AOD}) Measurements.''} \emph{Atmospheric Measurement Techniques} 12 (1): 169--209. \url{https://doi.org/10.5194/amt-12-169-2019}.

\leavevmode\vadjust pre{\hypertarget{ref-Graabak2016}{}}%
Graabak, Ingeborg, and Magnus Korpås. 2016. {``Variability Characteristics of European Wind and Solar Power Resources---a Review.''} \emph{Energies} 9 (6): 449. \url{https://doi.org/10.3390/en9060449}.

\leavevmode\vadjust pre{\hypertarget{ref-Gray2010}{}}%
Gray, L. J., J. Beer, M. Geller, J. D. Haigh, M. Lockwood, K. Matthes, U. Cubasch, et al. 2010. {``{SOLAR} {INFLUENCES} {ON} {CLIMATE}.''} \emph{Reviews of Geophysics} 48 (4): RG4001. \url{https://doi.org/10.1029/2009RG000282}.

\leavevmode\vadjust pre{\hypertarget{ref-Gueymard2017}{}}%
Gueymard, Christian A. 2017. {``Cloud and Albedo Enhancement Impacts on Solar Irradiance Using High-Frequency Measurements from Thermopile and Photodiode Radiometers. Part 1: Impacts on Global Horizontal Irradiance.''} \emph{Solar Energy} 153: 755--65. \url{https://doi.org/gb283c}.

\leavevmode\vadjust pre{\hypertarget{ref-Haurwitz1945}{}}%
Haurwitz, Bernhard. 1945. {``Insolation in {Relation} to {Cloudiness} and {Cloud} {Density}.''} \emph{Journal of Meteorology} 2 (September): 154--66.

\leavevmode\vadjust pre{\hypertarget{ref-Inman2013}{}}%
Inman, Rich H., Hugo T. C. Pedro, and Carlos F. M. Coimbra. 2013. {``Solar Forecasting Methods for Renewable Energy Integration.''} \emph{Progress in Energy and Combustion Science} 39 (6): 535--76. \url{https://doi.org/10.1016/j.pecs.2013.06.002}.

\leavevmode\vadjust pre{\hypertarget{ref-Jaervelae2020}{}}%
Järvelä, Markku, and Seppo Valkealahti. 2020. {``Operation of a {PV} Power Plant During Overpower Events Caused by the Cloud Enhancement Phenomenon.''} \emph{Energies} 13 (9): 2185. \url{https://doi.org/10.3390/en13092185}.

\leavevmode\vadjust pre{\hypertarget{ref-Jones1980}{}}%
Jones, Richard H. 1980. {``Maximum Likelihood Fitting of {ARMA} Models to Time Series with Missing Observations.''} \emph{Technometrics} 22 (3): 389--95. \url{https://doi.org/10.1080/00401706.1980.10486171}.

\leavevmode\vadjust pre{\hypertarget{ref-Kurucz1994}{}}%
Kurucz, Robert L. 1994. {``Synthetic Infrared Spectra.''} In \emph{Infrared Solar Physics}, edited by D. M. Rabin, J. T. Jefferies, and C. Lindsey, 523--31. Dordrecht: Springer Netherlands.

\leavevmode\vadjust pre{\hypertarget{ref-Lappalainen2020}{}}%
Lappalainen, Kari, and Jan Kleissl. 2020. {``Analysis of the Cloud Enhancement Phenomenon and Its Effects on Photovoltaic Generators Based on Cloud Speed Sensor Measurements.''} \emph{Journal of Renewable and Sustainable Energy} 12 (4): 043502. \url{https://doi.org/10.1063/5.0007550}.

\leavevmode\vadjust pre{\hypertarget{ref-Long2006}{}}%
Long, Charles N., and Y. Shi. 2006. {``The QCRad Value Added Product: Surface Radiation Measurement Quality Control Testing, Including Climatology Configurable Limits.''} DOE/SC-ARM/TR-074. Office of Science, Office of Biological; Environmental Research, U.S. Department of Energy.

\leavevmode\vadjust pre{\hypertarget{ref-Long2008a}{}}%
---------. 2008. {``An Automated Quality Assessment and Control Algorithm for Surface Radiation Measurements.''} \emph{The Open Atmospheric Science Journal}, 23--37.

\leavevmode\vadjust pre{\hypertarget{ref-Martins2022}{}}%
Martins, G. L., S. L. Mantelli, and R. Rüther. 2022. {``Evaluating the Performance of Radiometers for Solar Overirradiance Events.''} \emph{Solar Energy} 231 (January): 47--56. \url{https://doi.org/10.1016/j.solener.2021.11.050}.

\leavevmode\vadjust pre{\hypertarget{ref-Mol2023}{}}%
Mol, Wouter B., Wouter H. Knap, and Chiel C. Van Heerwaarden. 2023. {``Ten Years of 1 Hz Solar Irradiance Observations at Cabauw, the Netherlands, with Cloud Observations, Variability Classifications, and Statistics.''} \emph{Earth System Science Data} 15 (5): 2139--51. \url{https://doi.org/10.5194/essd-15-2139-2023}.

\leavevmode\vadjust pre{\hypertarget{ref-Natsis2023}{}}%
Natsis, Athanasios, Alkiviadis Bais, and Charikleia Meleti. 2023. {``Trends from 30-Year Observations of Downward Solar Irradiance in Thessaloniki, Greece.''} \emph{Applied Sciences} 14 (1): 252. \url{https://doi.org/10.3390/app14010252}.

\leavevmode\vadjust pre{\hypertarget{ref-Pecenak2016}{}}%
Pecenak, Zachary K., Felipe A. Mejia, Ben Kurtz, Amato Evan, and Jan Kleissl. 2016. {``Simulating Irradiance Enhancement Dependence on Cloud Optical Depth and Solar Zenith Angle.''} \emph{Solar Energy} 136 (October): 675--81. \url{https://doi.org/10.1016/j.solener.2016.07.045}.

\leavevmode\vadjust pre{\hypertarget{ref-RCT2023}{}}%
R Core Team. 2023. \emph{R: A Language and Environment for Statistical Computing}. Vienna, Austria: R Foundation for Statistical Computing. \url{https://www.R-project.org/}.

\leavevmode\vadjust pre{\hypertarget{ref-Thuillier2013}{}}%
Thuillier, Gérard, Jean-Marie Perrin, Philippe Keckhut, and François Huppert. 2013. {``Local Enhanced Solar Irradiance on the Ground Generated by Cirrus: Measurements and Interpretation.''} \emph{Journal of Applied Remote Sensing} 7 (1): 073543. \url{https://doi.org/10.1117/1.JRS.7.073543}.

\leavevmode\vadjust pre{\hypertarget{ref-Vamvakas2020}{}}%
Vamvakas, Ioannis, Vasileios Salamalikis, and Andreas Kazantzidis. 2020. {``Evaluation of Enhancement Events of Global Horizontal Irradiance Due to Clouds at Patras, South-West Greece.''} \emph{Renewable Energy} 151 (xxxx): 764--71. \url{https://doi.org/gq3sbh}.

\leavevmode\vadjust pre{\hypertarget{ref-Veerman2022}{}}%
Veerman, M. A., B. J. H. Van Stratum, and C. C. Van Heerwaarden. 2022. {``A Case Study of Cumulus Convection over Land in Cloud‐resolving Simulations with a Coupled Ray Tracer.''} \emph{Geophysical Research Letters} 49 (23). \url{https://doi.org/10.1029/2022GL100808}.

\leavevmode\vadjust pre{\hypertarget{ref-Yordanov2013}{}}%
Yordanov, G. H., O.-M. Midtgård, T. O. Saetre, H. K. Nielsen, and L. E. Norum. 2013. {``Overirradiance (Cloud Enhancement) Events at High Latitudes.''} \emph{{IEEE} Journal of Photovoltaics} 3 (1): 271--77. \url{https://doi.org/10.1109/JPHOTOV.2012.2213581}.

\leavevmode\vadjust pre{\hypertarget{ref-Yordanov2015}{}}%
Yordanov, Georgi Hristov, Tor Oskar Saetre, and Ole-Morten Midtgård. 2015. {``Extreme Overirradiance Events in Norway: 1.6 Suns Measured Close to 60°n.''} \emph{Solar Energy} 115 (May): 68--73. \url{https://doi.org/10.1016/j.solener.2015.02.020}.

\end{CSLReferences}

\end{document}
