\documentclass[preprint, 5p,
%DIF LATEXDIFF DIFFERENCE FILE
%DIF DEL /home/athan/MANUSCRIPTS/02_enhancement/SUBMISSION_02/article/article.tex   Wed Sep 11 11:07:21 2024
%DIF ADD /home/athan/MANUSCRIPTS/02_enhancement/article/article.tex                 Wed Oct  9 20:49:57 2024
authoryear]{elsarticle} %review=doublespace preprint=single 5p=2 column
%%% Begin My package additions %%%%%%%%%%%%%%%%%%%

\usepackage[hyphens]{url}

  \journal{Atmospheric Research} % Sets Journal name

\usepackage{graphicx}
%%%%%%%%%%%%%%%% end my additions to header

\usepackage[T1]{fontenc}
\usepackage{lmodern}
\usepackage{amssymb,amsmath}
% TODO: Currently lineno needs to be loaded after amsmath because of conflict
% https://github.com/latex-lineno/lineno/issues/5
\usepackage{lineno} % add
\usepackage{ifxetex,ifluatex}
\usepackage{fixltx2e} % provides \textsubscript
% use upquote if available, for straight quotes in verbatim environments
\IfFileExists{upquote.sty}{\usepackage{upquote}}{}
\ifnum 0\ifxetex 1\fi\ifluatex 1\fi=0 % if pdftex
  \usepackage[utf8]{inputenc}
\else % if luatex or xelatex
  \usepackage{fontspec}
  \ifxetex
    \usepackage{xltxtra,xunicode}
  \fi
  \defaultfontfeatures{Mapping=tex-text,Scale=MatchLowercase}
  \newcommand{\euro}{€}
\fi
% use microtype if available
\IfFileExists{microtype.sty}{\usepackage{microtype}}{}
\usepackage[]{natbib}
\bibliographystyle{elsarticle-harv}

\ifxetex
  \usepackage[setpagesize=false, % page size defined by xetex
              unicode=false, % unicode breaks when used with xetex
              xetex]{hyperref}
\else
  \usepackage[unicode=true]{hyperref}
\fi
\hypersetup{breaklinks=true,
            bookmarks=true,
            pdfauthor={},
            pdftitle={Analysis of cloud enhancement events in a 30-year record of global solar irradiance at Thessaloniki, Greece},
            colorlinks=false,
            urlcolor=blue,
            linkcolor=magenta,
            pdfborder={0 0 0}}

\setcounter{secnumdepth}{5}
% Pandoc toggle for numbering sections (defaults to be off)


% tightlist command for lists without linebreak
\providecommand{\tightlist}{%
  \setlength{\itemsep}{0pt}\setlength{\parskip}{0pt}}




\usepackage{caption}
\usepackage{placeins}
\captionsetup{font=small}
\usepackage{subcaption}
\usepackage{booktabs}
\usepackage{longtable}
\usepackage{array}
\usepackage{multirow}
\usepackage{wrapfig}
\usepackage{float}
\usepackage{colortbl}
\usepackage{pdflscape}
\usepackage{tabu}
\usepackage{threeparttable}
\usepackage{threeparttablex}
\usepackage[normalem]{ulem}
\usepackage{makecell}
\usepackage{xcolor}
%DIF PREAMBLE EXTENSION ADDED BY LATEXDIFF
%DIF UNDERLINE PREAMBLE %DIF PREAMBLE
\RequirePackage[normalem]{ulem} %DIF PREAMBLE
\RequirePackage{color}\definecolor{RED}{rgb}{1,0,0}\definecolor{BLUE}{rgb}{0,0,1} %DIF PREAMBLE
\providecommand{\DIFaddtex}[1]{{\protect\color{blue}\uwave{#1}}} %DIF PREAMBLE
\providecommand{\DIFdeltex}[1]{{\protect\color{red}\sout{#1}}}                      %DIF PREAMBLE
%DIF SAFE PREAMBLE %DIF PREAMBLE
\providecommand{\DIFaddbegin}{} %DIF PREAMBLE
\providecommand{\DIFaddend}{} %DIF PREAMBLE
\providecommand{\DIFdelbegin}{} %DIF PREAMBLE
\providecommand{\DIFdelend}{} %DIF PREAMBLE
\providecommand{\DIFmodbegin}{} %DIF PREAMBLE
\providecommand{\DIFmodend}{} %DIF PREAMBLE
%DIF FLOATSAFE PREAMBLE %DIF PREAMBLE
\providecommand{\DIFaddFL}[1]{\DIFadd{#1}} %DIF PREAMBLE
\providecommand{\DIFdelFL}[1]{\DIFdel{#1}} %DIF PREAMBLE
\providecommand{\DIFaddbeginFL}{} %DIF PREAMBLE
\providecommand{\DIFaddendFL}{} %DIF PREAMBLE
\providecommand{\DIFdelbeginFL}{} %DIF PREAMBLE
\providecommand{\DIFdelendFL}{} %DIF PREAMBLE
%DIF HYPERREF PREAMBLE %DIF PREAMBLE
\providecommand{\DIFadd}[1]{\texorpdfstring{\DIFaddtex{#1}}{#1}} %DIF PREAMBLE
\providecommand{\DIFdel}[1]{\texorpdfstring{\DIFdeltex{#1}}{}} %DIF PREAMBLE
\newcommand{\DIFscaledelfig}{0.5}
%DIF HIGHLIGHTGRAPHICS PREAMBLE %DIF PREAMBLE
\RequirePackage{settobox} %DIF PREAMBLE
\RequirePackage{letltxmacro} %DIF PREAMBLE
\newsavebox{\DIFdelgraphicsbox} %DIF PREAMBLE
\newlength{\DIFdelgraphicswidth} %DIF PREAMBLE
\newlength{\DIFdelgraphicsheight} %DIF PREAMBLE
% store original definition of \includegraphics %DIF PREAMBLE
\LetLtxMacro{\DIFOincludegraphics}{\includegraphics} %DIF PREAMBLE
\newcommand{\DIFaddincludegraphics}[2][]{{\color{blue}\fbox{\DIFOincludegraphics[#1]{#2}}}} %DIF PREAMBLE
\newcommand{\DIFdelincludegraphics}[2][]{% %DIF PREAMBLE
\sbox{\DIFdelgraphicsbox}{\DIFOincludegraphics[#1]{#2}}% %DIF PREAMBLE
\settoboxwidth{\DIFdelgraphicswidth}{\DIFdelgraphicsbox} %DIF PREAMBLE
\settoboxtotalheight{\DIFdelgraphicsheight}{\DIFdelgraphicsbox} %DIF PREAMBLE
\scalebox{\DIFscaledelfig}{% %DIF PREAMBLE
\parbox[b]{\DIFdelgraphicswidth}{\usebox{\DIFdelgraphicsbox}\\[-\baselineskip] \rule{\DIFdelgraphicswidth}{0em}}\llap{\resizebox{\DIFdelgraphicswidth}{\DIFdelgraphicsheight}{% %DIF PREAMBLE
\setlength{\unitlength}{\DIFdelgraphicswidth}% %DIF PREAMBLE
\begin{picture}(1,1)% %DIF PREAMBLE
\thicklines\linethickness{2pt} %DIF PREAMBLE
{\color[rgb]{1,0,0}\put(0,0){\framebox(1,1){}}}% %DIF PREAMBLE
{\color[rgb]{1,0,0}\put(0,0){\line( 1,1){1}}}% %DIF PREAMBLE
{\color[rgb]{1,0,0}\put(0,1){\line(1,-1){1}}}% %DIF PREAMBLE
\end{picture}% %DIF PREAMBLE
}\hspace*{3pt}}} %DIF PREAMBLE
} %DIF PREAMBLE
\LetLtxMacro{\DIFOaddbegin}{\DIFaddbegin} %DIF PREAMBLE
\LetLtxMacro{\DIFOaddend}{\DIFaddend} %DIF PREAMBLE
\LetLtxMacro{\DIFOdelbegin}{\DIFdelbegin} %DIF PREAMBLE
\LetLtxMacro{\DIFOdelend}{\DIFdelend} %DIF PREAMBLE
\DeclareRobustCommand{\DIFaddbegin}{\DIFOaddbegin \let\includegraphics\DIFaddincludegraphics} %DIF PREAMBLE
\DeclareRobustCommand{\DIFaddend}{\DIFOaddend \let\includegraphics\DIFOincludegraphics} %DIF PREAMBLE
\DeclareRobustCommand{\DIFdelbegin}{\DIFOdelbegin \let\includegraphics\DIFdelincludegraphics} %DIF PREAMBLE
\DeclareRobustCommand{\DIFdelend}{\DIFOaddend \let\includegraphics\DIFOincludegraphics} %DIF PREAMBLE
\LetLtxMacro{\DIFOaddbeginFL}{\DIFaddbeginFL} %DIF PREAMBLE
\LetLtxMacro{\DIFOaddendFL}{\DIFaddendFL} %DIF PREAMBLE
\LetLtxMacro{\DIFOdelbeginFL}{\DIFdelbeginFL} %DIF PREAMBLE
\LetLtxMacro{\DIFOdelendFL}{\DIFdelendFL} %DIF PREAMBLE
\DeclareRobustCommand{\DIFaddbeginFL}{\DIFOaddbeginFL \let\includegraphics\DIFaddincludegraphics} %DIF PREAMBLE
\DeclareRobustCommand{\DIFaddendFL}{\DIFOaddendFL \let\includegraphics\DIFOincludegraphics} %DIF PREAMBLE
\DeclareRobustCommand{\DIFdelbeginFL}{\DIFOdelbeginFL \let\includegraphics\DIFdelincludegraphics} %DIF PREAMBLE
\DeclareRobustCommand{\DIFdelendFL}{\DIFOaddendFL \let\includegraphics\DIFOincludegraphics} %DIF PREAMBLE
%DIF COLORLISTINGS PREAMBLE %DIF PREAMBLE
\RequirePackage{listings} %DIF PREAMBLE
\RequirePackage{color} %DIF PREAMBLE
\lstdefinelanguage{DIFcode}{ %DIF PREAMBLE
%DIF DIFCODE_UNDERLINE %DIF PREAMBLE
  moredelim=[il][\color{red}\sout]{\%DIF\ <\ }, %DIF PREAMBLE
  moredelim=[il][\color{blue}\uwave]{\%DIF\ >\ } %DIF PREAMBLE
} %DIF PREAMBLE
\lstdefinestyle{DIFverbatimstyle}{ %DIF PREAMBLE
	language=DIFcode, %DIF PREAMBLE
	basicstyle=\ttfamily, %DIF PREAMBLE
	columns=fullflexible, %DIF PREAMBLE
	keepspaces=true %DIF PREAMBLE
} %DIF PREAMBLE
\lstnewenvironment{DIFverbatim}{\lstset{style=DIFverbatimstyle}}{} %DIF PREAMBLE
\lstnewenvironment{DIFverbatim*}{\lstset{style=DIFverbatimstyle,showspaces=true}}{} %DIF PREAMBLE
%DIF END PREAMBLE EXTENSION ADDED BY LATEXDIFF

\begin{document}


\begin{frontmatter}

  \title{Analysis of cloud enhancement events in a 30-year record of
global solar irradiance at Thessaloniki, Greece}
    \author[LAP]{Athanasios N. Natsis%
  \corref{cor1}%
  }
   \ead{natsisphysicist@gmail.com} 
    \author[LAP]{Alkiviadis Bais%
  %
  }
   \ead{abais@auth.gr} 
    \author[LAP]{Charikleia Meleti%
  %
  }
   \ead{meleti@auth.gr} 
      \affiliation[LAP]{
    organization={Laboratory of Atmospheric Physics, Physics Department,
Aristotle University of
Thessaloniki},city={Thessaloniki},postcode={54124},country={Greece},}
    \cortext[cor1]{Corresponding author}

  \begin{abstract}
  In this study, we investigate the characteristics of global horizontal
  irradiance enhancement events induced by clouds over Thessaloniki for
  the period 1994 -- 2023 using data recorded every one minute. We
  identified the cloud enhancement (CE) events by creating an
  appropriate cloud-free irradiance reference using a radiative transfer
  model and aerosol optical depth data from a collocated Cimel sun
  photometer and a Brewer spectrophotometer. We found a trend in CE
  events of \(+112\pm 35\,\text{cases}/\text{year}\), and a trend in the
  corresponding irradiation of
  \DIFdelbegin \DIFdel{\(+329.9\pm 112\,\text{kJ}/\text{year}\)}\DIFdelend \DIFaddbegin \DIFadd{\(+329.9\pm 112.0\,\text{kJ}/\text{year}\)}\DIFaddend . To our knowledge, such
  long-term changes in CE events have not been presented in the past.
  The peak of the CE events was observed during May and June. CE events
  with duration longer than 10\nobreakspace{}min are very rare
  (\(<8\,\%\)), with exceptions lasting over an hour and up to 140
  minutes. Finally, we have detected enhancements above the total solar
  irradiance at the top of the atmosphere for the same solar zenith
  angle of up to \(204\,\text{W}/\text{m}^{2}\), with the \(75\,\%\) of
  the cases below \(40\,\text{W}/\text{m}^{2}\). Most of these extreme
  events occur in spring -- early summer, with a secondary peak in
  autumn.
  \end{abstract}
    \begin{keyword}
    cloud enhancement \sep total solar radiation \sep global horizontal
irradiance \sep 
    over irradiance
  \end{keyword}

 \end{frontmatter}

\hypertarget{introduction}{%
\section{Introduction}\label{introduction}}

The shortwave solar radiation, reaching Earth's surface, is the main
energy source for the atmosphere and the biosphere, and drives and
governs the climate \citep{Gray2010}. It has direct practical
application in industries related to energy and agricultural production.
The variability of its intensity can impose difficulties in predicting
the yield and in designing the specifications of the appropriate
equipment. A significant portion of the relevant research has been
focused on predicting renewable energy production in a fine timescale
and in near real-time \citep[for a review
see][]{Inman2013, Graabak2016}.

An important aspect of the variability of solar radiation is its
interaction with the clouds. In general, clouds can attenuate a fraction
of solar irradiance, but under certain conditions, can lead to
enhancement of the global horizontal irradiance (GHI) reaching the
ground. This cloud enhancement (CE) effect can locally increase the
observed GHI to levels even higher than the expected cloud-free
irradiance {[}\citet{Cordero2023}; \citet{Vamvakas2020};
\citet{CastillejoCuberos2020}; and references therein{]}.

Some of the proposed underlying mechanisms of those enhancements have
been summarized by \citet{Gueymard2017}; the most important being the
scattering of radiation on the edges of cumulus clouds. It has also been
suggested that enhancement of GHI can be produced by thin cirrus clouds
through refraction and scattering \citep{Thuillier2013}. Further
investigation with radiative transfer modeling and observations pointed
as the prevailing mechanism, the strong forward Mie scattering through
clouds of low optical depth
\citep{Pecenak2016, Thuillier2013, Yordanov2013, Yordanov2015}. Overall,
the appearance of CE events depends on different interactive factors,
which include cloud thickness, structure and type, and the relative
position of the sun and the clouds \citep{Gueymard2017, Veerman2022}.

On multiple sites, cloud enhancements have been reported that exceed
momentarily the solar constant, resulting in clearness indices above
unit. A summary of extreme cloud enhancement (ECE) cases has been
compiled by \citet{Martins2022}. Cloud enhancements can have also some
practical implications. The intensity and duration of enhancements can
affect the efficiency and stability of photovoltaic power production
\citep{Lappalainen2020, Jaervelae2020}, while ECEs have the potential to
compromise the integrity of photovoltaic plants infrastructure
\citep{DoNascimento2019}. It has also been demonstrated that these
events can interfere in the comparison of ground-based and satellite
observations of radiation \citep{Damiani2018}. Global warming has likely
affected cloud coverage in the last few decades. \citet{Liu2023}
reported increases in cloud cover over the tropical and subtropical
oceans and decreases over most continents, while \citet{Dong2023}
reported decreases over North America and Europe. To our knowledge,
there is no evidence of whether this trend has also affected the number
and strength of CE events.

Methods of identification of CE events usually include the use of
simulated cloud-free irradiance as baseline, combined with an
appropriate threshold or some other statistical characteristics, and in
some cases, with visual inspection of sky camera images \citep[ and
references therein]{Vamvakas2020, Mol2023}.

In this study, we evaluate the effects of CE events on GHI by
investigating their frequency of occurrence, intensity, and duration in
a thirty-year record of GHI observations at Thessaloniki, Greece. We
used modeled cloud-free irradiance as a baseline to identify cloud
enhancements, and we determined long-term trends of the above-mentioned
metrics, their climatology and some general characteristics. To our
knowledge, there are no other studies that provide trends from such a
long dataset.

\hypertarget{data-and-methodology}{%
\section{Data and methodology}\label{data-and-methodology}}

\hypertarget{instrumentation-and-data}{%
\subsection{Instrumentation and data}\label{instrumentation-and-data}}

The data used in this study were recorded at the monitoring site of the
Laboratory of Atmospheric Physics, Aristotle University of Thessaloniki,
in Thessaloniki, Greece (\(40^\circ\,38'\,\)N, \(22^\circ\,57'\,\)E,
\(80\,\)m~a.s.l.). The GHI data were measured with a Kipp~\& Zonen CM-21
pyranometer and cover the period 01~January 1994 to 31~December 2023.
During the study period, the pyranometer was independently calibrated
three times at the Meteorologisches Observatorium Lindenberg, DWD,
verifying that the stability of the instrument's sensitivity was better
than \(0.7\,\%\) relative to the initial calibration by the
manufacturer. For the acquisition of radiometric data, the signal of the
pyranometer was sampled at a rate of \(1\,\text{Hz}\) with the mean and
standard deviation of these samples calculated and recorded every
minute. The measurements were corrected for the zero offset (``dark
signal'' in volts), which was calculated by averaging all measurements
recorded for a period of \(3\,\text{h}\), before (morning) or after
(evening) the Sun reaches an elevation angle of \(-10^\circ\). The
signal was converted to irradiance using the ramped value of the
instrument's sensitivity between subsequent calibrations. We note that
the specific model of the pyranometer is not capable of recording the
instrument's temperature, therefore any temperature changes that may
occur during the day (including those due to sudden increases or
decreases in incidence irradiance) were not considered in the data
reduction procedure. However, the sensitivity of the instrument to
temperature is less than \(\pm1\,\%\) for its operational range
(\(-20^\circ\) -- \(50^\circ\)C).

To further improve the quality of the irradiance data, a manual
screening was performed, to remove inconsistent and erroneous recordings
that can occur stochastically or systematically during the long
operation of the instrument. The manual screening was aided by a
radiation data quality assurance procedure, adjusted for the site, which
was based on the methods of Long and
Shi~\DIFdelbegin \DIFdel{\mbox{%DIFAUXCMD
\citep{Long2006, Long2008a}}\hskip0pt%DIFAUXCMD
}\DIFdelend \DIFaddbegin \DIFadd{\mbox{%DIFAUXCMD
\citetext{\citeyear{Long2006}; \citeyear{Long2008a}}}\hskip0pt%DIFAUXCMD
}\DIFaddend . Thus,
problematic recordings have been excluded from further processing.
Furthermore, due to the significant measurement uncertainty in GHI when
the Sun is near the horizon, and due to some systematic obstructions by
nearby buildings, we have excluded all measurements with solar zenith
angle (SZA) greater than \(78^\circ\). Finally, images from a sky camera
have been used in the manual inspection of the CE events identification.
The sky camera has been operating since 2012 and stores images in 15 min
time steps. An overview of the GHI data used in this study is given in
Figure\nobreakspace{}\ref{fig:CLB-daily}, as the daily means. Daily
means were calculated only for days with at least \(60\,\%\) of data
availability; however, all available one-minute measurements have been
used for the detection of CE events.

\begin{figure}

{\centering \includegraphics[width=1\linewidth]{../images/P_daily_trend-5} 

}

\caption{Time-series of daily mean GHI measured at Thessaloniki for the period 1994 -- 2023.}\label{fig:CLB-daily}
\end{figure}

\hypertarget{detection-of-cloud-enhancements}{%
\subsection{Detection of cloud
enhancements}\label{detection-of-cloud-enhancements}}

In this study, we define an event as CE when the measured GHI at ground
level exceeds the expected GHI under cloud-free conditions. Similarly,
we define as extreme cloud enhancement (ECE) events the cases when GHI
at ground level exceeds the Total Solar Irradiance (TSI) at the top of
the atmosphere (TOA) for the same SZA. Although the duration of these
bursts can vary from seconds to several minutes, here we are constrained
by the temporal resolution of our data to identify events with a
duration of at minimum one-minute.

For the detection of CE cases, we established a baseline of irradiance
above which we characterized each data point as a CE event and
calculated the over irradiance (OI). The OI is defined as the irradiance
difference of the measured one-minute GHI from the
\(\text{GHI}_\text{ref}\) corresponding to cloud-free atmosphere. First,
we used a simple approach for the determination of
\(\text{GHI}_\text{ref}\): The Haurwitz's model \citep{Haurwitz1945},
which is a simple clear sky radiation model and was already adjusted and
applied to our data in \citet{Natsis2023}. We created a threshold by
using an appropriate relative and/or an additional constant offset. The
initial results showed that we can detect a big portion of the actual CE
events. However, by inspecting the daily plots of irradiance, it became
evident that changes in atmospheric conditions introduced numerous false
positive or false negative results. The main reason for these
discrepancies is the variability of the effects of aerosols and water
vapor which were not considered in the simple method. To produce a more
representative reference we included the effects of these factors using
a radiative transfer model (RTM). The applied methodology is discussed
in section\nobreakspace{}\ref{rtmcs}.

\hypertarget{rtmcs}{%
\subsection{Modeled cloud-free irradiance}\label{rtmcs}}

\hypertarget{climatology-of-cloud-free-irradiance}{%
\subsubsection{Climatology of cloud-free
irradiance}\label{climatology-of-cloud-free-irradiance}}

We approximated the expected cloud-free \(\text{GHI}_\text{ref}\) with
the radiative transfer model uvspec, part of libRadtran
\citep{Emde2016}, similarly to the approach used by
\citet{Vamvakas2020}. In uvspec we used the solar spectrum of
\citet{Kurucz1994} in the range \(280\) to \(2500\,\text{nm}\), the
radiative transfer solver ``disort'' in ``pseudospherical'' geometry and
the ``LOWTRAN'' gas parameterization. The model was run for
climatological values of the aerosol Ångström coefficients, water column
(WC), SZA and the appropriate seasonal atmospheric profile to create a
look-up table (LUT) for the estimation of the cloud-free reference
irradiance for each individual observation of our dataset. In this
context, the model was run for SZAs in the range \(10\) -- \(90^\circ\)
with a step of \(0.2^\circ\) and for the atmospheric profiles of the Air
Force Geophysics Laboratory \citep{Anderson1986} midlatitude summer and
midlatitude winter, representative of the warm and cold seasons.

Main factors responsible for the attenuation of the broadband downward
solar radiation under cloud-free atmospheres are aerosols and water
vapor. At Thessaloniki, such measurements have been available since 2003
from a Cimel sun photometer, which is part of the Aerosol Robotic
Network (AERONET) \citep{Giles2019, Buis1998}. From the observations in
the period 2003 -- 2023 we calculated the monthly climatological means
and standard deviations (\(\sigma\)) for the aerosol optical depth (AOD)
at \(500\,\text{nm}\) and the equivalent height of the water column
(WC). The monthly climatological values of AOD and WC, as well as
combinations with additional offsets of \(\pm1\sigma\) and
\(\pm2\sigma\), were used as inputs to the RTM in the construction of
the LUT.

For each measurement of the dataset, we calculated from the LUT a
\(\text{GHI}_\text{ref}\) value for the respective season and SZA (by
linear interpolation), and for the climatological values of AOD and WC
of the respective month. The same procedure was followed for the
estimation of the \(\text{GHI}_\text{ref}\) for all combinations of the
AOD and WC with the above-mentioned standard deviation offsets. Finally,
each \(\text{GHI}_\text{ref}\) value was adjusted to the actual
Sun-Earth distance derived by the Astropy software library
\citep{AstropyCollaboration2022}.

\hypertarget{long-term-changes-in-cloud-free-irradiance}{%
\subsubsection{Long-term changes in cloud-free
irradiance}\label{long-term-changes-in-cloud-free-irradiance}}

The cloud-free reference values discussed above are based on the
climatological AOD and WC; hence they cannot describe accurately the
long-term variation of \(\text{GHI}_\text{ref}\) due to long-term
changes in the two atmospheric constituents, mainly AOD. As reported by
\citet{Natsis2023}, there is a long-term brightening effect in the GHI
data of Thessaloniki for the period 1993 -- 2023, which for cloud-free
data was attributed to long-term changes in aerosol effects. Therefore,
an adjustment of the \(\text{GHI}_\text{ref}\) during the period of
study was made using RTM simulations to account for the long-term
variations of the AOD. High quality AOD data with a Cimel sun-photometer
at Thessaloniki start only in 2003, while spectral AOD measurements
using direct-sun observations with a MKIII Brewer spectrophotometer are
available for the period 1997 -- 2017. The Brewer AOD data are taken
sporadically during each day and are less dense compared to the AERONET
data. By comparing monthly AOD data at \(340\,\text{nm}\) of the two
instruments for the common periods of operation we adjusted the Brewer
data and filled the missing months of the Cimel record with Brewer data.
Using the extended monthly time series of AOD at \(340\,\text{nm}\), as
well as monthly climatological values of the Ångström exponent and
constant WC of \DIFdelbegin \DIFdel{\(15.6\,\text{mm}\) }\DIFdelend \DIFaddbegin \DIFadd{\(15.5\,\text{mm}\) }\DIFaddend derived from the Cimel record, we
simulated with the RTM the cloud-free GHI at SZA of \(55^\circ\) for
each month in the period 1997 -- 2023. The SZA of \(55^\circ\) was
chosen as representative of all days in the year to get a rough estimate
of the annually averaged change in cloud-free GHI. A second-degree
polynomial fit was applied to the simulated yearly averaged GHI to
derive the long-term change in GHI due to aerosols: \begin{equation}
\Delta(\text{GHI}) [\%] = \DIFdelbegin \DIFdel{-12170 }\DIFdelend \DIFaddbegin \DIFadd{-12154 }\DIFaddend + \DIFdelbegin \DIFdel{12.05 }\DIFdelend \DIFaddbegin \DIFadd{12.03 }\DIFaddend \cdot x \DIFdelbegin \DIFdel{+ -0.002981 }\DIFdelend \DIFaddbegin \DIFadd{-0.002977 }\DIFaddend \cdot x^2 \label{eq:AODchange}
\end{equation} where \DIFdelbegin \DIFdel{\(y\) is the date as a decimal fraction of the
year}\DIFdelend \DIFaddbegin \DIFadd{\(x\) the year as a continuous variable}\DIFaddend . The
resulting change in \(\text{GHI}_\text{ref}\) ranges between
\DIFdelbegin \DIFdel{\(-1.88\) }\DIFdelend \DIFaddbegin \DIFadd{\(-1.88\,\%\) }\DIFaddend in 1994 and \(-0.19\,\%\) in the end of 2023, peaking at
\(0.23\,\%\) in the mid-2020. Finally, we applied these long-term
changes of Equation\nobreakspace{}\ref{eq:AODchange} to the
\(\text{GHI}_\text{ref}\), to create a more realistic representation of
the cloud-free irradiance for the entire period of study. For the period
1994 -- 1996 where no AOD data are available, we assumed that the
changes in GHI follow the same polynomial fit.

\hypertarget{criteria-for-the-identification-of-ce-events}{%
\subsection{Criteria for the identification of CE
events}\label{criteria-for-the-identification-of-ce-events}}

One of the goals of this study was to quantify the OI related to CEs. A
key issue for achieving this goal is to define a threshold for the CE
identification, representative of the cloud-free irradiance at the time
of each GHI measurement. This depends on the selection of the
appropriate \DIFdelbegin \DIFdel{atmospheric parameterization }\DIFdelend \DIFaddbegin \DIFadd{values of the input parameters }\DIFaddend for the RTM simulations. The
implementation of the long-term changes in GHI due to AOD, discussed in
section\nobreakspace{}\ref{rtmcs}, allows capturing a large part of the
natural variability of cloud-free GHI. However, the short-term
variability of AOD cannot be taken adequately into account when using
monthly values in the model simulations. We tried different approaches
to strengthen the robustness of the methodology and to compensate for
the limited accuracy of the RTM input data and the unpredictable natural
variability of the atmosphere.

First, we evaluated the performance of the modeled
\(\text{GHI}_\text{ref}\) in relation to the measured GHI for different
levels of atmospheric clearness, by using in the RTM the monthly
climatological AOD and WC, less their respective standard deviations.
These values represent typical atmospheres in Thessaloniki with
lower-than-average load of aerosols and humidity, which are the main
factors that attenuate the GHI, in the absence of clouds. With this
approach, the simulated \(\text{GHI}_\text{ref}\) should generally be
greater than the measured GHI when aerosols are more abundant. The
correlation of the \(\text{GHI}_\text{ref}\) with the GHI was tested for
a subset of 513 cloud-free days with more than \(80\,\%\) data
availability (Figure\nobreakspace{}\ref{fig:validation-GHI}). The
scatter plot shows a good agreement between the modeled and the measured
GHI with a spread that arises mainly from the short-term variability of
AOD and WC. The linear regression reveals a positive bias of
\(18.4\,\text{W}/\text{m}^2\) and a slope very close to unity. To
achieve a robust and clear distinction of the CE cases, we applied to
the \(\text{GHI}_\text{ref}\) an additional offset of
\(25\,\text{W}/\text{m}^2\). \DIFaddbegin \DIFadd{We also calculated the trend of the
relative different of the measured GHI over the
\(\text{GHI}_\text{ref}\) for the same cloud-free days, and found a
change of \(0.03\,\%\) with a strong statistical significance. This
further assures that the produced reference can adequate describe the
long term variability of the observed GHI.
}\DIFaddend 

\begin{figure}[H]

{\centering \includegraphics[width=1\linewidth]{../images/P-validation-cloudfree-GHI-1} 

}

\caption{Scatter plot of the measured GHI and the $\text{GHI}_\text{ref}$, for cloud-free days with data availability of more than $80\,\%$. The linear regression on the data is also shown on the figure with the blue line.}\label{fig:validation-GHI}
\end{figure}

To further ensure the effectiveness of this CE criterion, we inspected
manually the CE identification results on selected days of the whole
dataset. We tested seven random groups of about 20 -- 30 days with the
following characteristics: (a) the strongest OI CE events, (b) the
largest daily total OI, (c) absence of clouds (by implementing a cloud
identification algorithm as discussed in \citet{Natsis2023}), (d)
absence of clouds and absence of CE events, (e) with at least \(60\,\%\)
of the day length without clouds and presence of CE events, (h) randomly
selected days, and (i) manually selected days. For the latter case and
where needed, we also used images from the sky-camera to further aid the
inspection.

The definition of the CE events with this method has a degree of
subjectivity, since the actual cloud-free irradiance is not known and
can only be approximated. However, this method was proven capable in
detecting all major CE events. Where some CE events with very low OI may
be not detected, these were few with small over-irradiance, and it is
unlikely that will significantly affect our results.

A subcategory of the CE events that is often discussed in the literature
\citep{Cordero2023, Martins2022, Yordanov2015}, are the extreme cloud
enhancement events. These are cases of CE where the measured intensity
of the irradiance exceeds the TSI on a horizontal plane at TOA. In this
case the threshold \(E\) is given by: \begin{equation}
\text{ECE}: E > \cos(\text{SZA}) \cdot E_{\odot} \frac{r^2_\text{m}}{r^2}
\label{eq:ECE}
\end{equation} where: \(\text{SZA}\) the solar zenith angle,
\(E_{\odot}\) is the solar constant at the mean Sun -- Earth distance
(\(1367\,\text{W}/\text{m}^2\)), \(r\) is the actual Sun - Earth
distance and \(r_\text{m}\) is the mean Sun -- Earth distance of
\(1.496\times10^8\,\text{km}\).

An example of the identification of CE events for a selected day is
given in the Figure\nobreakspace{}\ref{fig:example-day}, where the daily
course of the cloud-free reference irradiance and the CE and ECE
thresholds are shown along with the actual GHI measurements. Three
sky-camera images at the bottom illustrate the cloud conditions during
the highest ECE events (a and c), while (b) shows the conditions where
clouds caused a substantial reduction in GHI. The GHI data in the
periods 7:30 -- 8:30 and after 14:00 are very close to the modeled
cloud-free GHI (purple line) thus were identified as cloud-free
instances. There is some ambiguity for the data points lying between the
CE threshold and the cloud-free irradiance, which are not identified as
CE events. These data points correspond to cases either with AOD below
the one assumed in the model, or with very small OI.

In addition, we provide in Figure\nobreakspace{}\ref{fig:example-year}
an example scatter plot between the measured GHI and the modeled
cloud-free irradiance for year 2005, where the CE and ECE events are
clearly grouped above the cloud-free irradiance (approximated with the
green line). The gray colored data points above this line correspond to
the ambiguous data points discussed above. Finally, the black belt just
below the green line is formed from data measured mainly under
cloud-free conditions, with the spread likely caused by the short-term
variability of AOD. A small fraction of these data corresponds also to
cases with thin cirrus clouds causing weak attenuation of GHI, often
indistinguishable from the attenuation by aerosols.

\begin{figure}[H]

{\centering \includegraphics[width=1\linewidth]{../images/P-example-day-3} 

}

\caption{Example of CE identification in Thessaloniki for 2019-07-11. The green line with blue symbols depicts the measured GHI in one-minute steps. The red line shows the modeled threshold for the detection of CE events, which are denoted with red circles. The black line represents the TOA TSI on a horizontal plane, equivalent to the threshold for the identification of ECE events, shown with magenta circles. The purple line is the modeled cloud-free irradiance. The dark yellow line is the solar constant of $1367\,\text{W}/\text{m}^{2}$. The three sky images below the figure are taken at three characteristic instances (a, b, c) denoted on the figure with vertical orange dashed lines.}\label{fig:example-day}
\end{figure}

\begin{figure}[H]

{\centering \includegraphics[width=1\linewidth]{../images/P-example-years-12} 

}

\caption{Example scatter plot of the measured GHI and the reference cloud-free irradiance in Thessaloniki for the year 2005. The over-irradiance for CE and ECE events is color coded, while the remaining data points are shown in black. The reference green line goes through the origin with a slope of unity.}\label{fig:example-year}
\end{figure}

\hypertarget{results}{%
\section{Results}\label{results}}

Following the application of the above discussed methodology to the
entire dataset (\(>6\) million of one-minute GHI measurements),
\(2.32\,\%\) were identified as CE events and \(0.036\,\%\) as ECE
events. The highest recorded GHI due to CE was
\(1416.6\,\text{W}/\text{m}^2\) on 24~May 2007 at a SZA of
\(19.9^\circ\) corresponding to OI of \(369.3\,\text{W}/\text{m}^2\) or
\(35.3\,\%\) above the threshold. The strongest CE event of \(53\,\%\)
above the cloud-free threshold was observed on 28~October 2016 at a SZA
of \(59.2^\circ\) with a GHI of \(861.8\,\text{W}/\text{m}^2\) and a OI
of \(298.4\,\text{W}/\text{m}^2\). Both cases are also ECE events with
\(131.2\,\text{W}/\text{m}^2\) and \(161.5\,\text{W}/\text{m}^2\) above
the TSI at TOA for the same SZA, respectively. In the following sections
we are discussing the long-term trends and variability of the CE events
as well as of the corresponding OI and excess irradiation.

\hypertarget{long-term-trends}{%
\subsection{Long-term trends}\label{long-term-trends}}

One aspect of this study is to investigate the time evolution of the CE
events by analyzing the GHI measurements at Thessaloniki. Cloud
enhancements can be influenced by different factors, such as the
geometry, size and optical thickness of clouds, their height in the
atmosphere and local weather regimes
\citep{Mol2023, Veerman2022, Gristey2022, Tzoumanikas2016}. Some of
these factors are related to changes in climate; hence it would be
reasonable to expect contributing to the frequency of occurrence of CE
events over Thessaloniki, as well as to the average OI and excess
irradiation. The long-term trends were calculated using a first-order
autoregressive model with the `maximum likelihood' fitting method
\citep{Gardner1980, Jones1980}, by implementing the function `arima'
from the library `stats' of the R programming language \citep{RCT2023}.
All trends are reported together with their \(2\sigma\) error.

Figure\nobreakspace{}\ref{fig:P-energy} shows the time series of the
yearly number of CE cases (each with duration of one minute), the yearly
mean OI and the yearly excess irradiation for the period 1994 -- 2023,
together with corresponding linear trends. To account for missing data,
all three quantities have been divided with the fraction of the valid
GHI observations in each month. Statistically significant (at the
\(95\,\%\) confidence level) increasing trends appear for the yearly
number of CE events (\(+112\pm 35\,\text{cases}/\text{year}\)) and the
excess irradiation (\DIFdelbegin \DIFdel{\(+329.9 \pm 112\,\text{kJ}/\text{year}\)}\DIFdelend \DIFaddbegin \DIFadd{\(+329.9\pm 112.0\,\text{kJ}/\text{year}\)}\DIFaddend ), while
the trend in the yearly mean OI is negligible
(\(+0.1\pm 0.2\,\text{W}/\text{m}^2/\text{year}\)) and of no statistical
significance. The average OI for the entire period is
\(+42.6\pm 2.8\,\text{W}/\text{m}^2\). The yearly excess irradiation due
to CE events ranges between about \(8\) and
\(24\,\text{MJ}/\text{m}^2\). On average, it is about one half of the
highest daily irradiation of \textasciitilde31 MJ/m\textsuperscript{2}
under cloud-free conditions recorded at the summer solstice in
Thessaloniki. Therefore, from solar energy perspective the yearly excess
irradiation is only a tiny fraction of the available solar energy.
Although the interannual variability of the mean OI is rather weak, the
variability about the trend lines is quite large for the number of CE
cases and the excess irradiation (panels b and c). For these two
quantities the spread tends to increase with time, suggesting a
significant variability in cloud patterns over the area, possibly
associated with changes in climate.

We must note that the excess irradiation from the CE events that is
received at the surface does not affect the balance of the energy budget
in the atmosphere. The incoming solar radiation in the region is not
increased by the OI but is rather redistributed at the specific location
as additional scattered radiation. This is also depicted by the ECE
irradiance values, which exceed the equivalent cloud-free irradiance by
a significant amount.

\begin{figure}

{\centering \includegraphics[width=1\linewidth]{../images/P-energy-complete-multi-3} 

}

\caption{Time-series of (a) the yearly CE number of occurrences, (b) the yearly mean OI and (c) the yearly excess irradiation at Thessaloniki for the period 1994 – 2023. The black lines represent the linear trends on the yearly data.}\label{fig:P-energy}
\end{figure}

\hypertarget{climatology-of-cloud-enhancement-events}{%
\subsection{Climatology of cloud enhancement
events}\label{climatology-of-cloud-enhancement-events}}

Next, we investigated the distribution of the CE events within the year.
Figure\nobreakspace{}\ref{fig:relative-month-occurrences} shows the
monthly box and whisker plot of the number of occurrences of the CE
events in percent, calculated relative to the available one-minute data
in each month. Although CE events are present throughout the year,
showing a clear seasonal cycle, the most active months are May and June,
with their median representing about \(3\,\%\) of the data. The least
frequent events (about \(1\,\%\)) occur in August and during winter
(December -- February). This seasonality is a combined effect of
different factors, among them the types of clouds, their frequency of
occurrence, the seasonally varying relative position of the sun, as well
as the local landscape characteristics that may influence the formation
of the clouds. For example, during May and June the frequent formation
of cumulus clouds in the area leads to a significant increase in the CE
events. However, the lack of detailed data on cloud characteristics in
the area does not allow further analysis. Finally, the interannual
variability of the monthly CE events is quite high as manifested by the
size of the boxes and the large monthly extremes, especially in the
summer. Investigation of the causes of this variability is beyond the
scope of this study. It could possibly be related to the cloud cover
variability in winter and autumn over southern Europe, associated with
the North Atlantic Oscillation (NAO) circulation \citep{Chiacchio2010},
or to the observed decreasing trend in cloud cover as a result of global
warming \citep[e.g.,][]{SanchezLorenzo2017}.

\begin{figure}

{\centering \includegraphics[width=1\linewidth]{../images/P-CE-climatology-normlz-1} 

}

\caption{Seasonal variability of the relative occurrence of CE events in Thessaloniki for the period 1994 -- 2023, in the form of a box and whisker plot. The box contains the data between the lower $25\,\%$ and the upper $75\,\%$ percentiles, with the thick horizontal line representing the median. The vertical lines (whiskers) extend between the maximum and minimum monthly values.}\label{fig:relative-month-occurrences}
\end{figure}

The distribution of the number of CE events as a function of OI is shown
in Figure\nobreakspace{}\ref{fig:ovir-distribution}. There is an inverse
relationship between the frequency of CE events and OI with an
exponential-like decline. This is expected, as the stronger the CE
events are, the rarer the conditions favoring the occurrence of CE
events. For the majority (over \(62\,\%\)) of the CE events the OI is
below the long-term average of \(42.6\,\text{W}/\text{m}^2\), while
\(9.6\,\%\) of the events correspond to OI larger than
\(100\,\text{W}/\text{m}^2\) and up to the highest value of
\(437\,\text{W}/\text{m}^2\). This distribution is indicative of the
magnitude and the probability of the expected CE events over
Thessaloniki. Similar distribution of CE events, albeit with larger OI
values, has been reported by \citet{Vamvakas2020}, for the city of
Patras. This site is located \(2.5^\circ\) south of Thessaloniki and is
exposed to air masses coming mainly from the eastern Mediterranean,
resulting in different cloud patterns that may affect the
characteristics and magnitude of the CE events.

\begin{figure}

{\centering \includegraphics[width=1\linewidth]{../images/P-relative-distribution-diff-2} 

}

\caption{Relative frequency distribution of CE events in Thessaloniki for the period 1994 -- 2023 as a function of OI. The histogram was split in two plots with different y-axis scales for better readability.}\label{fig:ovir-distribution}
\end{figure}

\hypertarget{duration-of-cloud-enhancement-events}{%
\subsection{Duration of cloud enhancement
events}\label{duration-of-cloud-enhancement-events}}

The duration of the CE events is variable and can last for several
minutes or even more than an hour. To study the characteristics of these
consecutive events, we grouped them into bins of increasing duration in
steps of one minute. We have identified 34849 groups of CE events in the
whole period of study, where the group with the longest duration of 140
minutes occurred on 07~July 2013 with SZA ranging between \(52.1^\circ\)
and \(77.9^\circ\).
Figure\nobreakspace{}\ref{fig:ceg-duration-distribution} shows the
frequency distribution of the CE events according to their duration. We
conclude that although some groups of events last for more than an hour,
about \(79\,\%\) have a duration of equal or less than 5 minutes, likely
due to the movements of clouds causing frequent blockings of the direct
irradiance, which reduce the duration of the CE events.

\begin{figure}

{\centering \includegraphics[width=1\linewidth]{../images/groups-7} 

}

\caption{Relative frequency distribution of CE groups of consequent CE events according to their duration for Thessaloniki in the period 1994 -- 2023. The histogram was split in two plots with different y-axis scales for better readability.}\label{fig:ceg-duration-distribution}
\end{figure}

The relation between the duration and the mean OI of the CE groups is
shown in Figure\nobreakspace{}\ref{fig:group-2d}. As expected, events
with small duration and low OI are accumulated close to the origin of
the plot. Events of high excess irradiation have a short duration and
vice versa. The vast majority of grouped events are associated with
small excess irradiation (\(<5\,\text{kJ}/\text{m}^2\)) and short
duration (\(<5\,\text{min}\)). For example, the excess irradiation of
\(5\,\text{kJ}/\text{m}^2\) for \(5\,\text{min}\) is equivalent to an OI
of \(16.7\,\text{W}/\text{m}^2\) with duration of \(5\,\text{min}\). On
the contrary, groups with strong excess irradiation and long duration
are very rare. Similar results of this relation have been reported by
\citet{Zhang2018} in a study using a far higher sampling rate
(\(100\,\text{Hz}\)) than ours.

\begin{figure}

{\centering \includegraphics[width=1\linewidth]{../images/P-groups-bin2d-1} 

}

\caption{Relation of excess irradiation of CE groups with their duration for Thessaloniki in the period 1994 -- 2023. The logarithmic color scale denotes the frequency of the respective groups. Note that 19 \DIFdelbeginFL \DIFdelFL{pints }\DIFdelendFL \DIFaddbeginFL \DIFaddFL{points of groups }\DIFaddendFL with duration higher than 60 min are not shown.}\label{fig:group-2d}
\end{figure}

\hypertarget{extreme-cloud-enhancement-events}{%
\subsection{Extreme cloud enhancement
events}\label{extreme-cloud-enhancement-events}}

An aspect of the CE events that is commonly reported and has some
significance on the solar energy production infrastructure are the
extreme CE (ECE) events, where the measured GHI exceeds the expected
irradiance at the TOA at the same SZA
(Equation\nobreakspace{}\ref{eq:ECE}). In analogy to
Figure\nobreakspace{}\ref{fig:relative-month-occurrences}, we show in
Figure\nobreakspace{}\ref{fig:relative-month-occurancies-ECE} the number
of ECE events relative to the available number of one-min observations
in each month. Judging by the distribution of the monthly medians, the
most active months in ECE events are January -- March and November and
the least active are July and August. The highest interannual
variability in the 30-year record occurs from October to April. In the
summer period, the fraction of ECE events is very small, with almost
zero variability in July and August. Although the most active months for
CE events are May and June
(Figure\nobreakspace{}\ref{fig:relative-month-occurrences}), these
months behave differently for the ECE events, suggesting that the
cumulus clouds that usually cause enhancements of irradiance are
incapable to produce extreme events, possibly because of their size and
structure, in conjunction with the smaller SZAs in the summer. In
contrast, during winter the Sun is lower in the sky and clouds are
frequently less dense allowing a large fraction of radiation to be
scattered forward and enhance the irradiance at the surface. These
conclusions are based on manual inspection of many sky-camera images
during the summer and winter periods. An automated procedure of checking
the sky-camera images for such events would greatly improve our ability
to classify the CE and ECE events and help in the attribution of the
causes of the events.

In general, the fraction of ECE events during the year is very small,
well below \(0.1\,\%\) of the data, therefore on average it would not
affect the production of solar energy. However, during short, isolated
periods extreme enhancements of GHI can be of concern.

\begin{figure}

{\centering \includegraphics[width=1\linewidth]{../images/P-ECE-climatology-normlz-1} 

}

\caption{Seasonal variability of the relative occurrence of ECE events in Thessaloniki for the period 1994 -- 2023, in the form of a box and whisker plot. The box contains the data between the lower $25\,\%$ and the upper $75\,\%$ percentiles, with the thick horizontal line representing the median. The vertical lines (whiskers) extend between the maximum and minimum monthly values.}\label{fig:relative-month-occurancies-ECE}
\end{figure}

The distribution of the ECE events
(Figure\nobreakspace{}\ref{fig:P-extreme-distribution}) shows that in
rare cases the measured GHI exceeds the TOA TSI at the same SZA even by
more than \(200\,\text{W}/\text{m}^2\), while in \(75\,\%\) of the cases
the OI is below \(40\,\text{W}/\text{m}^2\). The \(50\,\%\) OI of the
most frequent ECEs cases ranges between \(127\) and
\(211\,\text{W}/\text{m}^2\). These findings are in accordance with the
results of \citet{Vamvakas2020}, the only difference being that the OI
values reported are higher than those for Thessaloniki.

\begin{figure}

{\centering \includegraphics[width=1\linewidth]{../images/P-extreme-distribution-2} 

}

\caption{Relative frequency distribution of ECE events, in Thessaloniki for the period 1994 -- 2023, as a function of the extra irradiance over TSI at TOA for the same SZA.}\label{fig:P-extreme-distribution}
\end{figure}

\hypertarget{conclusions}{%
\section{Conclusions}\label{conclusions}}

By creating a cloud-free approximation representing the long- and
short-term variability of the expected cloud-free GHI, we were able to
identify cases of CE and ECE events in Thessaloniki for the period 1994
-- 2023. After analyzing the CE cases, we found an increasing tendency
of \(+112.1\pm 34.9\,\text{cases}/\text{year}\), with the mean annual
irradiation of the CE events increasing with a rate of
\DIFdelbegin \DIFdel{\(+329.9\pm 112\,\text{kJ}/\text{year}\)}\DIFdelend \DIFaddbegin \DIFadd{\(+329.9\pm 112.0\,\text{kJ}/\text{year}\)}\DIFaddend . The most active months of CE
events are May and June. We found that continuous CE events can last up
to \(140\) minutes, while the duration of \(79\,\%\) of them is equal or
bellow \(5\) minutes.

We have observed ECE cases with GHI exceeding the TOA TSI on a
horizontal plane by more than \(200\,\text{W}/\text{m}^{2}\) while for
\(75\,\%\) of the cases the excess irradiance (relative to TOA TSI on a
horizontal plane) is below \(40\,\text{W}/\text{m}^{2}\). The
climatological characteristics of the ECE events showed that the most
active months are spread over half of the year, and particularly in the
months January -- March and November, while ECE events are practically
absent in July and August. The magnitude of the ECE events identified in
Thessaloniki does not exceed the values reported for sites with more
favorable conditions for the phenomenon \citep[e.g.,][]{Cordero2023}.
Some of the characteristics of CE and ECE events we analyzed have strong
similarities with the results of \citet{Vamvakas2020} for the city of
Patras, south of Thessaloniki, albeit with differences in the magnitude
of OI.

The investigation of the CE trends in Thessaloniki through this 30-year
period shows that the interaction of GHI with clouds is a complex
phenomenon that merits further attention. For such studies, it would be
essential to have more detailed information on cloud characteristics,
especially in order to investigate possible associations of the observed
trends with changes in climate. Finally, automated procedures to analyze
the sky-camera images would greatly improve our knowledge about the
generation of these events.

\hypertarget{reproducibility}{%
\section*{Reproducibility}\label{reproducibility}}
\addcontentsline{toc}{section}{Reproducibility}

The GHI data are available as hourly values in the WMO World Radiation
Data Center\footnote{\url{http://wrdc.mgo.rssi.ru}}.The relevant source
code is available in three repositories for (a) the preparation of GHI
data \citep{natsis_2024_13744467} (b) the detection of clouds
\citep{thanasis_n_2024_13744460} and (c) the processes for this
manuscript \citep{thanasis_n_2024_13744472}.

\bibliography{bibliography.bib}


\end{document}
